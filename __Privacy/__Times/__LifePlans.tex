\documentclass[UTF8,a4paper,8pt]{ctexbook} 

\usepackage{graphicx}%学习插入图
\usepackage{verbatim}%学习注释多行
\usepackage{booktabs}%表格
\usepackage{geometry}%图片
\usepackage{amsmath}
\usepackage{amssymb}
\usepackage{listings}%代码
\usepackage{xcolor}  %颜色
\usepackage{enumitem}%列表格式
\usepackage{tcolorbox}
\usepackage{algorithm}  %format of the algorithm
\usepackage{algorithmic}%format of the algorithm
\usepackage{multirow}   %multirow for format of table
\usepackage{tabularx} 	%表格排版格式控制
\usepackage{array}	%表格排版格式控制
\usepackage{hyperref}
\usepackage{dirtree}

\CTEXsetup[format+={\flushleft}]{section}

%%%%%% 设置字号 %%%%%%
\newcommand{\chuhao}{\fontsize{42pt}{\baselineskip}\selectfont}
\newcommand{\xiaochuhao}{\fontsize{36pt}{\baselineskip}\selectfont}
\newcommand{\yihao}{\fontsize{28pt}{\baselineskip}\selectfont}
\newcommand{\erhao}{\fontsize{21pt}{\baselineskip}\selectfont}
\newcommand{\xiaoerhao}{\fontsize{18pt}{\baselineskip}\selectfont}
\newcommand{\sanhao}{\fontsize{15.75pt}{\baselineskip}\selectfont}
\newcommand{\sihao}{\fontsize{14pt}{\baselineskip}\selectfont}
\newcommand{\xiaosihao}{\fontsize{12pt}{\baselineskip}\selectfont}
\newcommand{\wuhao}{\fontsize{10.5pt}{\baselineskip}\selectfont}
\newcommand{\xiaowuhao}{\fontsize{9pt}{\baselineskip}\selectfont}
\newcommand{\liuhao}{\fontsize{7.875pt}{\baselineskip}\selectfont}
\newcommand{\qihao}{\fontsize{5.25pt}{\baselineskip}\selectfont}
  
%设置文章宽度
\geometry{textwidth=18cm}
%设置页面布局
\pagestyle{plain}
\author{郑华}
\title{Life PlanS}

 %正文排版开始
 \begin{document} 
 	\maketitle
	\tableofcontents
	
\chapter{习惯}
	\section{必须养成的}
		\paragraph{生活习惯}
			\begin{enumerate}
				\item 早起 \ 7:20 \ 天亮就起床,醒了之后绝不贪恋
				\item 早睡 \ 11:20
				\item 看书 \ 每天至少半小时
				\item 学习Linux  \ 每天10分钟
				%\item 学习英语 \ 每天10分钟
				
				\item 积累话题 \ 每天整理一个 \ 自己尝试表达 \ \textbf{每天记《茶余偶谈》一则,分德行门、学问门、经济门、艺术门等类}
				\item 完成清单 \ 每天完成清单内容,前天定义清单
				
				\item 锻炼身体 \ 每周2次-3次瑜伽
				\item 记日记 \ 每天\textbf{记自己行为不端之处,言语过失之处},终身不间断
				\item 查看日记,自我反思 \ 每周一次
			\end{enumerate}
			
		\paragraph{学习习惯}
			\begin{enumerate}
				\item 读书不二 \ 一本书没有点读完毕,一定不看他书;\textbf{东翻西阅,都是为外界所左的人}
				\item 备份资料 \ 以防失误
				\item 记录过程 \ 便于总结反思
			\end{enumerate}
			
		\paragraph{交际习惯}
			\begin{enumerate}
				\item 谨言 \ 说话时时刻刻都要小心留意
				\item 说话前 \ 换位思考下先: 己所不欲勿施于人。
				\item 谢谢不能少..
			\end{enumerate}


\chapter{求学}
	\paragraph{读书须有恒心}
		\begin{itemize}
			\item \textbf{学问之道无穷,而总以有恒为主},兄往年极无恒,近年略好,而犹未纯熟。自七月初一起,至今则无一日间断,每日临帖百字,抄书百字,看书少须满二十页,多则不论。
		
			\item \textbf{虽极忙,亦须了本日功课,不以昨日耽搁,而今日补做,不以明日有事,而今日预做}。诸弟若能有恒如此,则虽四弟中等之资,亦当有所成就。
		
			\item \textbf{切勿以家中有事,而间断看书之事,又勿以考试将近,而间断看书之课。}虽走路之日,到店亦可看,考试之日,出场亦可看也。兄日夜悬望,独此有恒二字告诸弟,伏愿诸弟刻刻留心。
		\end{itemize}
	\paragraph{学业宜精}
		\begin{itemize}
			\item 些小得失不足为患,\textbf{特患业之不精耳}。
		\end{itemize}
		
	\paragraph{求学之法}
		\begin{itemize}
			\item “\textbf{为学壁如熬肉,先须用猛火煮,然后用漫火温。}”予生平工夫,全未用猛火煮过,虽有见识,乃是从悟境得来,偶用工亦不过优游玩索②已耳,如未沸之汤,遽用漫火温之,将愈翥愈不熟也。以是急思般进城内,以是急思搬进城内,屏除一切,从事于克己之学。
			
			\item \textbf{用功譬若掘井,与春多掘数井,而皆不及泉,何若老衬一井,力求及泉而用之不竭乎广此语正与予病相合,盖予所谓掘井而皆不及泉者且。}
			
			\item 至于\textbf{修业以卫身},吾请言之。\textbf{卫身莫大如谋食},农工商劳力以求食者也,士劳心以求食者也。故或食禄于朝,教授于乡,或为传食之客,或为入幕之宾,皆须计其所业,足以得食而无愧。科名,食禄之阶也,亦须计吾所业,将来不至尸位素餐,而后得科名而无愧,\textbf{食之得不得,究通由天作主,予夺由人作主,业之精不精,由我作主}。
			
			\item \textbf{然吾未见业果精而终不得食者也,农果力耕,虽有饥馑,必有丰年;商果积货,虽有雍滞,必有通时;士果能精其业,安见其终不得科名哉?即终不得科名,又岂无他途可以求食者哉?然则特患业之不精耳。求业之精,别无他法,曰专而已矣。谚曰:“艺多不养身,谓不专也。”吾掘井多而无泉可饮,不专之咎也!}
			
			\item \textbf{诸弟总须力图专业},\textit{如九弟志在习字,亦不尽废他业};但每日习安工夫,不可不提起精神,随时随事,皆可触悟。四弟六弟,吾不知其心有专嗜否?\textbf{若志在穷经,则须专守一经,志在作制义,则须专看一家文稿,志在作古文,则须专看一家文集}。作各体诗亦然,作试帖亦然,\textbf{万不可以兼营并鹜},兼营则必一无所能矣。切嘱切嘱!
		\end{itemize}
	
	\paragraph{自立课程}
		\begin{itemize}
			\item 余自十月初一立志自新以来,虽懒惰如故,\textbf{而每日楷书写日记,每日读史十页,每日记《茶余偶谈》一则,此三事未尝一日间断}。十月廿一日\textbf{誓永戒吃水烟,洎已两月不吃烟,已习惯成自然矣}。予自立课程甚多,惟记《茶余偶谈》、读史十页、写日记楷本,此三事者誓终身不间断也。\textbf{诸弟每日自立课程,必须有日日不断之功,虽行船走路,俱须带在身边。予除此三事外,他课程不必能有成,而此三事者,将终身以之。}
			
			\item \textbf{盖求友以匡己之不逮,此大益也};标榜以盗虚名,是大损也。\textbf{天下有益之事,即有足损者寓乎其中,不可不辨}。
			\item 言有矩,动有法
			
			\item 盖士人读书,\textbf{第一要有志,第二要有识,第三要有恒}。\textit{有志则断不敢为下流;有识则知学问无尽,不敢以一得自足,如河伯之观海,如井蛙之窥天,皆无识者也;有恒则断无不成之事}:此三者缺一不可。
		\end{itemize}
		曾国藩课程:
		\begin{enumerate}
			\item \textbf{主敬}——整齐严肃、无时不俱。无事时心在腔子里,应事时专一不杂。
			\item \textbf{静坐}——每日不拘何时,静坐一会,体验静极生阳来复之仁心。正位凝命,如鼎之镇
			\item \textbf{早起}——黎明即起,醒后勿沾恋
			\item \textbf{读书不二}——一书未点完,断不看他书;东翻西阅,都是徇外为人
			\item \textbf{读史}——二十三史每日读十页,虽有事不间断
			\item \textbf{写日记}——须端楷,凡日间过恶,身过、心过、口过皆己出,终身不间断
			\item \textbf{日知其所亡}——每日记《茶余偶谈》一则,分德行门、学问门、经济门、艺术门
			\item \textbf{月无忘所能}——每月作诗文数首,以验积理之多寡,养气之盛否
			\item \textbf{谨言}——刻刻留心
			\item \textbf{养气}——无不可对人言之事。气藏丹田。
			\item \textbf{保身}——谨遵大人手谕,节欲、节劳、节饮食
			\item \textbf{作字}——早饭后作字,凡笔墨应酬,当作自己功课。
		\end{enumerate}
		

	\paragraph{学习四法}
		\begin{itemize}
			\item \textbf{一曰看生书宜求速,不多读则太陋}
			\item \textbf{一曰温旧书宜求熟,不背诵则易忘}
			\item \textbf{一曰习字宜有恒,不善写则如身之无衣,山之无木},对应于今天的口才
			\item \textbf{一曰作文宜苦思,不善作则如人之哑不能言,马之肢不能行},对应于本人的程序
		\end{itemize}
\chapter{修身}    
    \section{制定一份自己的准则}
	    \paragraph{专注} 
		    \begin{itemize}
		    	\item 凡全副精神专注一事,终身必有成就
		    \end{itemize}
	    
	    \paragraph{戒-懒傲}
		    \begin{itemize}
		    	\item 古今之庸人,\textbf{皆以一惰字致败};古今之人才,\textbf{皆以一傲字致败}	
		    \end{itemize}
		        
	    \paragraph{说到做到}
		    \begin{itemize}
		    	\item 少说,但说到一定要做到。
		    \end{itemize}
	     
	    \paragraph{少找借口} 
		    \begin{itemize}
		    	\item 凡事找借口,会造成别人感觉此人不可靠。
		    \end{itemize}
	    
	    \paragraph{诚信}
		    \begin{itemize}
		    	\item  诚信会给人与好感,\textbf{真诚可交}。
		    \end{itemize}
	    
		\paragraph{糊涂}
			\begin{itemize}
				\item 【难得糊涂】孔子发现了糊涂,取名中庸;老子发现了糊涂,取名无为;庄子发现了糊涂,取名逍遥;墨子发现了糊涂,取名非攻;如来发现了糊涂,取名忘我。世间万事惟糊涂难也。有些事,问清楚便是无趣,连佛都说,\textbf{人不可太尽,事不可太尽,凡事太尽,缘份势必早尽}。所以有时候,难得糊涂才是上道 ​​
			\end{itemize}

		 \paragraph{自我反思} 
			 \begin{itemize}
			 	\item 男人前于过世,每自忽略,自十月以来,\textbf{念念改过,虽小必惩}.
			 \end{itemize}
		 
		 \paragraph{诚者}
			 \begin{itemize}
			 	\item 知一事便行一事
			 	\item 十一月信言:观看《庄子》并《史记》,甚善!\textbf{但作事必须有恒,不可谓考试在即便将之书丢下,必须从首至尾句句看完}。若能明年将《史记》看完,则以后看书不可限量,不必问进学与否也。贤弟\textbf{论}袁诗,\textbf{论}作字,亦皆有所见;\textbf{然空言无益,须多做诗,多临帖乃可谈耳}。\textbf{譬如人欲进京一步不行,而在家空言进京程途,亦向益哉?即言之津津,人谁得而信之哉?}
			 \end{itemize}
			  		 
		 \paragraph{平易近人}
			 \begin{itemize}
			 	\item \textbf{但不能庄严威厉,使人望若神明耳}..(但为人不能太严肃厉害,使人像望着神明一样..)
			 \end{itemize}

		 \paragraph{去 傲}
			 \begin{itemize}
			 	\item  傲气既长,终不进功,所以潦倒一生,而无寸进也。
			 	\item  余平生科名极为顺遂,惟小考七次始售。\textbf{然每次不进,未尝敢出一怨言,但深愧自己试场之诗文太丑而已}。至今思之,如芒在背。当时之不敢怨言,诸弟问父亲、叔父及朱尧阶便知。盖场屋之中,\textbf{只有文丑而侥幸者,断无文佳而埋没者,此一定之理也}
			 	\item  故\textbf{吾人用功,力除傲气,力戒自满,毋为人所冷笑,乃有进步也}
			 	\item  弟累年\textbf{小试不售},\textit{恐因愤激之久,致生骄惰之气,故特作书戒之}
			 \end{itemize}
		  
		 \paragraph{去 牢骚}
			 \begin{itemize}
			 	\item 盖植弟今年一病,百事荒废,场中之患目疾,自难见长。\textbf{温弟天分,本甲于诸弟,惟牢骚太多,性情太懒,前在京华,不好看书,又不作文,余即心甚忧之}。近闻还家后,亦复牢骚如常,或数月不搦管为文。吾家之无人继起,诸弟犹可稍宽其责,\textbf{温弟则实自弃,不昨尽诿其咎于命运}。
			 	\item \textbf{吾尝见朋友不中牢骚太甚者,其后必多抑塞},如吴(木云)台凌荻舟之流,指不胜屈。\textbf{盖无故而怨天,则天必不许,无故而尤天,则天必不许,无故而尤人,则人必不服,感应之理,自然随之}。温弟所处,乃读书人中最顺之境,乃动则怨尤满腹,百不如意,实我之所不解。以后务宜力除此病,以吴(木云)台凌荻舟为眼前之大戒。\textbf{凡遇牢骚欲发之时,则反躬自思,吾果有何不足,而蓄此不平之气,猛然内省,决然去之。不惟平心谦抑,可以早得科名,亦一养此和气,可以稍减病患}。万望温弟再三细想,勿以吾言为老生常谈,不直一哂②也。
			 \end{itemize}
			 
		 \paragraph{去傲 去多言}
			 \begin{itemize}
			 	\item 古来言\textbf{凶德致败者约有二端}:曰长傲,曰多言。 \textbf{凡傲之凌物,不必定以言语加人,有以神气凌之者矣,有以面色凌之者矣}。 温弟之神气稍有英发之姿,面色间有蛮横之象,最易凌人。
			 \end{itemize}
		
		 \paragraph{不宜非宜讥笑他人}
			 \begin{itemize}
			 	\item 谚云:"富家子弟多娇,贵家子弟多傲。” 非比锦衣玉石,动手打人,而后谓之骄傲也。但使志得意满,毫无畏忌,开口议人短长,既是极骄极傲耳。
			 	\item 戒骄字,\textbf{以不轻非笑人第一义};
			 	\item 戒傲字,\textbf{以不晏起(早起)为第一义};
			 	\item 古来无与宗族、乡党为仇之圣贤,\textbf{弟辈万不可专责他人也ss}。
			 \end{itemize}

		\paragraph{自知、清、谦、勤}
			\begin{itemize}
				\item  雪琴与沅弟嫌隙已深,难遽期其水乳。沅弟所批雪信稿,有是处,亦有未当处。弟谓雪声色俱厉,\textbf{凡目能见千里,而不能自见其睫,声音笑貌之拒人,每苦于不自见,苦于不自知}。雪之厉,雪不自知;沅之声色,恐亦未始不厉,特不自知耳。
				\item 沅弟昔年于银钱取与之际\textbf{不甚斟酌},朋辈之讥议菲薄,其根实在于此。
				\item 谦字存诸中者不可知,其著于外者约有四端:\textbf{曰面色,曰言语,曰书函,曰仆从属员}。
				\item 每日临睡之时,\textbf{默数本日劳心者几件},劳力者几件,则知宣勤王事之处无多,更竭诚以图之,此劳字工夫也。
			\end{itemize}
		
		\paragraph{爱才为主}
			\begin{itemize}
				\item 既爱其才,宜略其小节,甚是甚是
			\end{itemize}

		\paragraph{只问积劳,不问成名}
			\begin{itemize}
				\item 吾兄弟但在积劳二字上着力,成名二字不必问及,享福二字更不必问。
			\end{itemize}

		\paragraph{自修处求强}
			\begin{itemize}
				\item  	吾辈在自修处求强则可,在胜人处求强则不可。若专在胜人处求强,其能强到底与否尚未可知,即使终身强横安稳,亦君子所不屑道也。
				\item   在自修方面求强是可以的,在与人争胜负时求强就不可以了。如果专门在胜过别人的地方求强,是否能强到底,还不可知,即使终身强横安稳,也是君子所不屑一提的。
			\end{itemize}
			
		\paragraph{时刻悔悟大有进益}
			\begin{itemize}
				\item  兄自问近年得力惟有一\textbf{悔}字诀。\textbf{兄昔年自负本领甚大,可屈可伸,可行可藏,又每见得人家不是。自从丁己、戊午大悔大悟之后,乃知自己全无本领,凡事都见得人家有几分是处},故自戊午至九载,与四十岁以前迥不相同。大约以能\textbf{立能达为体,以不怨不尤为用}。\textit{立者,发奋自强,站得住也;达者,办事圆融,行得通也}。
				\item 吾九年以来,\textbf{痛戒无恒之弊,看书写字,从未间断},选将练兵,亦常留心,此皆自强能立工夫。奏疏公牍,再三斟酌,无—过当之语自夸之词,此皆圆融能达工夫。\textbf{至于怨天本有所不敢,尤人则常不能免,亦皆随时强制而克去之。}
			\end{itemize}
			
		\paragraph{切勿占人便宜}
			\begin{itemize}
				\item \textbf{从前施情于我者,或数百,或数千},皆钓饵也。渠若到任上来,\textbf{不应则失之刻薄,应之则施一报十,尚不足满其欲}。
				
				\item 故自庚子到京以来,于今八年,\textbf{不肯轻受人惠,情愿人占我的便宜,断不肯我占人的便宜},将来若做外官,京城以内,\textbf{无责报于我者}。
				
				\item 澄弟在京年余,亦得略见其概矣,此次澄弟所受各家之情,\textbf{成事不说,以后凡事不可占人半点便宜,不可轻取人财,切记切记}!
			\end{itemize}

\chapter{交友}  
	\section{制定一份自己的交友准则}
		  \subparagraph{观人4法} 讲信用、无官气、有条理、少大话
		  
		  \subparagraph{求友需专一} 既以此附课,则不必送诗文于他处看,以明有所专主也,\textbf{凡事皆贵专},求师不专,则受益也不入,\textbf{求友不专,则博爱而不亲,心有所专宗,而博观他涂以扩其只,亦无不可,无所专宗,而见异思迁,此眩彼夺,则大不可}
		  
		  \subparagraph{亲近良友} 不可不殷勤亲近良友,亲近愈多,获益愈多.
		  
		  \subparagraph{分虽严明而情贵周通} 交友须勤加来往
			
		  \subparagraph{勿有歹心} 切勿占人便宜
			
		  \subparagraph{患难与共} 患难与共 勿有遗憾
		  
		  \subparagraph{有负友定当虚心改之}实使次青难堪,今弟指出,余益觉大负次青,愧悔无地!\textbf{余生平于朋友中,负人甚少,惟负次青实甚,两弟为我设法,有可挽回之处,余不惮改过也}
		  
		 \subparagraph{好夸坏劝}然为兄者,\textbf{总宜奖其所长,而兼规其短,若明知其错,而一概不说,则又非特沅一人之错,而一家之错也}。
		  
    \section{建立一份打算深交和维持关系的名单}
	    \subsection{高中}
			\begin{itemize}
				 \item  刘丹
				 \item  郑杨斌
				 \item  郝楠
				 \item  房文博
				 \item  尹建航
				 \item  曹强
				 \item  折战峰
				 \item  赵文全
				 %\item  王世斌
				 %\item  雷子文
			\end{itemize}
		\subsection{大学}
			\begin{itemize}
				\item  尤文浩
				\item  罗振
				\item  张凌杰
				\item  田宏城
			\end{itemize}
		
		\subsection{研究生}
			\begin{itemize}
				\item  魏拓
				\item  常存宝
				\item  宋泽鲁
				\item  李飞
			\end{itemize}

\chapter{治家}
	\paragraph{分享}
		\begin{itemize}
			\item 予生平伦常中,惟兄弟一伦,抱愧尤深!\textbf{盖父亲以其所知者,尽以教我,而吾不能以吾所知者,尽教诸弟,是不孝之大者也}!
		\end{itemize}

	\paragraph{家和万事兴}
		\begin{itemize}
			\item \textbf{夫家和则福自生},若一家之中兄有言,弟无不从,弟有请,兄无不应,和气蒸帮而家不兴者,未之有也。反是而不败者,亦未之有也。
		\end{itemize}
		
	\paragraph{表率}
		\begin{itemize}
			\item \textbf{然徒谦亦不好,总要努力前进},此全在为兄者倡率之,余他无所取,惟近来日日不恒,\textbf{可为诸弟倡率}。
		\end{itemize}
		
\chapter{踏步-经历}    
    \section{踏步}
	    \paragraph{美国转一圈}
	    \paragraph{英国转一圈}
	    \paragraph{埃及看金字塔}
	    \paragraph{日本看看文化}
	    \paragraph{法国转转}
	    \paragraph{德国转转}
	    \paragraph{俄罗斯转转}
	    \paragraph{韩国转转}
	    \paragraph{非洲部落转转}
	    \paragraph{巴西看瀑布}
	    \paragraph{内蒙古草原溜达}
	    \paragraph{云南丽江}
	    \paragraph{滑滑雪,唠唠嗑}Done
	    \paragraph{香港购个物}
	    \paragraph{澳门赌场小赌一把}
	    \paragraph{重庆\ 转转}
	    
    \section{经历}
	    \paragraph{大胆的尝试、去做}
		   \begin{itemize}
			   	\item 2016-11 主动在饭店里搭讪了一位生活中的女生,并成功在以后的生活中沟通交流
		   \end{itemize}

\chapter{家庭}    
    \section{父母}
	    \begin{enumerate}
	    	\item 一周至少打两个电话
	    	\item 时刻挂念二老的身体
	    \end{enumerate}
    
	\section{妻子儿女}
		\begin{enumerate}
			\item 让他们在平凡的世界里活的有趣
			\item 争吵难免,但一定得处理好、及时沟通,男女思维有别..注意方式
		\end{enumerate}
	
\setlength{\DTbaselineskip}{20pt}
\DTsetlength{1em}{3em}{0.1em}{1pt}{4pt}

\chapter{职业}   
	\section{Linux 网络 服务器主程}
		\begin{itemize}
			\item 算法
			\item Linux
			\item 网络编程
			\item 多线程编程
			\item 服务器架构
			\item 代码设计知识
		\end{itemize}
	
	\clearpage	
	\section{游戏引擎 主程}
		\dirtree{%
			.1 \textbf{游戏职业技术路线规划}.
			.2 \textbf{图形学} .
				.3 Mesh .
				.3 光照 .
				.3 物理 .
				.3 渲染 .
			.2 \textbf{底层API} .
				.3 OpenGL .
				.3 DirectX .
			.2 \textbf{游戏引擎} .
				.3 Unity .
				.3 Unreal .
				.3 NeoX .
			.2 \textbf{计算机基础} .
				.3 数据结构 .
				.3 算法导论 .
				.3 操作系统 -Linux.
				.3 编程语言 .
					.4 C++ .
						.5 Mem管理 .
						.5 STL源码 .
						.5 auto、share\_ptr等的实现原理 .
					.4 Python . 
				.3 设计模式 .
				.3 多线程编程 .
		}
	
	

		
\chapter{事业}   
    \section{饭店}
	    \paragraph{想一个很有钱途的点子并实现 \quad 挣钱为目的好像有点不对}
	    
	\section{理财}
	    \paragraph{学会理财}
	    \paragraph{学会炒股并学会分析市场}
	    
	\section{创业}
	    \paragraph{创办个 农产品 销售平台}替农民们解决产品出路问题..别忙活了一年卖不出去..
   
\chapter{挑战}
	\section{游泳}
		
		
	\section{过山车}
		Half 
		
	\section{跳楼机}    
		Done
		
	\section{大摆锤}
		Done
		
    \section{蹦极}
    
    
    \section*{近期时间表}
     \begin{itemize}
     	\item 8:00\ -\ 8:40 起床吃饭
     	\item 8:50\ -\ 11:30 DirectX
     	\item 11:30-1:20  吃饭 + 睡觉 +洗衣服
    	\item 1:50\ -\ 3:00  MFC
     	\item 3:20\ -\ 5:30  C++
     	\item 5:30\ -\ 7:00  吃饭,拉屎
     	\item 7:30\ -\ 9:00  lua
     	\item 9:00\ -10:00   linux
        \item 10:00-12:00 DirectX / 图形学  + 日记
        \item 0:00\ -\ 0:40 微博
        \item 0:40\ -\ 1:40 读书
        \item Sunday and Saturday night need to take a rest
      \end{itemize} 	
     		    
     		     
\end{document} 
 		    