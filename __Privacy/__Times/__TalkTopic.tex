\documentclass[UTF8,a4paper,12pt]{ctexbook} 

 \usepackage{graphicx}%学习插入图
 \usepackage{verbatim}%学习注释多行
 \usepackage{booktabs}%表格
 \usepackage{geometry}%图片
 \usepackage{amsmath}
 \usepackage{amssymb}
 \usepackage{listings}%代码
 \usepackage{xcolor}  %颜色
 \usepackage{enumitem}%列表格式
 \usepackage{tcolorbox}
 \usepackage{algorithm}  %format of the algorithm
 \usepackage{algorithmic}%format of the algorithm
 \usepackage{multirow}   %multirow for format of table
 \usepackage{tabularx} 	%表格排版格式控制
 \usepackage{array}	%表格排版格式控制
 
 \CTEXsetup[format+={\flushleft}]{section}
  %%%% 下面的命令添加新字体 %%%%%

 %%%%%% 设置字号 %%%%%%
 \newcommand{\chuhao}{\fontsize{42pt}{\baselineskip}\selectfont}
 \newcommand{\xiaochuhao}{\fontsize{36pt}{\baselineskip}\selectfont}
 \newcommand{\yihao}{\fontsize{28pt}{\baselineskip}\selectfont}
 \newcommand{\erhao}{\fontsize{21pt}{\baselineskip}\selectfont}
 \newcommand{\xiaoerhao}{\fontsize{18pt}{\baselineskip}\selectfont}
 \newcommand{\sanhao}{\fontsize{15.75pt}{\baselineskip}\selectfont}
 \newcommand{\sihao}{\fontsize{14pt}{\baselineskip}\selectfont}
 \newcommand{\xiaosihao}{\fontsize{12pt}{\baselineskip}\selectfont}
 \newcommand{\wuhao}{\fontsize{10.5pt}{\baselineskip}\selectfont}
 \newcommand{\xiaowuhao}{\fontsize{9pt}{\baselineskip}\selectfont}
 \newcommand{\liuhao}{\fontsize{7.875pt}{\baselineskip}\selectfont}
 \newcommand{\qihao}{\fontsize{5.25pt}{\baselineskip}\selectfont}
 
 \makeatother
 
 %%%% 段落首行缩进两个字 %%%%
 \makeatletter
 \let\@afterindentfalse\@afterindenttrue
 \@afterindenttrue
 \makeatother
 \setlength{\parindent}{2em}  %中文缩进两个汉字位
 
 
 %%%% 下面的命令重定义页面边距,使其符合中文刊物习惯 %%%%
 \addtolength{\topmargin}{-54pt}
 \setlength{\oddsidemargin}{0.63cm}  % 3.17cm - 1 inch
 \setlength{\evensidemargin}{\oddsidemargin}
 \setlength{\textwidth}{14.66cm}
 \setlength{\textheight}{24.00cm}    % 24.62
 
 %%%% 下面的命令设置行间距与段落间距 %%%%
 \linespread{1.4}
 % \setlength{\parskip}{1ex}
 \setlength{\parskip}{0.5\baselineskip}
 
 \geometry{left=1.6cm,right=1.8cm,top=2cm,bottom=1.7cm} %设置文章宽度
 
 \pagestyle{plain} 		  %设置页面布局

	
 \author{\kaishu 郑华}
 \title{\heiti 话题}
 
\begin{document}          %正文排版开始
 	\maketitle
	\tableofcontents
\chapter{历史}
	\section{人物}
		\subparagraph{宇文泰-一箭多雕}\textbf{宇文泰}向苏绰讨教治国之道,苏答:\textbf{善用贪官},宇又问:那结果出现了民怨太大的官吏怎么办? 苏答:杀贪官,为民伸冤!把他搜刮来的民财放进你的腰包。这样你可以不用背负搜刮民财之名。总之,\textbf{用贪官来培植死党},\textbf{除贪官来消除异己,杀贪官来收买人心,没收贪财来充实自己腰包,这就是玩弄权术的艺术}!

		\subparagraph{纪晓岚}一天,乾隆找纪晓岚对对子。乾隆出的对子听着是:\textbf{两碟豆}。纪晓岚心想:皇上一般都出难句,这次怎如此简单?要小心应对,答道:\textbf{一瓯油}。果然,乾隆改了口:我说的是—— \textbf{两蝶斗}。纪晓岚似乎早已料到,不慌不忙地说:我指的是——\textbf{一鸥游}。皇上又说:\textbf{花间两蝶斗}。纪晓岚随口应道:\textbf{水面一鸥游}。
		
		\subparagraph{唐玄宗}安禄山起兵造反,唐玄宗准备逃亡四川。当晚,玄宗见众人拿着火把等他。玄宗对杨国忠说:用那么多火把做什么?杨国忠答:准备焚毁仓库钱粮,以免被安禄山所得。玄宗严肃地说:安禄山到了,如果得不到钱粮,一定会搜刮百姓。不如将这些东西送给百姓,使他们免遭穷困之苦。于是下旨熄灭火把。\\ 好似世人渲染他的爱情多过于他的政绩。 不过缠绵悱恻的爱情故事,也确实动人。
		
		\subparagraph{伍子胥}春秋时期,据传伍子胥逃亡时得到一位少女的救助,伍子胥告诫少女,不要说曾与自己见面。少女为获信任,投江自尽。伍为此伤感不已, 在石上血书:“十年之后,千金报德!”(伍子胥还问人家嫁人没有,没嫁的话说以后会娶她。女子知道他的身份,感其赤诚,跃入河中而亡。)
		
		\subparagraph{包公的工资}包公清廉底气从何而来,包公年收入约为:铜钱二万零八百五十六贯、大米二千一百八十石、小麦一百八十石、绫十疋、绢三十四疋、罗两疋、绵一百両、木炭十五枰、柴禾二百四十捆、干草四百八十捆。包公的年薪折合成现在的美元约为三百八十七万美元。
		
		\subparagraph{林则徐-笑而不语}林则徐初到广州禁烟,西方多国领事特备西餐宴请林则徐。在吃冰激凌时,因为冒着气,林大人以为很烫便张嘴吹了吹才放进口中,遭西人耻笑。事后,林盛宴回请。端上来芋泥(闽菜一种甜食),芋泥不冒热气,乍看犹如凉菜,实则烫舌。果然众领事一见佳肴,纷纷舀起就吃,直烫得哇哇大叫。林在一旁微笑不语。
		
		\subparagraph{潘金莲}《水浒》里的潘金莲可谓是臭名昭著,人人得而诛之。而据史载:武植,山东清河县武家那村人,身高1.78米以上,虽出身贫寒,但聪颖过人,崇文尚武,中年即考中进士,出任山东阳谷县县令。而潘金莲乃知州家的千金,名门淑媛。武、潘二人和睦恩爱,育有四子。原本贤良的县令夫人却背负了千载恶名。
		
		\subparagraph{六不总理}他“不抽不喝不嫖不赌不贪不占”,人称“六不总理”;他曾逼迫清帝退位,抵制袁世凯称帝,讨伐张勋复辟,得“三造共和”美誉;他晚年拒绝日本人拉拢,誓不卖国;他资助吴清源等大批棋手;“三一八”惨案后,他对死者长跪不起并终生吃素;他一生无积蓄房产,他是段祺瑞。
	
		\subparagraph{北大由于三只兔子而成名}\textbf{蔡元培},北大校长,生肖属兔。\textbf{陈独秀},北大文学院院长兼《新青年》刊主编,比蔡元培小十二岁,也属兔。\textbf{胡适},北大哲学系教授,比陈独秀小十二岁,亦属兔。1917年,三位联手,打破旧传统,人称“改变中国文化的三只兔子”。胡适曾俏皮地说:“北大是由于三只兔子而成名的。”
		
		\subparagraph{赵匡怡}赵匡胤登基后,为防止议事时朝臣交头接耳,就下诏书改变乌纱帽的样式:在乌纱帽的两边各加一个翅,这样只要脑袋一动,软翅就忽悠忽 悠地颤动,皇上居高临下,看得清清楚楚;并在乌纱帽上装饰不同的花纹,以区别官位的高低。
		
		\subparagraph{中国史上第一个殉职记者:沈荩}1903年,天津《新闻西报》披露沈荩搜集到《中俄密约》内容。国人无不愤怒指斥清廷卖国。清政府恼羞成怒,慈禧下旨将沈荩杖毙。刑部用竹鞭捶击沈荩,打得沈荩血肉横飞,沈荩骂声不绝,尚未气尽,后被绳索勒死,时年31岁。这是用生命说真话的\textbf{湖南}汉子!
		
		\subparagraph{外国史上十大单身名人}最有思想的光棍:\textbf{柏拉图};最可理解的光棍:\textbf{达•芬奇};女人中最成功的光棍:\textbf{伊丽莎白一世};最有理想的光棍:\textbf{牛顿};最不孤单的光棍:\textbf{伏尔泰};最身残志坚的光棍:\textbf{贝多芬};最神秘的光棍:\textbf{简•奥斯汀};最记仇的光棍:\textbf{诺贝尔};最痴情的光棍:\textbf{安徒生};最文艺的光棍:\textbf{文森特•凡高}!
		
		\subparagraph{万贞儿}明朝明宪宗与一个叫万贞儿的姑娘有一段年龄差距在17岁的超级姐弟恋,该姑娘手段了得,从宪宗10岁还是太子时就跟其有暧昧,到最后自己58岁背过气去足足摆布了宪宗二十几年,而且再其去世后几个月,宪宗身染重病也共赴黄泉了。他与万贞儿这对“老妻少夫“封建时代美女如云的宫廷中难得一见的生死恋.
		
		\subparagraph{饥者口中尽佳肴}慈禧终生都穿旗袍。惟一一次改穿平民化的汉族服装,是八国联军兵临城下,十万火急,慈禧只好化装成农妇(怕被追兵发现),去西安逃荒要饭去了。那一路上凄风苦雨,慈禧不仅披着老棉袄,而且吃了窝窝头。吃腻了山珍海味的"老佛爷",居然还觉得窝窝头是天下顶好吃的东西;事后还宫时曾令御膳房仿制。
		
		\subparagraph{董卓}东汉末年有一少年侠客,仗剑独行,游走边荒.许多人仰慕他,前来投奔,可是少年家徒四壁,就杀了家里唯一的耕牛,招待客人。追随者很感动,送了千匹牲畜给他。后来少年从军,立下战功,朝廷赏赐九千匹绢,他全分给了部下……这样的人,你愿不愿意追随?别急着回答,这少年就是祸乱天下的国贼;董卓。
		
		\subparagraph{“红颜祸水”到底是指史上哪位绝色美人}一位绝色美女使三个国家兵祸相接,其中两个国家分崩离析。尽管有人称她为“祸水”,后世却始终把她当作主宰桃花的神仙祭拜。这个女人是春秋时期著名的息夫人,又称“桃花夫人”。
		
		\subparagraph{刘病已}他是皇帝却坐过牢,为了不让百姓避讳而自己改名。他是汉武帝曾孙,戾太子之孙,出生不久即因\textbf{巫蛊事件}牵连被投入狱中,差点被处死。后流落民间,亲历民间疾苦。昭帝崩,昌邑王被立皇帝一月被废,\textbf{霍光}将他迎入宫立为皇帝。他在位期间政治清明社会和谐,史称宣帝中兴。他是汉宣帝刘询,原名刘病已。
		
		\subparagraph{洪秀全}洪秀全80多个妃子,\textbf{都认不全,只能编号},\textit{完全实现数字化管理},清朝嫔妃最多的乾隆爷也只有50多个。“幼天王”洪天贵福被俘后称:我有88个母后,我是第二个赖氏所生。我9岁时就给我4个妻子。 洪秀全到底有多少个美女?有一本书叫做《江南春梦笔记》记载总计有\textbf{两千三百多名妇女}在天王府陪侍他一个人!
		
		\subparagraph{让你吃个够}北宋有个叫\textbf{张咏}的地方官,在四川任职时,有一次吃馄饨,可能是\textbf{头巾没戴好},上面的带子垂落到碗里,他用手往上拢了拢,带子又垂落下来。再系,再落。张咏气坏了,一把将头巾扯下,狠狠丢到馄饨碗里,一边大叫道:“你自己吃个够吧,老子不吃了!”
		
		\subparagraph{隋文帝}隋文帝不但\textbf{结束南北分裂}两百多年的局面,而且\textbf{成功将原北周制度移植到大江南北}。他是\textbf{最早实行政治民主}的帝王,是\textbf{人类历史上最轻徭薄赋的帝国},\textbf{废除酷刑是最仁慈的帝王},\textbf{是拿破仑最崇拜的偶像}。\textbf{开创的科举制、三省六部制等}都影响了中国政治千年。是西方人眼中,中国历史上最伟大的帝王。
		
		\subparagraph{江南四大才子}眼下所知明末四大才子讲的都是\textbf{唐伯虎}、\textbf{祝枝山}、\textbf{文徵明}、\textbf{周文宾}四位苏州文人,而历史上根本没有周文宾其人。真正的江南四大才子(也称吴门四才子)是指唐、祝、文、徐,徐即\textbf{徐祯卿},\textbf{其诗风格清朗但不通书画,性格与另外三位不同}。大概这个缘故,\textbf{后人杜撰了相貌秀美的周文宾来凑数}。
		
		\subparagraph{“过五关斩六将”是哪五关?哪六将?}东汉末年,刘备、关羽、张飞在徐州失散后,关羽留在曹营。得知刘备在河北袁绍处,就带领二位皇嫂去投奔,曹操不许。关羽凭自己的勇猛,连续过东岭、洛阳、沂水、荥阳和黄河五关,斩杀孙秀、孟坦、韩福、下喜、王植和秦琪六位战将。

		\subparagraph{朱元璋}1398年的今天,朱元璋逝世,他出身贫苦,从小饱受元朝贪官污吏敲诈勒索,父母及长兄都死于剥削和瘟疫,自己被迫从小出家。所以参加起义队伍后发誓:一旦称帝,先杀尽天下贪官。从登基到驾崩,朱元璋“杀尽贪官”运动贯穿始终,但贪官现象从未根除,晚年哀叹:“如何贪官此锁,不足以为杀,早杀晚生”	
		
		\subparagraph{朱元璋}传说,朱元璋下葬时,也搞了一个“迷魂阵”,当天,十三城门同时出棺。这个说法已传述了六百年,是南京民间最经典的段子之一。南京过去有个民谣:“南京有三怪,龙潭的姑娘像老太,萝卜当作小菜卖,十三个城门抬棺材。”竟然把朱元葬当年下葬的传说,当成了南京代表性的一件事。
		
		\subparagraph{杀神}在两千多年前的中国曾出现过一位至今令后人无法超越的名将。\textbf{他因为以惊人的速度杀死了数量惊人的敌国士兵而名噪一时},据《史记》统计,他在其职业军事生涯中有据可查的杀人记录就多达九十万,他被赋予了“人屠”的称号,死后谤满天下。他就是中国历史上最高效的战争机器——\textbf{白起}。
		
		\subparagraph{从不对家人生气的宰相}北宋宰相王旦自幼性格沉稳,不大生气。他父亲从小就认为儿子是宰相材料。王旦从知县到宰相,20几年却脾气却一点没涨。他家人打赌看他会不会生气,就故意在饭菜的肉上撕上灰尘,他只吃饭不吃肉。第二天,家人又在饭菜上全撒上灰尘,他说:今天不想吃饭,随便喝点粥吧。
		
		\subparagraph{八大奇女子}
			\begin{itemize}
				\item 班昭:博学高才,女子修史为奇
				\item 卓文君:凤求凰兮,破礼私奔为奇
				\item 王昭君:绝代佳人,出塞匈奴为奇
				\item 谢道韫:咏絮才女,孤身杀贼为奇
				\item 李清照:才华卓绝,词坛称皇为奇
				\item 黄道婆:人间织女,授纺传世为奇
				\item 妇好:横戈跃马,开疆辟土为奇
				\item 武则天:善谋能断,女子称帝为奇
			\end{itemize}
		\subparagraph{中国古代最具魅惑力十大美女}被妖魔化的\textbf{妲己} ,“美人计”第一主角\textbf{西施}, 命犯桃花的\textbf{赵飞燕} ,从妓女到太皇后的\textbf{赵姬}, 柔骨侠肠\textbf{虞美人}, 征服男人的高手\textbf{冯小怜} ,历经五主而不衰的\textbf{萧皇后} ,肥美人\textbf{杨玉环} ,勇闯大漠的美人王\textbf{昭君}, 第一女间谍\textbf{貂蝉}, 天下第一“二奶”\textbf{李师师}。	
		
		\subparagraph{晋武帝司马炎羊车临幸嫔妃法}坐在\textbf{羊拉的御车上},游历宫苑,羊车在那个嫔妃宫前停住,就召幸那个嫔妃。有人想出一个办法,将竹叶子插在门口,地上洒了许多盐汁。羊生性喜欢吃竹叶,又喜欢吃咸的东西,因此就停下来不走。这些美女就得到了晋武帝惠顾临幸的机会。后众人纷纷效仿,这招就不好使了
		
		\subparagraph{萧衍}有一位皇帝以长寿和不近女色闻名,他便是\textbf{梁武帝萧衍}。《梁史》记载:萧衍“\textbf{五十外便断房室}”。天监十二年,萧衍始“不与女人同屋”。梁武帝去世时是86岁,如此算来,他竟有近40年没有碰过女人。
		
		\subparagraph{雍正很勤政}雍正是清朝入关后第三位皇帝,他于1722年继承皇位,到1735年去世,在位仅12年8个月,但他所做出的改革,\textbf{比他父亲康熙担任61年皇帝所做出的改革还要多} 。他勤于政事,每天都工作到深夜,睡眠时间不足4个小时。一年之中只有生日那天他才会休息 ​​
		​​
		\subparagraph{史上最忠贞的爱情}南北朝时的\textbf{西魏废帝元钦},一生钟情\textbf{宇文云英}。宇文云英品行端淑,深受元钦爱重,两人相爱甚欢。元钦当太子时,她是太子妃;元钦当皇帝后,晋为皇后,偌大后宫只此一人。元钦被宇文觉废掉后又被鸩害,宇文云英亦自杀殉情。皇宫里如此爱情真是难能可贵,可惜情深缘浅,徒叹奈何!
		
		\subparagraph{孔融}孔融让梨的故事流传至今,\textbf{但他43岁时为保命丢妻弃儿的行为却鲜有人知}。孔融因忠于汉室而得罪袁绍,守地被袁军攻打。虚妄狂放的他生怕有损其处变不惊的名士形象,仍故作镇定,饮酒吟诗,从不督战。直到敌军破城,,他在亲兵掩护下出逃,被丢下的妻子和两个儿子为其殉难
		
		\subparagraph{巨蟒转世}曾国藩出生时,祖父曾经梦到有一只巨蟒缠在他家的柱子上,所以认为曾国藩是巨蟒转世,曾国藩出生后家中的一棵死梧桐树竟然重新焕发出了生命,让其祖父更加相信巨蟒转世这一梦语。而凑巧的是曾国藩患有类似“牛皮癣”一类的皮肤病,(有一说“火蟒藓”),浑身上下都是像蛇的鳞片一样的癣,所以曾国藩也相信了巨蟒转世这一梦语。曾国藩还有一个奇怪的爱好——爱吃鸡,却又最怕鸡毛。当时紧急公文,在信封口处往往要粘上鸡毛,俗称鸡毛信、鸡毛令箭。每当曾国藩看到这种信,总是毛骨悚然,如见蛇蝎,必须要别人帮他取掉鸡毛,他才敢拆读。古时候曾有这样的说法:“焚烧鸡毛,毒蛇闻气就死了,龙蛇之类,也畏惧这种气味。”曾国藩对鸡毛害怕到这种程度,难免也被人理解为蟒蛇转世。在岳麓书院学习时因为怕别人看到身上的鳞片,所以夏天燥热时还穿戴整齐地读书,让先生大加赞赏

		\subparagraph{秦淮八艳之一柳如是}柳如是与他约好一起殉国,他却剃发降清,他一生很复杂,如果脱离那个时代,他绝对的宰辅之才,无奈生在乱世,又只愿保命。他一生最得意的是自己娶了柳如是,最失意的莫过于剃发降清的颜面尽失,但是他抗清之心昭昭,也是世人公认的。 ​他是钱谦益。柳如是,本名杨爱,字如是,又称河东君,因喜欢宋朝辛弃疾《贺新郎》中:\textbf{“我见青山多妩媚,料青山见我应如是”},故自号如是。柳如是在嫁给钱谦益之前,曾与宋征舆、陈子龙等文士均有过一段恋情,可惜迫于封建礼教无疾而终。崇祯十四年(1641年),文坛领袖钱谦益迎娶了柳如是。那一年,钱谦益59岁,柳如是23岁。可惜,君生我未生我生君已老,钱谦益去世时,柳如是还不到五十岁。在爱人走后,柳如是悬梁自尽。
		
		\subparagraph{北宋词人张先}\verb|80岁时娶了18岁的小妾|。苏轼和众朋友去拜访,问老先生得此美眷有何感想,张先随口念道:“我年八十卿十八,卿是红颜我白发。与卿颠倒本同庚,只隔中间一花甲。”风趣幽默的苏东坡则当即和一首:“十八新娘八十郎,苍苍白发对红妆。鸳鸯被里成双夜,一树梨花压海棠。” 
		
		\subparagraph{济公}济公原名\textbf{李修缘},南宋高僧,李家世代信佛,所以俗名李修缘。他破帽破扇破鞋垢衲衣,貌似疯颠,初在杭州灵隐寺出家,后住净慈寺,不受戒律拘束,嗜好酒肉,举止似痴若狂,\textbf{精通医术,好打不平,救人性命}。也是一位学问渊博的得道高僧,被列为禅宗第\textbf{五十}祖,杨岐派第六祖。
	\section{神话}
		\subparagraph{“黑白无常”到底姓甚名谁}传说中白无常名叫谢必安,黑无常名叫范无救,也称「七爷」、「八爷」。据说,因谢范二人自幼结义,情同手足,阎王爷嘉勋其信义深重,命他们在城隍爷前捉拿不法之徒。有人说,谢必安,就是酬谢神明则必安;范无救,就是犯法的人无救,当然这都是传说。
	
	
	\section{人文}
		\subparagraph{一别两宽,各生欢喜}唐朝的离婚协议书:凡为夫妇之因,前世三生结缘,始配今生之夫妇。若结缘不合,比是冤家,故来相对……既以二心不同,难归一意,快会及诸亲,各还本道。愿娘子相离之后,重梳婵鬓,美扫蛾眉,巧呈窈窕之姿,选聘高官之主。解怨释结,更莫相憎。一别两宽,各生欢喜。	
		
		\subparagraph{古代限房令}元世祖\textbf{忽必烈}曾禁止\textbf{蒙古官员}在原南宋统治区买房。原因:一是从前朝继承了许多房产,可供分给。二是一些蒙古官员强买强卖,民怨四起。到明清此令更甚。不光禁止官员在工作地置业,还限制旗人。违者重打50板,房产没收拍卖,公职开除
		
		\subparagraph{男女搭配年龄有说法}女小五,人楚楚; 女小四,好脾气; 女小三,男当官; 女小二,生宝儿; 女小一,住京师; 若同岁,常富贵。 女大一,穿锦衣; 女大二,生进儿; 女大三,抱金砖; 女大四,有喜事; 女大五,快致富; 女大六,总吃肉; 女大七,是闲妻; 女大八,事事发; 女大九,人长久。
		
		\subparagraph{中国四合院}所谓四合,"四"指东、西、南、北四面,"合"即四面房屋围在一起,形成一个"口"字形。北京正规四合院一般依东西向的胡同而坐北朝南,大门辟于宅院东南角"巽"位。四合院中间是庭院,是四合院布局的中心
		
		\subparagraph{顶戴花翎的顶珠}一品为\textbf{红宝石},二品为\textbf{珊瑚},三品为\textbf{蓝宝石},四品用\textbf{青金石},五品用\textbf{水晶},六品用\textbf{砗磲},七品为\textbf{素金},八品用阴纹缕花金,九品为阳纹镂花金。无顶珠者无官品。清朝爵位中最为显贵的亲王、郡王、贝勒,按清初的规定是不戴花翎的;乾隆年后部分亲王、郡王、贝勒开始佩戴三眼花翎
		
		\subparagraph{秋后问斩}汉、唐、宋、清\textbf{法津规定,刑杀要在秋冬进行}。来源‘天人感应’学说,天有四时,王有四政,庆赏刑罚与春夏秋冬以类相应。\textbf{再有秋属金主萧杀},也是一说。因执行期长,多有绝处逢生,幸逢大赦戏剧情节。\textbf{而午时三刻,这个时间阳气最盛,人的影子最短。此时可以用旺盛的阳气来冲淡杀人的阴气}。
		
		\subparagraph{清8旗的由来}清兵入关以前,努尔哈赤把满洲军队分成四旗,每一旗起初是7500人。后来因为人数增加(满人为主,包括少量蒙、汉、朝鲜、俄罗斯族人),由四旗扩充为八旗。八旗旗色分别为:正黄、正红、正白、正蓝、镶黄、镶红、镶白、镶蓝。旗的编制,合军政、民政于一体,极具战斗力。
		
		\subparagraph{清明节荡秋千}荡秋千是古代清明节的传统习俗,所以清明节也称“\textbf{秋千节}”。荡秋千是从南北朝时开始流行的,古时的秋千多用树枝为架,再栓上彩带制成,后来逐步发展出用两根绳索加上踏板的秋千。传说荡秋千可以驱除百病,荡得越高,象征生活越美好,这天男男女女结伴出游踏青,就好比现在的情人节
		
		\subparagraph{孑然一身}jié  :我现在离乡背井,孑然一身。谓孤零零一个人.:孑然一身有两层意思,但都是指孤独一人。 \textbf{其一}:不想找另一半,心如止水。\textbf{ 其二}:想找但是留不住他/她,最后还是一人到终老。
		
		\subparagraph{吹牛}源于屠夫。\textbf{从前宰羊时放完血,屠夫会在羊的腿上割开一个小口,把嘴凑上去使劲往里吹气,直到羊全身都膨胀起来,用刀轻轻一拉,皮就会自己裂开。这叫吹羊。}如果谁要说可以把牛皮吹起来,那就是说大话了,因为牛皮很大,而且非常坚韧,根本吹不起来。所以吹牛就是说大话的代名词
		
		\subparagraph{衣冠禽兽}在明朝的时候,官员的级别是靠朝服上的纹绣来区分的,\textbf{文官绣禽},包括仙鹤、锦鸡、孔雀、云雁等,\textbf{武官绣兽},包括狮、虎、豹、熊等。所以“衣冠禽兽”本来是身份地位的标识,表示你是穿官服的人,高人一等,但后来含义被扭曲了,成了品行低劣的代名词。
		
		\subparagraph{古代剩女要罚款}宋仁宗时期让男子十五岁娶,女子十三岁嫁;明太祖规定男子十六岁而娶,女子十四岁而嫁到了法定年龄不嫁人的女子,那是要罚款的,譬如,汉朝孝惠皇帝时,谁家要有女儿十五岁以上至三十岁还没有嫁人,罚款600钱;唐代对于男子二十岁以上,女孩十五岁以上还没有配对结婚的也要处罚。
		
		\subparagraph{”丈夫”一词来源}我国有些部落,有抢婚的习俗。\textbf{女子选择夫婿,主要看这个男子是否够高度,一般以身高一丈为标准。当时的一丈约等于10尺,商代以前一尺为16.95cm,一丈基本相当于现在的一米七。有了这个身高一丈的夫婿,才可以抵御强人的抢婚。根据这种情况,女子都称她所嫁的男人为“丈夫”。}
		
		\subparagraph{古代的厕纸}元朝皇帝用的厕纸我以为该是“肤卵如膜,坚洁如玉,细薄光润”的澄心堂宣纸。但史料记载了这么个事:裕圣皇后当太子妃的时候对婆婆非常孝顺,学李后主拿脸试手纸柔软度,顺便磨脸皮去角质。“至溷厕所用纸,亦以面擦,令柔软以进”,这纸应该还很粗糙
		
		\subparagraph{凤冠霞帔}宋代,霞帔正式作为贵族妇女的服饰,并随其丈夫或儿子品级的高低,式样各不相同,而且还有个硬性规定:非恩赐不得服,不是皇帝恩赐的人,不能穿着霞帔。而是在国家大典之外的各种礼仪场合所应着的正式礼服。明代也沿袭了这一制度,霞帔被用作后妃、命妇们的服饰
		
		\subparagraph{“跳槽”的来历}\textbf{"跳槽"原是青楼语},在明清时代,\textbf{最早这个词是说妓女的}。一个妓女和一个嫖客缠绵了一段之后,又发现了更有钱的主,于是丢弃旧爱,另就新欢,\textbf{如同马从一个槽换到了另外一个槽吃草},这种另攀高枝的做法被形象地称为“跳槽”。如今“跳槽”被当成变换工作的代语,也是为了钱。	
			
		\subparagraph{小蛮腰到底是谁的腰}小蛮腰典出有二:一说是唐代\textbf{诗人白居易的家姬小蛮}的腰像杨柳。二说是\textbf{楚王好细腰}的故事,因当时楚地被视为\textbf{荆蛮之地},故细腰别称蛮腰。第二说似乎更可信。《战国策》载:楚灵王喜欢臣子有细腰,所以大臣都每天只吃一顿饭,上朝前屏住呼吸把腰带束紧,弄得扶墙才能站起来。
\chapter{政治}


\chapter{经济}


\chapter{人文}
	\section{名词解释}
		\subparagraph{戒指的传闻}戒指 ,现已经成为爱情和信任的象征,但在\textbf{古罗马},戒指作为印章,是权利的象征。而我国在距今4千多年前就已有人佩戴戒指。到\textbf{秦汉}时期,妇女佩戴戒指已很普遍。\textbf{东汉}时,戒指已作定情之物,到了\textbf{唐代},戒指作为定情信物已极盛行
		
		\subparagraph{青楼}“青楼”\textbf{最初}指华美居所是帝王居所:清代著名诗人袁枚的《随园诗话》载,《南齐书·东昏侯纪》:“齐武王于兴光楼上施青漆,世谓之青楼。”可见,\textbf{青楼原先乃是帝王之居}。为此,袁枚指出:“今以妓院为青楼,实是误矣。”“青楼”就因袭前人之误传,成为妓院的别称。
		
		\subparagraph{对食-女同性恋}宫女们长时间的压抑,很容易在互相扶持的过程中产生强烈的感情,从而向两性一样,结成稳定的关系,称为“磨镜”,意思是指女同性恋双方相互以厮磨或抚摩对方身体得到一定的满足,由于双方有同样的身体结构,似乎在中间放置了一面镜子而在厮磨,故称“磨镜”,也称为“对食”
	
		\subparagraph{鸿雁传书}故事源于苏武牧羊。汉武帝时期,苏武奉命出使匈奴,却被囚禁于北海。后来汉匈和好要求释放苏武,匈奴却诈称苏武已死,而后汉朝得到一只从北飞来的鸿雁,鸿雁脚上系有帛书称苏武还在北海,匈奴不得不放回了苏武。从此,\textbf{鸿雁便成为了信使的象征}。
		
		\subparagraph{京剧}分为生、旦、净、丑四大类型,强调以 西皮、二黄为主,用胡琴和锣鼓等伴奏。他的表演艺术趋于虚实结合的表现手法,最大限度的超脱了舞台空间和时间的限制,以达到“以形传神,形神兼备”的艺术境界。
		
		\subparagraph{古琴}琴是中国古代文化地位最崇高的乐器,亦称古琴、玉琴、瑶琴、七弦琴等,是中国最为古老的弹拨乐器之一。古琴是中国“琴棋书画”四艺之首,是中国传统文化的重要组成部分,其文化渊源贯穿“道儒释”,被誉为哲学性的艺术。
		
		\subparagraph{倒楣}一词源于明朝末年。那时中国“八股取士”的科举制度严重限制了广大知识分子的才智发挥,加之考场舞弊之风甚盛,所以要想中举是极不容易的。为求个吉利,举子们在临考之前一般都要在自家门前竖起一根旗杆,当地人称之为“楣”。考中了,旗杆照竖不误,考不中就把旗杆撤去,叫做“倒楣”。
		
		\subparagraph{古代穿裙子的讲究}清朝时,红色的裙子不是每个女人都能穿的。“红裙子要夫妇双全才可以穿。若是一个孀妇,不许穿红裙,而且永远不许穿红裙”。而具体到一家之内,“惟正室可以穿红裙,姨太太不许穿红裙,即使是她的儿子已是科甲发达做了大官”。穿裙子也是门学问啊。
		
		\subparagraph{三餐制度的由来}在宋朝之前,老百姓一天只吃两顿,\textbf{只有皇室四餐},诸侯三餐,西汉时,给叛变被流放的淮南王的圣旨上,就专门点出,“减一日三餐为两餐”。 这归功于宋代经济的繁荣,除了酒肆之间不再如唐代被约束在市坊中间,夜市晚上不实行宵禁,晚上夜生活丰富,所以才加一顿晚饭。
		
		\subparagraph{两厢厮守}江南大户人家,\textbf{若生女婴,便在家中庭院栽香樟树一棵},女儿到待嫁年龄时,香樟树也长成。媒婆在院外只要看到此树,便知该家有待嫁姑娘,便可来提亲。\textbf{女儿出嫁时,家人要将树砍掉,做成两个大箱子,并放入丝绸},作为嫁妆,取“两厢厮守(两箱丝绸)”之意 ​​​​
		
		\subparagraph{时间概念}刹那、弹指或瞬间到底是多长时间。《僧祇律》记载:1剎那者为1念,20念为1瞬,20瞬为1弹指,20弹指为1罗预,20罗预为1须臾,1日1夜有30须臾。换算结果:须臾=48分钟,弹指=7.2秒,瞬间=0.36秒,剎那=1念=0.018秒。须臾>弹指>瞬间>刹那=1念
		
		\subparagraph{清朝皇帝龙袍值多少钱}织造一件鹅黄缎细绣五彩云水全洋金龙袍,需用绣匠608工,绣洋金工285工,画匠26工,每件工料银合计为392两2钱1分9厘。折合现在的价钱,\textbf{要十几万元}。一件朝袍耗资十几万 皇帝只穿一次,一件就需耗时两年
		
		\subparagraph{古时父母对孩子有“七不责”}1.对众不责,要有尊严;2.愧悔不责,因其自省;3.暮夜不责,不利入眠;4.饮食不责,易致脾虚;5.欢庆不责,经脉受损;6.悲忧不责,恐伤倍至;7.疾病不责,爱如良药 ​
		
		\subparagraph{古代对教师的称呼}​\textbf{夫子}:原为对孔子的尊称。\textbf{山长}:对山中书院主讲教师的称谓。\textbf{宗师}:原为掌管宗室子弟训导的官员。\textbf{教授}:讲学的博士。\textbf{助教}:在国子监任教的教师。\textbf{学博}:原为唐代府郡学官。\textbf{教谕}:原为宋代京师小学和武学中的学官名。此外还有“先生、师长、师傅、师父、师保、讲郎、老师”等称呼。​​
	\section{交际}
		\subparagraph{古代人口中的网络语言}城门失火殃及池鱼——躺着也中枪;呜呼!——我靠;是可忍孰不可忍!——草泥马;黄袍加身——屌丝的逆袭;桃花潭水深千尺,不及汪伦送我情——好基友,一辈子;暗风吹雨入寒窗——寂寞空虚冷;何以竟至于此——肿么酱紫
		
\chapter{娱乐}
	\section{故事}
		\subparagraph{史上著名八大饭局}	
			\begin{itemize}
				\item 杀机四伏的饭局——鸿门宴
				\item 最霸气的饭局——煮酒论英雄
				\item 最坑人的饭局——群英会
				\item 最香艳的饭局——贵妃醉酒
				\item 四两拨千斤的饭局——杯酒释兵权
				\item 最豪华的饭局——乾隆千叟宴
				\item 最鼓舞人心的饭局——东晋新亭会
				\item 最不辱使命的饭局——渑池之会 ​​​​
			\end{itemize}	    
\end{document} 
 		    