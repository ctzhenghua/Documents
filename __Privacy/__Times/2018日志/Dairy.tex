\documentclass[UTF8,a4paper,8pt]{ctexart} 

\usepackage{graphicx}%学习插入图
\usepackage{verbatim}%学习注释多行
\usepackage{booktabs}%表格
\usepackage{geometry}%图片
\usepackage{amsmath} 
\usepackage{amssymb}
\usepackage{listings}%代码
\usepackage{xcolor}%颜色
\usepackage{enumitem}%列表格式
\CTEXsetup[format+={\flushleft}]{section}

%设置文章宽度
\geometry{textwidth=18cm}
%设置页面布局
\pagestyle{plain}
\author{郑华}
\title{2018 Diary}

%正文排版开始
\begin{document} 
	\maketitle
	
\newpage
\section{January}
	\paragraph{Day 1   北京行总结    \quad     }
	\paragraph{Day 2       \quad     }
	\paragraph{Day 3       \quad     }
	\paragraph{Day 4   西安行总结    \quad     }
	\paragraph{Day 5   又有任务了    \quad     }
		写专利
	\paragraph{Day 6       \quad     }
	\paragraph{Day 7       \quad     }
	\paragraph{Day 8       \quad     }
	\paragraph{Day 9       \quad     }
	\paragraph{Day 10      \quad     }
	\paragraph{Day 11      \quad     }
	\paragraph{Day 12      \quad     }
		 \begin{itemize}[itemindent = 1em]
		 	% 重要
		 	\renewcommand\labelitemi{\makebox[0pt][l]{$\square$}\raisebox{.15ex}{\hspace{0.1em}$\checkmark$}}		
		 	\item   linux -so 动态库原理
		 	\item   1集  linux  开发
		 	
		 	% 较重要
		 	\renewcommand\labelitemi{\makebox[0pt][l]{$\square$}\hspace{1em}}
		 	\item   读书  30 Mins	:\verb|百年孤独|
		 	\item   leetCode 动态规划 一题
		 	%\item   Net	
		 	
		 	
		 	%\item   设计模式 :\verb|复习| 
		 	%\item   数据结构 :\verb|| 
		 	\item  曾国藩家书:劝学
		 \end{itemize}
	\paragraph{Day 13  25岁的自己,12点起床    \quad     }
		Hello:
		
		最近也不知怎么,一股暗黑势力从 心头浮生起来。浮躁、追求安逸。
		\begin{itemize}[itemindent = 1em]
			% 重要
			\renewcommand\labelitemi{\makebox[0pt][l]{$\square$}\raisebox{.15ex}{\hspace{0.1em}$\checkmark$}}		
			\item   linux -muduo 安装
			\item   1集  linux  开发
			
			% 较重要
			\renewcommand\labelitemi{\makebox[0pt][l]{$\square$}\hspace{1em}}
			\item   读书  30 Mins	:\verb|百年孤独|
			\item   leetCode 动态规划 一题
			%\item   Net	
			
			
			%\item   设计模式 :\verb|复习| 
			%\item   数据结构 :\verb|| 
			\item  曾国藩家书:劝学
		\end{itemize}
	\paragraph{Day 14   CMake 大概   \quad     }
		因为muduo安装 需要 学习 CMake.
		
	\paragraph{Day 15   动态库与静态库   \quad     }
		学习了动态库与静态库原理。
		
		
	\paragraph{Day 16   面向对象    \quad     }
	\paragraph{Day 17   基于对象  \quad     }
	\paragraph{Day 18   岛上书店   \quad     }
		爱情的相遇、相识、当然还有回忆。
		
	\paragraph{Day 19      \quad     }
	\paragraph{Day 20  整理了房间,帮一个博士写代码    \quad     }
		项目挺大的,全是代码,幸好知道怎么找主干逻辑。
		
		学习的东西总能在一天用上。--有道理。
	\paragraph{Day 21      \quad     }
	\paragraph{Day 22  美丽姐终于回消息了    \quad     }
		hello:
		
		提醒可以开始写大论文l.
	\paragraph{Day 23  总结下    \quad     }
		总体说,一天一节课程确实速度太慢,但是又静不下来该怎么半呢。
		
		听歌就停不下来。
		
		此时(3:09),房外是宁静无声的夜,房内是舍友在熟睡中伴随的打鼾声。
		独特的台灯下,
		一个人,
		带着耳机,听着歌,看着书评。
		
		这就是幸福。
		
		温婉的细涓长流
		惬意不过的生活方式。
		
		正是这种理想生活,简单而又难得。
		\begin{itemize}[itemindent = 1em]
			% 重要
			\renewcommand\labelitemi{\makebox[0pt][l]{$\square$}\raisebox{.15ex}{\hspace{0.1em}$\checkmark$}}		
			\item   linux -muduo 2小时

			% 较重要
			\renewcommand\labelitemi{\makebox[0pt][l]{$\square$}\hspace{1em}}
			\item   读书  30 Mins	:\verb|百年孤独|
			\item   leetCode 动态规划 一题
			%\item   Net	
			
			
			%\item   设计模式 :\verb|复习| 
			%\item   数据结构 :\verb|| 
			\item  曾国藩家书:劝学
		\end{itemize}
	\paragraph{Day 24      \quad     }
	\paragraph{Day 25      \quad     }
	\paragraph{Day 26      \quad     }
	\paragraph{Day 27      \quad     }
	\paragraph{Day 28      \quad     }
	\paragraph{Day 29      \quad     }   
			\begin{itemize}[itemindent = 1em]
				% 重要
				\renewcommand\labelitemi{\makebox[0pt][l]{$\square$}\raisebox{.15ex}{\hspace{0.1em}$\checkmark$}}		
				\item   linux -muduo 2小时
				\item   C++ - 线程 2小时
				
				
				% 较重要
				\renewcommand\labelitemi{\makebox[0pt][l]{$\square$}\hspace{1em}}
				\item   读书  30 Mins	:\verb|百年孤独|
				\item   编程  牛客,2小时
				%\item   Net	
				
				
				%\item   设计模式 :\verb|复习| 
				%\item   数据结构 :\verb|| 
				\item  曾国藩家书:劝学
			\end{itemize}
	\paragraph{Day 30  Fate/Zero  看完  \quad     }
		Hello,  感觉不能舒服的过节假日,这样的日子太过安逸了。
		
	\paragraph{Day 31     \quad     }
		\begin{itemize}[itemindent = 1em]
			% 重要
			\renewcommand\labelitemi{\makebox[0pt][l]{$\square$}\raisebox{.15ex}{\hspace{0.1em}$\checkmark$}}		
			\item   linux -muduo 1解
			\item   C++ - 线程 
			\item   C++ http项目半
						
			% 较重要
			\renewcommand\labelitemi{\makebox[0pt][l]{$\square$}\hspace{1em}}
			\item   读书  30 Mins	:\verb|百年孤独|
			%\item   编程  牛客,2小时
			%\item   Net	
			
			%\item   设计模式 :\verb|复习| 
			%\item   数据结构 :\verb|| 
			%\item  曾国藩家书:劝学
		\end{itemize}
\section{February}
	\paragraph{Day 1   离春招又进了一步   \quad     }
		今天还是比较满意的,从起床学习到天黑,中途跑毛次数偏少,主要精力集中在了线程学习上与服务器模型上。下午则了解了Memcache与Redis
		
		\begin{itemize}[itemindent = 1em]
			% 重要
			\renewcommand\labelitemi{\makebox[0pt][l]{$\square$}\raisebox{.15ex}{\hspace{0.1em}$\checkmark$}}		
			\item   linux-muduo 1解
			\item   C++ 线程 
			\item   C++ http项目半
			\item   Redis 数据库
			%\item   脚本语言 shell Python
			% 较重要
			\renewcommand\labelitemi{\makebox[0pt][l]{$\square$}\hspace{1em}}
			\item   读书  30 Mins	:\verb|百年孤独|
			%\item   编程  牛客,2小时
			%\item   Net	
			
			%\item   设计模式 :\verb|复习| 
			%\item   数据结构 :\verb|| 
			%\item   曾国藩家书:劝学
		\end{itemize}
		
		加油
	\paragraph{Day 2   坚持了下来   \quad     }
		\begin{itemize}[itemindent = 1em]
			% 重要
			\renewcommand\labelitemi{\makebox[0pt][l]{$\square$}\raisebox{.15ex}{\hspace{0.1em}$\checkmark$}}		
			\item   linux-muduo 2节
			\item   C++ 泛型编程
			\item   C++ http项目
			\item   Redis 数据库
			%\item   脚本语言 shell
					
			% 较重要
			\renewcommand\labelitemi{\makebox[0pt][l]{$\square$}\hspace{1em}}
			\item   读书  30 Mins	:\verb|百年孤独|
			%\item   编程  牛客,2小时
			%\item   Net	
					
			\item   设计模式 :\verb|Observer| 
			%\item   数据结构 :\verb|| 
			%\item   曾国藩家书:劝学
		\end{itemize}
	\paragraph{Day 3   独处    \quad     }
		又好好学习了一天,晚上打了2小时游戏。 对vim  有了新的认识
		\begin{itemize}[itemindent = 1em]
			% 重要
			\renewcommand\labelitemi{\makebox[0pt][l]{$\square$}\raisebox{.15ex}{\hspace{0.1em}$\checkmark$}}		
			%\item   linux-muduo 2节
			\item   Linux 系统编程基础
			%\item   C++ http项目
			\item   Redis 数据库
			%\item   MySql 数据库
			\item   VIM  常用命令
			%\item   脚本语言 shell
							
			% 较重要
			\renewcommand\labelitemi{\makebox[0pt][l]{$\square$}\hspace{1em}}
			\item   读书  30 Mins	:\verb|百年孤独|
			%\item   编程  牛客,2小时
			%\item   Net	
							
			\item   设计模式 :\verb|Observer| 
			%\item   数据结构 :\verb|| 
			%\item   曾国藩家书:劝学
		\end{itemize}
	\paragraph{Day 4       \quad     }
		
		休息,昨天晚上发现白起挺好玩,今天就陶醉在这,然后5点多才睡,白天没有效率,所以干什么都没了精神,不论学习还是打游戏
		
	\paragraph{Day 5       \quad     }	
		git  学习完了
		
		\begin{itemize}[itemindent = 1em]
			% 重要
			\renewcommand\labelitemi{\makebox[0pt][l]{$\square$}\raisebox{.15ex}{\hspace{0.1em}$\checkmark$}}		
			%\item   linux-muduo 2节
			%\item   Linux 系统编程2章
			%\item  C++ http项目
			%\item   Redis 数据库
			%\item   MySql 数据库
			%\item   脚本语言 shell
			\item   Git	
			%\item  Python
			%\item  Web 前端	
			% 较重要
			\renewcommand\labelitemi{\makebox[0pt][l]{$\square$}\hspace{1em}}
			\item   读书  30 Mins	:\verb|百年孤独|
			%\item   编程  牛客,2小时
			%\item   Net	
			
			%\item   设计模式 :\verb|Observer| 
			%\item   数据结构 :\verb|| 
			%\item   曾国藩家书:劝学
		\end{itemize}
	\paragraph{Day 6   收到论文返修通知    \quad     }
		中午与下午看了C++的两个大问题,一是cdecl 和stdcall  的区别,二是c++内存模型,三是static 的初始化问题,最后一个就是singleton 多线程问题。
		
		在晚上刚叫完外卖吃的时候,想着查个论文的结果,结果就回来了。
		
		开心,但是本来预定明天西安陪爸妈逛街的计划就推迟到了后天。
		
		希望再接受就是交版面费吧。
		\begin{itemize}[itemindent = 1em]
			% 重要
			\renewcommand\labelitemi{\makebox[0pt][l]{$\square$}\raisebox{.15ex}{\hspace{0.1em}$\checkmark$}}		
			%\item   linux-muduo 2节
			%\item   Linux 系统编程2章
			%\item  C++ http项目
			%\item   Redis 数据库
			%\item   MySql 数据库
			%\item   UML 
			%\item   脚本语言 shell
			%\item  Python
			%\item  Web 前端	
			
			% 较重要
			\renewcommand\labelitemi{\makebox[0pt][l]{$\square$}\hspace{1em}}
			\item   读书  30 Mins	:\verb|百年孤独|
			%\item   编程  牛客,2小时
			%\item   Net	

			%日常
			%\item   曾国藩家书:劝学
		\end{itemize}
		
	\paragraph{Day 7  美丽姐请吃饭     \quad     }
	\paragraph{Day 8  回家     \quad     }
	\paragraph{Day 9  键盘666-全新樱桃3494     \quad     }
		今天是陪阿爸去洛川贿赂厅长,希望可以把钱尽量收回来。逛了下街,发现这个城市的下游好像一直没有发展的样子,与父亲去了老面馆吃了干拌然后在车上等,我看着心仪的键盘。
		
		虽然外面的风很大,但车里却很暖和,不仅是温度上的,心里也是。虽然父亲还是一样不可预测的脾气。
		
	\paragraph{Day 10      \quad     }
		我一直忘记的是母亲为这个家做的所有事情,好像是理所当然,但是,不是的。 岁月已经留给她身上太多的痕迹,我不能这样再让她们生气了。
		
	\paragraph{Day 11   提交回修稿   \quad     }
	\paragraph{Day 12   学习   \quad     }
		hello:
		
		回家好几天了,也没有写日记,回家我也想脾气好好的,毕竟现在与父母在一起的日子已经开始做减法了,相处一天少一天。唉
		
		晚上一狠心买了有生一来最贵的键盘 \verb|realForce 87U| 原价1699元,我淘的1050. 毕竟是个经常敲字的,希望可以用的顺手。
		
		今天学习了MySQL 的查询语句们,把内连接和基本查询语句都学习了。比较充实
		
		下午,两个弟弟来了,初二了,突然发现与他们好像真的找不到话题了。是自己长大了,还是他们长大了。
		
		明天尽量勤快起来,至少在家里帮帮忙之类的,虽然要学习,但是与家人相处才是难能可贵的。
	\paragraph{Day 13      \quad     }
		\begin{itemize}[itemindent = 1em]
			% 重要
			\renewcommand\labelitemi{\makebox[0pt][l]{$\square$}\raisebox{.15ex}{\hspace{0.1em}$\checkmark$}}		
			\item   linux-muduo 2节
			%\item   Linux 系统编程2章
			%\item   C++ http项目
			%\item   Redis 数据库
			\item   MySql 数据库
			%\item   UML 
			%\item   脚本语言 shell
			%\item  Python
			%\item  Web 前端	
			
			% 较重要
			\renewcommand\labelitemi{\makebox[0pt][l]{$\square$}\hspace{1em}}
			\item   读书  30 Mins	:\verb|百年孤独|
			%\item   编程  牛客,2小时
			%\item   Net	
			
			%日常
			%\item   曾国藩家书:劝学
		\end{itemize}
	\paragraph{Day 14      \quad     }
	\paragraph{Day 15      \quad     }
	\paragraph{Day 16      \quad     }
	\paragraph{Day 17      \quad     }
	\paragraph{Day 18      \quad     }
	\paragraph{Day 19      \quad     }
	\paragraph{Day 20      \quad     }
	\paragraph{Day 21      \quad     }
	\paragraph{Day 22      \quad     }
	\paragraph{Day 23      \quad     }
	\paragraph{Day 24  计划安排西安旅行行程    \quad     }
	\paragraph{Day 25  10    \quad     }
		\begin{itemize}
			\item 同盛祥 羊肉泡 
			\item MUJI
			\item 开元商城 民生百货
			\item 回民街 羊肉串 贾三包子
		\end{itemize}
		
		胜似闲庭 歇脚
	\paragraph{Day 26  11    \quad     }
		\begin{itemize}
			\item 去医院
			\item 吃顿牛排(王者荣耀主题牛排)
			\item 看电影(红海行动)
			\item 大雁塔广场(7点前)
			\item 晚饭 重庆老火锅
		\end{itemize}
		
		如家歇脚
	\paragraph{Day 27  12    \quad     }
		\begin{itemize}
			\item 吃德克士快餐全家桶
			\item 五环买衣服(买鞋子)爸妈穿过 最贵的鞋子了算是659,679
			\item 大唐芙蓉园(4-9点)
			\item 赛格吃饭 (蜀香圆)
		\end{itemize}
		
		如家歇脚,晚上对今天各自的表现进行了抱怨,尤其是阿妈不让我给老爸买那件衣服。但其实想想也是,从来没有这么大手笔的花过钱的他们确实难以接受和转化,这是一个过程,需要慢慢转化。
		
	\paragraph{Day 28  13   \quad     }
		\begin{itemize}
			\item 五环买衣服外套
			\item 去看望四姑
			\item 回家
			\item 陪苗苗看电影
		\end{itemize}
		
		怡家歇脚

		晚上期间问了王猛稿费的具体流程,早上起床吃完饭后,做了第一件事情,将稿费汇了过去,接着就再剩两件事情了
		
			\begin{enumerate}
				\item 权利转移书签字盖章 (李书琴、美丽姐、以及学院综合办盖章)
				\item 修改论文
			\end{enumerate}
		
		人事上昨天才知道,李老师儿子结婚,而她门下的弟子都提前给了份子钱,这份红包只能去给了,不给的话又怕是要难为我了
		
		其次,在学业计划制定方面,需要李老师帮忙删除一门课程。 总体来说也就3件事情,一天一件尽量,但是周6又是李老师儿子的婚宴,不知道又要拖几日。入学何曾想过,挂个名也是如此的痛苦。要是给美丽老师,想必是心甘情愿的。
		
\section{March}
 	 \paragraph{Day 1   干第一件事情-汇稿费   周四 \quad     }
 	 
 	 \paragraph{Day 2   修改论文    周五 \quad     }
	 	 论文改的差不多了
	 	 
	 	 完成了修改的指标:
	 	 
	 	 修改了英文摘要中 的第一人称。
	 	 
	 	 修改了参考文献的格式,与引用格式
	 	 
	 	 修改了文章的格式,将算法使用原文格式进行替换表格,检查各表题图题、
	 	 
	 	 添加作者信息等基本内容。
	 	 
	 	 明天就要参见,还是玩跳楼机时候的总结,等待的过程总是可怕的,但是经历了后其实没啥。
 	 \paragraph{Day 3   参加婚宴    周六 \quad     }
		 参加了婚宴,城市的婚宴虽然场面好看点但是热闹程度远不及农村的有氛围。果真是传统的东西还得到农村去找。
		 	 
		 下午与文浩到邦哥那坐了会,聊了会儿,将去年没看成邦哥的愿望实现了。
		 
		 了解-“区块链”技术,去中心化,点对点模式,go 语言技术
		 
		 TiDB 分布式数据库技术
		 
 	 \paragraph{Day 4   修改摘要    签字盖章  周末 \quad     }
		 \begin{itemize}[itemindent = 1em]
		 	% 重要
		 	\renewcommand\labelitemi{\makebox[0pt][l]{$\square$}\raisebox{.15ex}{\hspace{0.1em}$\checkmark$}}		
		 	
		 	\item    重装系统
		 	%\item    linux-muduo 2节
		 %	\item    Linux 系统编程2章
		 	%\item    Redis 数据库
		 	%\item    毕设摘要 半页。
		 	
		 	
		 	%\item   C++ http项目
		 	%\item   go 语言学习
		 	%\item   MySql 数据库
		 	%\item   UML 
		 	\item    学习配置vim 插件。
		 	%\item   脚本语言 shell
		 	%\item   Python
		 	%\item   Web 前端	
		 	
		 	% 较重要
		 	\renewcommand\labelitemi{\makebox[0pt][l]{$\square$}\hspace{1em}}
		 	\item   读书  30 Mins	:\verb|三国演义|
		 	%\item   编程  牛客,2小时
		 	%\item   Net	
		 	
		 	%日常
		 	%\item   曾国藩家书:劝学
		 \end{itemize}
		 
		 今天键盘回来了。3494,感觉确实不一样,手感很好。
 	 \paragraph{Day 5   签字盖章    周一    \quad     }
	 	 帮常存宝改了论文,提了意见。
	 	 
	 	 学院改了章子,找签字没找到。	 	 
	 	 
	 	 买了第二个realforce  键盘,买键盘今年就花了3000了快。 不过想想以后又不用花了又反而省下来不少。
	 	 
	 	 但是最近真的要节省开支了,没钱了。
	 	 
	 	 晚上还吃了 码头故事的火锅,花了340多。
	 	 
 	 \paragraph{Day 6   签字	周二    \quad     }
	 	  \begin{itemize}[itemindent = 1em]
	 	  	% 重要
	 	  	\renewcommand\labelitemi{\makebox[0pt][l]{$\square$}\raisebox{.15ex}{\hspace{0.1em}$\checkmark$}}		
	 	  	
	 	  	\item    签字
	 	  	\item    写摘要。
	 	  	\item    linux-muduo 2节
	 	  	%\item    Linux 系统编程2章
	 	  	%\item    Redis 数据库
	 	  	
	 	  	%\item   C++ http项目
	 	  	%\item   go 语言学习
	 	  	%\item   MySql 数据库
	 	  	%\item   UML 
	 	  	%\item   脚本语言 shell
	 	  	%\item   Python
	 	  	
	 	  	%日常
	 	  	\renewcommand\labelitemi{\makebox[0pt][l]{$\square$}\hspace{1em}}
	 	  	\item   读书  30 Mins	:\verb|三国演义|
	 	  	\item   编程  牛客,1小时
	 	  	\item   曾国藩家书:劝学
	 	  \end{itemize}
 	 \paragraph{Day 7       \quad     }
 	 
 	 \paragraph{Day 8       \quad     }
	 	 提交修改了文章,将摘要的中英文重新书写。
 	 
 	 \paragraph{Day 9   西安吃火锅牛排    \quad     }
	 	 王者荣耀牛排
	 	 
	 	 小龙坎火锅
	 	 
	 	 在锦江之星落脚
	 	 
 	 \paragraph{Day 10  截至日期    		\quad     }
	 	 中午回来,下午拆开键盘。
	 	 
	 	 休整。
 	 \paragraph{Day 11      \quad     }
	 	 \begin{itemize}[itemindent = 1em]
	 	 	% 重要
	 	 	\renewcommand\labelitemi{\makebox[0pt][l]{$\square$}\raisebox{.15ex}{\hspace{0.1em}$\checkmark$}}		

	 	 	\item    写完第一章。
	 	 	\item    linux-muduo 2节
	 	 	%\item    Linux 系统编程2章
	 	 	%\item    Redis 数据库
	 	 	
	 	 	%\item   C++ http项目
	 	 	%\item   go 语言学习
	 	 	%\item   MySql 数据库
	 	 	%\item   UML 
	 	 	%\item   脚本语言 shell
	 	 	%\item   Python
	 	 	
	 	 	%日常
	 	 	\renewcommand\labelitemi{\makebox[0pt][l]{$\square$}\hspace{1em}}
	 	 	\item   读书  30 Mins	:\verb|三国演义|
	 	 	\item   编程  牛客,1小时
	 	 	\item   曾国藩家书:劝学
	 	 \end{itemize}
 	 \paragraph{Day 12      \quad     }
 	 \paragraph{Day 13      \quad     }
	 	 \begin{itemize}[itemindent = 1em]
	 	 	% 重要
	 	 	\renewcommand\labelitemi{\makebox[0pt][l]{$\square$}\raisebox{.15ex}{\hspace{0.1em}$\checkmark$}}		
	 	 	
	 	 	\item    写完第2章。
	 	 	%\item    linux-muduo 2节
	 	 	%\item    Linux 系统编程2章
	 	 	%\item    Redis 数据库
	 	 	
	 	 	%\item   C++ http项目
	 	 	%\item   go 语言学习
	 	 	%\item   MySql 数据库
	 	 	%\item   UML 
	 	 	%\item   脚本语言 shell
	 	 	%\item   Python
	 	 	
	 	 	%日常
	 	 	\renewcommand\labelitemi{\makebox[0pt][l]{$\square$}\hspace{1em}}
	 	 	\item   国学  一节
	 	 	\item   读书  30 Mins	:\verb|三国演义 2-3章|
	 	 	\item   曾国藩家书:劝学
	 	 \end{itemize}
	 	 
	 	 
 	 \paragraph{Day 14      \quad     }
	 	 \begin{itemize}[itemindent = 1em]
	 	 	% 重要
	 	 	\renewcommand\labelitemi{\makebox[0pt][l]{$\square$}\raisebox{.15ex}{\hspace{0.1em}$\checkmark$}}		
	 	 	
	 	 	\item    完成2章数值计算方法并写第3章35\%。
	 	 	%\item    linux-muduo 2节
	 	 	%\item    Linux 系统编程2章
	 	 	%\item    Redis 数据库
	 	 	
	 	 	%\item   C++ http项目
	 	 	%\item   go 语言学习
	 	 	%\item   MySql 数据库
	 	 	%\item   UML 
	 	 	%\item   脚本语言 shell
	 	 	%\item   Python
	 	 	
	 	 	%日常
	 	 	\renewcommand\labelitemi{\makebox[0pt][l]{$\square$}\hspace{1em}}
	 	 	\item   国学  一节
	 	 	\item   读书  30 Mins	:\verb|三国演义 4-5章|
	 	 	\item   曾国藩家书:劝学
	 	 \end{itemize}
 	 \paragraph{Day 15      \quad     }
		 \begin{itemize}[itemindent = 1em]
		 	% 重要
		 	\renewcommand\labelitemi{\makebox[0pt][l]{$\square$}\raisebox{.15ex}{\hspace{0.1em}$\checkmark$}}		
		 	
		 	\item    写完第3章70\%。
		 	%\item    linux-muduo 2节
		 	%\item    Linux 系统编程2章
		 	%\item    Redis 数据库
		 	
		 	%\item   C++ http项目
		 	%\item   go 语言学习
		 	%\item   MySql 数据库
		 	%\item   UML 
		 	%\item   脚本语言 shell
		 	%\item   Python
		 	
		 	%日常
		 	\renewcommand\labelitemi{\makebox[0pt][l]{$\square$}\hspace{1em}}
		 	\item   国学  一节
		 	\item   读书  30 Mins	:\verb|三国演义 6-7章|
		 	\item   曾国藩家书:劝学
		 \end{itemize}
 	 \paragraph{Day 16      \quad     }
	 	 \begin{itemize}[itemindent = 1em]
	 	 	% 重要
	 	 	\renewcommand\labelitemi{\makebox[0pt][l]{$\square$}\raisebox{.15ex}{\hspace{0.1em}$\checkmark$}}		
	 	 	
	 	 	\item    写完第3章。
	 	 	%\item    linux-muduo 2节
	 	 	%\item    Linux 系统编程2章
	 	 	%\item    Redis 数据库
	 	 	
	 	 	%\item   C++ http项目
	 	 	%\item   go 语言学习
	 	 	%\item   MySql 数据库
	 	 	%\item   UML 
	 	 	%\item   脚本语言 shell
	 	 	%\item   Python
	 	 	
	 	 	%日常
	 	 	\renewcommand\labelitemi{\makebox[0pt][l]{$\square$}\hspace{1em}}
	 	 	\item   国学  一节
	 	 	\item   读书  30 Mins	:\verb|三国演义 8-9章|
	 	 	\item   曾国藩家书:劝学
	 	 \end{itemize}
 	 \paragraph{Day 17      \quad     }
	 	 \begin{itemize}[itemindent = 1em]
	 	 	% 重要
	 	 	\renewcommand\labelitemi{\makebox[0pt][l]{$\square$}\raisebox{.15ex}{\hspace{0.1em}$\checkmark$}}		
	 	 	
	 	 	\item    写完第4章35\%。
	 	 	%\item    linux-muduo 2节
	 	 	%\item    Linux 系统编程2章
	 	 	%\item    Redis 数据库
	 	 	
	 	 	%\item   C++ http项目
	 	 	%\item   go 语言学习
	 	 	%\item   MySql 数据库
	 	 	%\item   UML 
	 	 	%\item   脚本语言 shell
	 	 	%\item   Python
	 	 	
	 	 	%日常
	 	 	\renewcommand\labelitemi{\makebox[0pt][l]{$\square$}\hspace{1em}}
	 	 	\item   国学  一节
	 	 	\item   读书  30 Mins	:\verb|三国演义 10=11章|
	 	 	\item   曾国藩家书:劝学
	 	 \end{itemize}
 	 \paragraph{Day 18      \quad     }
	 	 \begin{itemize}[itemindent = 1em]
	 	 	% 重要
	 	 	\renewcommand\labelitemi{\makebox[0pt][l]{$\square$}\raisebox{.15ex}{\hspace{0.1em}$\checkmark$}}		
	 	 	
	 	 	\item    写完第4章70\%。
	 	 	%\item    linux-muduo 2节
	 	 	%\item    Linux 系统编程2章
	 	 	%\item    Redis 数据库
	 	 	
	 	 	%\item   C++ http项目
	 	 	%\item   go 语言学习
	 	 	%\item   MySql 数据库
	 	 	%\item   UML 
	 	 	%\item   脚本语言 shell
	 	 	%\item   Python
	 	 	
	 	 	%日常
	 	 	\renewcommand\labelitemi{\makebox[0pt][l]{$\square$}\hspace{1em}}
	 	 	\item   国学  一节
	 	 	\item   读书  30 Mins	:\verb|三国演义 12-13章|
	 	 	\item   曾国藩家书:劝学
	 	 \end{itemize}
	 	 
	 	 第四章写到了传统层次化方法,预计明天就可以写的差不多了。可是最近春招他们的好消息还有海能达的坏消息还是传到了我的耳朵里,带来了很多烦心的忧虑,会不会影响以后发展的路子之类的。思虑又多了。
	 	 
	 	 是时候学习linux 分布式服务器编程之类的内容了,这样自己提升起来跳槽才存在可能性,而且海能达通信跳也只能往华为跳了。但是我想做的是服务器开发,或者游戏开发之类的。
	 	 
	 	 烦恼多了,所以干事就会带有烦虑,尽量记录下来,这样就会好很多。
	 	 
	 	 这两天还是没有给家里打电话,尽量抽时间给家里打个电话。
 	 \paragraph{Day 19      \quad     }
	 	 面对了,啥都不是
	 	 
	 	 真正的勇士,要直面惨淡的人生,我现在才懂是什么意思了
	 	 
	 	 原来还是我们太年轻,以为自己什么都懂
	 	 
	 	 \begin{itemize}[itemindent = 1em]
	 	 	% 重要
	 	 	\renewcommand\labelitemi{\makebox[0pt][l]{$\square$}\raisebox{.15ex}{\hspace{0.1em}$\checkmark$}}		
	 	 	
	 	 	\item    写完第4章100\%。
	 	 	%\item    linux-muduo 2节
	 	 	%\item    Linux 系统编程2章
	 	 	%\item    Redis 数据库
	 	 	
	 	 	%\item   C++ http项目
	 	 	%\item   go 语言学习
	 	 	%\item   MySql 数据库
	 	 	%\item   UML 
	 	 	%\item   脚本语言 shell
	 	 	%\item   Python
	 	 	
	 	 	%日常
	 	 	\renewcommand\labelitemi{\makebox[0pt][l]{$\square$}\hspace{1em}}
	 	 	\item   国学  一节
	 	 	\item   读书  30 Mins	:\verb|三国演义 14-15章|
	 	 	\item   曾国藩家书:劝学
	 	 \end{itemize}
 	 \paragraph{Day 20      \quad     }
	 	 \begin{itemize}[itemindent = 1em]
	 	 	% 重要
	 	 	\renewcommand\labelitemi{\makebox[0pt][l]{$\square$}\raisebox{.15ex}{\hspace{0.1em}$\checkmark$}}		
	 	 	
	 	 	\item    写完第2章-碰撞检测。
	 	 	%\item    linux-muduo 2节
	 	 	%\item    Linux 系统编程2章
	 	 	%\item    Redis 数据库
	 	 	
	 	 	%\item   C++ http项目
	 	 	%\item   go 语言学习
	 	 	%\item   MySql 数据库
	 	 	%\item   UML 
	 	 	%\item   脚本语言 shell
	 	 	%\item   Python
	 	 	
	 	 	%日常
	 	 	\renewcommand\labelitemi{\makebox[0pt][l]{$\square$}\hspace{1em}}
	 	 	\item   国学  一节
	 	 	\item   读书  30 Mins	:\verb|三国演义 16-17章|
	 	 	\item   曾国藩家书:劝学
	 	 \end{itemize}
 	 \paragraph{Day 21      \quad     }
	 	 \begin{itemize}[itemindent = 1em]
	 	 	% 重要
	 	 	\renewcommand\labelitemi{\makebox[0pt][l]{$\square$}\raisebox{.15ex}{\hspace{0.1em}$\checkmark$}}		
	 	 	
	 	 	\item    写完感谢。
	 	 	%\item   linux-muduo 2节
	 	 	%\item   Linux 系统编程2章
	 	 	%\item   Redis 数据库
	 	 	
	 	 	%\item   C++ http项目
	 	 	%\item   go 语言学习
	 	 	%\item   MySql 数据库
	 	 	%\item   UML 
	 	 	%\item   脚本语言 shell
	 	 	%\item   Python
	 	 	
	 	 	%日常
	 	 	\renewcommand\labelitemi{\makebox[0pt][l]{$\square$}\hspace{1em}}
	 	 	\item   国学  一节
	 	 	\item   读书  30 Mins	:\verb|三国演义 18-19章|
	 	 	\item   曾国藩家书:劝学
	 	 \end{itemize}
 	 \paragraph{Day 22      \quad     }
	 	 \begin{itemize}[itemindent = 1em]
	 	 	% 重要
	 	 	\renewcommand\labelitemi{\makebox[0pt][l]{$\square$}\raisebox{.15ex}{\hspace{0.1em}$\checkmark$}}		
	 	 	
	 	 	\item    检查第一章。
	 	 	%\item   linux-muduo 2节
	 	 	%\item   Linux 系统编程2章
	 	 	%\item   Redis 数据库
	 	 	%\item   C++ http项目
	 	 	%\item   go 语言学习
	 	 	%\item   MySql 数据库
	 	 	%\item   UML 
	 	 	%\item   脚本语言 shell
	 	 	%\item   Python
	 	 	
	 	 	%日常
	 	 	\renewcommand\labelitemi{\makebox[0pt][l]{$\square$}\hspace{1em}}
	 	 	\item   国学  一节
	 	 	\item   读书  30 Mins	:\verb|三国演义 20-21章|
	 	 	\item   曾国藩家书:劝学
	 	 \end{itemize}
 	 \paragraph{Day 23      \quad     }
	 	 \begin{itemize}[itemindent = 1em]
	 	 	% 重要
	 	 	\renewcommand\labelitemi{\makebox[0pt][l]{$\square$}\raisebox{.15ex}{\hspace{0.1em}$\checkmark$}}		
	 	 	
	 	 	\item    检查重读第2章。
	 	 	%\item   linux-muduo 2节
	 	 	%\item   Linux 系统编程2章
	 	 	%\item   Redis 数据库
	 	 	%\item   C++ http项目
	 	 	%\item   go 语言学习
	 	 	%\item   MySql 数据库
	 	 	%\item   UML 
	 	 	%\item   脚本语言 shell
	 	 	%\item   Python
	 	 	
	 	 	%日常
	 	 	\renewcommand\labelitemi{\makebox[0pt][l]{$\square$}\hspace{1em}}
	 	 	\item   国学  一节
	 	 	\item   读书  30 Mins	:\verb|三国演义 22-23章|
	 	 	\item   曾国藩家书:劝学
	 	 \end{itemize}
	 	 
 	 \paragraph{Day 24      \quad     }
	 	 \begin{itemize}[itemindent = 1em]
	 	 	% 重要
	 	 	\renewcommand\labelitemi{\makebox[0pt][l]{$\square$}\raisebox{.15ex}{\hspace{0.1em}$\checkmark$}}		
	 	 	
	 	 	\item    检查重读第3章。
	 	 	%\item    linux-muduo 2节
	 	 	%\item    Linux 系统编程2章
	 	 	%\item    Redis 数据库
	 	 	
	 	 	%\item   C++ http项目
	 	 	%\item   go 语言学习
	 	 	%\item   MySql 数据库
	 	 	%\item   UML 
	 	 	%\item   脚本语言 shell
	 	 	%\item   Python
	 	 	
	 	 	%日常
	 	 	\renewcommand\labelitemi{\makebox[0pt][l]{$\square$}\hspace{1em}}
	 	 	\item   国学  一节
	 	 	\item   读书  30 Mins	:\verb|三国演义 24-25章|
	 	 	\item   曾国藩家书:劝学
	 	 \end{itemize}
 	 \paragraph{Day 25      \quad     }
	 	 \begin{itemize}[itemindent = 1em]
	 	 	% 重要
	 	 	\renewcommand\labelitemi{\makebox[0pt][l]{$\square$}\raisebox{.15ex}{\hspace{0.1em}$\checkmark$}}		
	 	 	
	 	 	\item    检查重读第4章。
	 	 	%\item    linux-muduo 2节
	 	 	%\item    Linux 系统编程2章
	 	 	%\item    Redis 数据库
	 	 	
	 	 	%\item   C++ http项目
	 	 	%\item   go 语言学习
	 	 	%\item   MySql 数据库
	 	 	%\item   UML 
	 	 	%\item   脚本语言 shell
	 	 	%\item   Python
	 	 	
	 	 	%日常
	 	 	\renewcommand\labelitemi{\makebox[0pt][l]{$\square$}\hspace{1em}}
	 	 	\item   国学  一节
	 	 	\item   读书  30 Mins	:\verb|三国演义 26-27章|
	 	 	\item   曾国藩家书:劝学
	 	 \end{itemize}
 	 \paragraph{Day 26      \quad     }
	 	 \begin{itemize}[itemindent = 1em]
	 	 	% 重要
	 	 	\renewcommand\labelitemi{\makebox[0pt][l]{$\square$}\raisebox{.15ex}{\hspace{0.1em}$\checkmark$}}		
	 	 	
	 	 	\item    检查重读第5章。
	 	 	%\item    linux-muduo 2节
	 	 	%\item    Linux 系统编程2章
	 	 	%\item    Redis 数据库
	 	 	
	 	 	%\item   C++ http项目
	 	 	%\item   go 语言学习
	 	 	%\item   MySql 数据库
	 	 	%\item   UML 
	 	 	%\item   脚本语言 shell
	 	 	%\item   Python
	 	 	
	 	 	%日常
	 	 	\renewcommand\labelitemi{\makebox[0pt][l]{$\square$}\hspace{1em}}
	 	 	\item   国学  一节
	 	 	\item   读书  30 Mins	:\verb|三国演义 28-29章|
	 	 	\item   曾国藩家书:劝学
	 	 \end{itemize}
	 	 
	 	 怒冲1200元,抽了 10 把黄金武器,集齐了两套黄金套装。激动
 	 \paragraph{Day 27      \quad     }
	 	 \begin{itemize}[itemindent = 1em]
	 	 	% 重要
	 	 	\renewcommand\labelitemi{\makebox[0pt][l]{$\square$}\raisebox{.15ex}{\hspace{0.1em}$\checkmark$}}		
	 	 	
	 	 	\item    制作PPT 20页。
	 	 	%\item    linux-muduo 2节
	 	 	%\item    Linux 系统编程2章
	 	 	%\item    Redis 数据库
	 	 	
	 	 	%\item   C++ http项目
	 	 	%\item   go 语言学习
	 	 	%\item   MySql 数据库
	 	 	%\item   UML 
	 	 	%\item   脚本语言 shell
	 	 	%\item   Python
	 	 	
	 	 	%日常
	 	 	\renewcommand\labelitemi{\makebox[0pt][l]{$\square$}\hspace{1em}}
	 	 	\item   国学  一节
	 	 	\item   读书  30 Mins	:\verb|三国演义 30-31章|
	 	 	\item   曾国藩家书:劝学
	 	 \end{itemize}
	 	 
	 	 炫酷吊炸天,拿着新号去装逼。
 	 \paragraph{Day 28      \quad     }
	 	 \begin{itemize}[itemindent = 1em]
	 	 	% 重要
	 	 	\renewcommand\labelitemi{\makebox[0pt][l]{$\square$}\raisebox{.15ex}{\hspace{0.1em}$\checkmark$}}		
	 	 	
	 	 	\item     论文stuff
	 	 	%\item    linux-muduo 2节
	 	 	%\item    Linux 系统编程2章
	 	 	%\item    Redis 数据库
	 	 	
	 	 	%\item   C++ http项目
	 	 	%\item   go 语言学习
	 	 	%\item   MySql 数据库
	 	 	%\item   UML 
	 	 	%\item   脚本语言 shell
	 	 	%\item   Python
	 	 	
	 	 	%日常
	 	 	\renewcommand\labelitemi{\makebox[0pt][l]{$\square$}\hspace{1em}}
	 	 	\item   国学  一节
	 	 	\item   读书  30 Mins	:\verb|三国演义 32-33章|
	 	 	\item   曾国藩家书:劝学
	 	 \end{itemize}
	 	 这两天节奏有点乱,被游戏冲昏了头脑又。需要调整下,从明天开始需要重新跟着节奏走了。
	 	 
	 	 今天整体就是将论文的第五章总结与第2章的总结与碰撞检测进行了重写。然后认识了游戏里的几个好友。
	 	 
	 	 生死狙击新买的号 429. 然后相关的抽黄金武器1200 人名币。
	 	 
	 	 天哪,钱全让胡乱花了。
 	 \paragraph{Day 29      \quad     }   
	 	 \begin{itemize}[itemindent = 1em]
	 	 	% 重要
	 	 	\renewcommand\labelitemi{\makebox[0pt][l]{$\square$}\raisebox{.15ex}{\hspace{0.1em}$\checkmark$}}		
	 	 	
	 	 	\item    论文第2章检查改完,并对第三章的碰撞检测部分进行填加。
	 	 	\item    下午去学院找范老师批准选课系统。
	 	 	%\item   linux-muduo 2节
	 	 	%\item   Linux 系统编程2章
	 	 	%\item   Redis 数据库
	 	 	
	 	 	%\item   C++ http项目
	 	 	%\item   go 语言学习
	 	 	%\item   MySql 数据库
	 	 	%\item   UML 
	 	 	%\item   脚本语言 shell
	 	 	%\item   Python
	 	 	
	 	 	%日常
	 	 	\renewcommand\labelitemi{\makebox[0pt][l]{$\square$}\hspace{1em}}
	 	 	\item   国学  一节
	 	 	\item   读书  30 Mins	:\verb|三国演义 34-35章|
	 	 	\item   曾国藩家书:劝学
	 	 \end{itemize}
	 	 
		 完成一部分目标,将第2章改完,并将参考文献格式与文章内容填充完毕,接下来剩下的内容就为碰撞检测与语句检查了。
		 
		 下午接受到了4399 的面试通知电话,明天早上10:30, 好开心,希望自己可以顺利完成面试并被录取。

 	 \paragraph{Day 30      \quad     }
	 	 \begin{itemize}[itemindent = 1em]
	 	 	% 重要
	 	 	\renewcommand\labelitemi{\makebox[0pt][l]{$\square$}\raisebox{.15ex}{\hspace{0.1em}$\checkmark$}}		
	 	 	\item    4399 服务器开发面试。
	 	 	\item    碰撞检测算法。
	 	 	%\item   linux-muduo 2节
	 	 	%\item   Linux 系统编程2章
	 	 	%\item   Redis 数据库
	 	 	
	 	 	%\item   C++ http项目
	 	 	%\item   go 语言学习
	 	 	%\item   MySql 数据库
	 	 	%\item   UML 
	 	 	%\item   脚本语言 shell
	 	 	%\item   Python
	 	 	
	 	 	%日常
	 	 	\renewcommand\labelitemi{\makebox[0pt][l]{$\square$}\hspace{1em}}
	 	 	\item   国学  一节
	 	 	\item   读书  30 Mins	:\verb|三国演义 36-37章|
	 	 	\item   曾国藩家书:劝学
	 	 \end{itemize}
	 	 
	 	 
	 	 早上接受了4399 的面试,感觉自己好多东西都忘的差不多了,勉勉强强撑下来,希望可以被录取吧至少是个后台开发,这样至少以后会有一个更好的跳槽机会。
	 	 
	 	 下午睡了觉、因为昨天晚上激动的没睡好觉。
	 	 
	 	 晚上将基于位置的碰撞检测进行了完善。并将第2章存在的重复进行了修改,明天早上需要早起去学院进行给李老师汇报。
 	 \paragraph{Day 31      \quad     }
	 	 \begin{itemize}[itemindent = 1em]
	 	 	% 重要
	 	 	\renewcommand\labelitemi{\makebox[0pt][l]{$\square$}\raisebox{.15ex}{\hspace{0.1em}$\checkmark$}}		
	 	 	
	 	 	\item    汇报文章、报账事宜。
	 	 	%\item   linux-muduo 2节
	 	 	%\item   Linux 系统编程2章
	 	 	%\item   Redis 数据库
	 	 	
	 	 	%\item   C++ http项目
	 	 	%\item   go 语言学习
	 	 	%\item   MySql 数据库
	 	 	%\item   UML 
	 	 	%\item   脚本语言 shell
	 	 	%\item   Python
	 	 	
	 	 	%日常
	 	 	\renewcommand\labelitemi{\makebox[0pt][l]{$\square$}\hspace{1em}}
	 	 	\item   国学  一节
	 	 	\item   读书  30 Mins	:\verb|三国演义 38-39章|
	 	 	\item   曾国藩家书:劝学
	 	 \end{itemize}

		早上因为心理想着上次李老师说是让美丽姐帮着看所以就没有去汇报,希望不会坏事。
		
		晚上对论文进行了查重,只有6\%的重复率,心理实在感到高兴,剩下的工作就是修改语句,润色语句了。
		
\section{April}
 	 \paragraph{Day 1   修改第3章句子与12章重复句子    \quad     }
	 	 \begin{itemize}[itemindent = 1em]
	 	 	% 重要
	 	 	\renewcommand\labelitemi{\makebox[0pt][l]{$\square$}\raisebox{.15ex}{\hspace{0.1em}$\checkmark$}}		

	 	 	%\item   linux-muduo 2节
	 	 	%\item   Linux 系统编程2章
	 	 	%\item   Redis 数据库
	 	 	
	 	 	%\item   C++ http项目
	 	 	%\item   go 语言学习
	 	 	%\item   MySql 数据库
	 	 	%\item   UML 
	 	 	%\item   脚本语言 shell
	 	 	%\item   Python
	 	 	
	 	 	%日常
	 	 	\renewcommand\labelitemi{\makebox[0pt][l]{$\square$}\hspace{1em}}
	 	 	\item   国学尝试 一则 
	 	 	\item 	资治通鉴 一则 商鞅变法
	 	 	\item 	论语中庸 一则 子夏曰
	 	 	\item   读书  30 Mins	:\verb|三国演义 2-3章|
	 	 	\item   曾国藩家书:劝学
	 	 \end{itemize}
		早上美丽姐就将问题帮我反馈了回来,并且在细节上有待提高。
	
		下午与邦哥打了场篮球,锻炼了自己的身体,一周一场,也挺好。
		
		晚上回来洗了澡,然后打了几场枪战。
 	 \paragraph{Day 2       \quad     }
	 	 \begin{itemize}[itemindent = 1em]
	 	 	% 重要
	 	 	\renewcommand\labelitemi{\makebox[0pt][l]{$\square$}\raisebox{.15ex}{\hspace{0.1em}$\checkmark$}}		
	 	 	
	 	 	\item 	 找李书琴老师报账、找范老师帮忙确认选课系统。
	 	 	\item    修改论文第2章重复部分
	 	 	%\item   linux-muduo 2节
	 	 	%\item   Linux 系统编程2章
	 	 	%\item   Redis 数据库
	 	 	
	 	 	%\item   C++ http项目
	 	 	%\item   go 语言学习
	 	 	%\item   MySql 数据库
	 	 	%\item   UML 
	 	 	%\item   脚本语言 shell
	 	 	%\item   Python
	 	 	
	 	 	%日常
	 	 	\renewcommand\labelitemi{\makebox[0pt][l]{$\square$}\hspace{1em}}
	 	 	\item   国学尝试 一则
	 	 	\item 	资治通鉴 一则
	 	 	\item 	论语中庸 一则
	 	 	\item   读书  30 Mins	:\verb|三国演义 2-3章|
	 	 	\item   曾国藩家书:劝学
	 	 \end{itemize}
	 	 
	 	完成了其中的2项,早上9点睁开眼睛,到学院找李老师,可是早上不在,然后只是简单的将选课系统的问题进行了确认。下午回来把昨天的觉补了回来。晚上呢,将第一章、第二章的重复问题都逐一进行了修改,并将第三章的部分语句进行修改,除此之外完成了摘要部分的修改。
	 	
	 	接下来的问题就是如何完成胜于部分的修改和润色和细节检查。
	 	
	 	
 	 \paragraph{Day 3       \quad     }
	 	  \begin{itemize}[itemindent = 1em]
	 	  	% 重要
	 	  	\renewcommand\labelitemi{\makebox[0pt][l]{$\square$}\raisebox{.15ex}{\hspace{0.1em}$\checkmark$}}		
	 	  	
	 	  	\item 	 找李书琴老师报账。
	 	  	\item    修改论文第2章碰撞检测部分。
	 	  	%\item   linux-muduo 2节
	 	  	%\item   Linux 系统编程2章
	 	  	%\item   Redis 数据库
	 	  	
	 	  	%\item   C++ http项目
	 	  	%\item   go 语言学习
	 	  	%\item   MySql 数据库
	 	  	%\item   UML 
	 	  	%\item   脚本语言 shell
	 	  	%\item   Python
	 	  	
	 	  	%日常
	 	  	\renewcommand\labelitemi{\makebox[0pt][l]{$\square$}\hspace{1em}}
	 	  	\item   国学尝试 一则
	 	  	\item 	资治通鉴 一则 
	 	  	\item 	论语中庸 一则 夫子至邦
	 	  	\item   读书  30 Mins	:\verb|三国演义 2-3章|
	 	  	\item   曾国藩家书:劝学
	 	  \end{itemize}
 	 \paragraph{Day 4       \quad     }
 	 \paragraph{Day 5       \quad     }
 	 \paragraph{Day 6       \quad     }
		 	完成论文书写
		 	
		 	\begin{itemize}[itemindent = 1em]
		 		% 重要
		 		\renewcommand\labelitemi{\makebox[0pt][l]{$\square$}\raisebox{.15ex}{\hspace{0.1em}$\checkmark$}}	
		 			
		 		\item    修改论文第3章图部分。
		 		\item    修改论文第4章简介部分。
		 		%\item   linux-muduo 2节
		 		%\item   Linux 系统编程2章
		 		%\item   Redis 数据库
		 		
		 		%\item   C++ http项目
		 		%\item   go 语言学习
		 		%\item   MySql 数据库
		 		%\item   UML 
		 		%\item   脚本语言 shell
		 		%\item   Python
		 		
		 		%日常
		 		\renewcommand\labelitemi{\makebox[0pt][l]{$\square$}\hspace{1em}}
		 		\item   国学尝试 一则
		 		\item 	资治通鉴 一则 
		 		\item 	论语中庸 一则 夫子至邦
		 		\item   读书  30 Mins	:\verb|三国演义 2-3章|
		 		\item   曾国藩家书:劝学
		 	\end{itemize}
 	 \paragraph{Day 7       \quad     }
 	 \paragraph{Day 8       \quad     }
 	 \paragraph{Day 9       \quad     }
 	 \paragraph{Day 10      \quad     }
	 	 Hello:
	 	 
	 	 是时候让李老师看论文了。
 	 \paragraph{Day 11      \quad     }
	 	 未果
 	 \paragraph{Day 12      \quad     }
	 	 终于找到了李书琴老婆娘,而且是通过塞门缝的方式将文章让她看的,唉,不是亲身的学生可能都这样吧
	 	 
	 	 
 	 \paragraph{Day 13      \quad     }
	 	 面了网易, 大概问题如下(56分钟左右):
		 	 \begin{enumerate}[itemindent = 1em]
		 	 	\item c++ 虚函数的实现机制
		 	 	\item c++ 内存布局
		 	 	\item c++ 初始化流程
		 	 	\item c++ stl使用
		 	 	\item c++ 符号表,编译过程了解不
		 	 	
		 	 	\item net TCP/IP 过程
		 	 	\item net http 了解不
		 	 	
		 	 	\item os 内存调度算法
		 	 	\item os 死锁与避免算法
		 	 	
		 	 	\item 手撕代码:链表倒数第k个、二叉树的反转
		 	 	\item 动态规划了解不
		 	 	\item 空间加速算法知道不-> 图形学碰撞检测
		 	 \end{enumerate}
	 	 
	 	 
	 	 与父母通了视频
	 	 
	 	 收到4399 感谢邮件!
	 	 
	 	 晚上又听说不能用小论文的文字之类的,现在就剩改论文了
	 	 
 	 \paragraph{Day 14  7天改论文    \quad     }
	 	 完成第一章目的一样的修改
 	 
	 	 完成第三章修改
	 	 
	 	 完成摘要 修改
	 	 
	 	 并完成50\%第四章修改
	 	 
 	 \paragraph{Day 15      \quad     }
	 	 完成第四章修改
	 	 
	 	 完成第五章修改
	 	 
 	 \paragraph{Day 16  接到网易的2面面试通知    \quad     }
	 	 Hello:
	 	 
	 	 早上还在被窝的时候,大概10点的样子(10:17)接到hr小哥哥的电话,说是通过了一面,希望到杭州参加现场面试并报销车票,激动,赶集起床复习。
	 	 
	 	 完成了对旋转平移等变换原理的学习
	 	 
 	 \paragraph{Day 17  订了票    \quad     }
	 	 Hello:
	 	 
	 	 论文见面给了李书琴老师
	 	 
	 	 完成了c++装载和内存管理的学习
	 	 
	 	 订了杭州的列车票,订了杭州两天晚上的房间。
	 	 
 	 \paragraph{Day 18      \quad     }
	 	 准备面试简历。
	 	 
	 	 完成STL 和 设计模式的复习,对TCP进行复习。
	 	 
 	 \paragraph{Day 19  出发去杭州    \quad     }
		 Hello:
		 
		 10点出发先到打印室把这两天总结的图形学相关知识打印出来。
		 
		 11点到达校门口吃了小份的炒细面。
		 
		 在去西安北的高铁上大概把图形学相关的知识进行了浏览学习。
		 
		 在13:36去杭州的高铁上大概把带去的笔记都翻了个遍,坐着8个小时确实时间挺长的,所以为了充分的利用这段时间这能在火车上进行复习了。
		 
		 晚上20:40 坐上杭州的地铁,21:50左右完成了房间的手续,10点出来吃了份挂面,然后去踩网易公司的大体位置点。
 	 \paragraph{Day 20  面试    \quad     }
	 	 Hello:
	 	 
	 	 10点大概洗漱完毕,好好的把自己收拾了一番,听着歌对自己的面试流程又模拟了几遍,并发现高铁票所剩无几,赶紧订了票。
	 	 
	 	 13点左右出去吃了水饺。
	 	 
	 	 14点10从鑫耀假日酒店出发坐滴滴去网易,并预约了第二天早上去杭州东的滴滴。
	 	 
	 	 14:25左右签的到,由于属于校招补招,耽误到15:30左右开始进行面试。
	 	 
	 	 在2面的过程中还是非常紧张的,这个与等待的时间是有一定关系的。首先按照惯例自我介绍,完成自我介绍完后我把笔记之类的进行了展示。然后就开始了正式的问答环节。
	 	 问题如下所示:
		 	 \begin{enumerate}[itemindent = 1em]
		 	 	\item \textbf{c++} inline 实现机制
		 	 	\item \textbf{c++} 虚函数实现机制
		 	 	\item \textbf{c++} 内存布局
		 	 	\item \textbf{c++} 引用计数->循环引用问题
		 	 	\item \textbf{代码} Faccinaci 数列
		 	 	\item \textbf{代码} 递归区间段求和
		 	 	\item \textbf{os} 消息通信机制
		 	 	\item \textbf{os} 管道通信的技术要点
		 	 	\item \textbf{net} socket服务器的大体流程
		 	 	\item \textbf{net} tcp/ip 流程
		 	 	\item \textbf{net} accept 函数发生在哪步之后
		 	 	\item \textbf{net} http了解不
		 	 	\item \textbf{开放性问题求解}:10G数据、2G内存、求中位数->更注重中间的分析阶段
		 	 \end{enumerate}
		2面完后,老铁估计去约3面面试官,然后就去3楼3面了。首先也是自我介绍,然后忘记展示自己笔记了。大体问题如下:
			\begin{enumerate}[itemindent = 1em]
				\item 基础代码能力:逆转单链表
				\item 性能测试工具:插桩算法实现统计函数运行时间。
				\item 洗牌算法:分段算法
				\item 笔记
				\item 薪资
				\item 问题:游戏研发主要涉及哪些技术->数据结构、语言、引擎、设计模式等。
			\end{enumerate} 	 
		 3面后面试官一起下楼,给我提建议说要学会提问。然后建议去参观下网易大楼。
		 
		 
		 不得不说,这两位面试官的洞察力真是强,思路一提出来马上能发现其中的错误,犀利。
		 总得来说,在面试中需要以下几项技能。
			 \begin{itemize}
			 	\item 基础理论的掌握与更深层次的主动学习
			 	\item 实际问题的求解能力:学习迁移能力
			 	\item 沟通能力,能够正确、准确、有效的提出问题
			 \end{itemize}

		回到宾馆后,先后与父母、小喵、好友们分享了其中的心情。
 	 \paragraph{Day 21  回来    \quad     }
	 	 Hello:
	 	 
	 	 晚上一直忐忑的睡不着,大概就从2:30 睡到 4:40 吧,然后就起床出发了。
	 	 
	 	 在5:04 的时候接到了昨天预约的司机电话,本来还在担心的问题没有了,于是将正在吃的粥打包出发了,到达杭州东的大概时间为5:40。进去后吃了麦当劳并为中午买了一个汉堡。
	 	 
	 	 下午4点左右回到学校,完成了车票的报销流程。
	 	 
	 	 五点吃了煮馍、睡觉。
	 	 
	 	 晚上起来后叫着四个哥们出去吃了烧烤,分享了下心事之类的
 	 \paragraph{Day 22  论文签字截至   \quad     }
 	 \paragraph{Day 23  毕业资格审查   \quad     }
 	 \paragraph{Day 24      \quad     }
 	 \paragraph{Day 25  预答辩    \quad     }
 	 \paragraph{Day 26      \quad     }
 	 \paragraph{Day 27      \quad     }
 	 \paragraph{Day 28      \quad     }
 	 \paragraph{Day 29      \quad     }   
 	 \paragraph{Day 30  论文成型 截止日期    \quad     }
\section{May}
 	 \paragraph{Day 1   盲审    \quad     }
 	 \paragraph{Day 2       \quad     }
 	 \paragraph{Day 3       \quad     }
 	 \paragraph{Day 4       \quad     }
 	 \paragraph{Day 5       \quad     }
 	 \paragraph{Day 6       \quad     }
 	 \paragraph{Day 7       \quad     }
 	 \paragraph{Day 8       \quad     }
 	 \paragraph{Day 9       \quad     }
 	 \paragraph{Day 10      \quad     }
 	 \paragraph{Day 11      \quad     }
 	 \paragraph{Day 12      \quad     }
 	 \paragraph{Day 13      \quad     }
 	 \paragraph{Day 14      \quad     }
 	 \paragraph{Day 15      \quad     }
 	 \paragraph{Day 16      \quad     }
 	 \paragraph{Day 17      \quad     }
 	 \paragraph{Day 18      \quad     }
 	 \paragraph{Day 19      \quad     }
 	 \paragraph{Day 20   正式答辩   \quad     }
 	 \paragraph{Day 21      \quad     }
 	 \paragraph{Day 22      \quad     }
 	 \paragraph{Day 23      \quad     }
 	 \paragraph{Day 24      \quad     }
 	 \paragraph{Day 25      \quad     }
 	 \paragraph{Day 26      \quad     }
 	 \paragraph{Day 27      \quad     }
 	 \paragraph{Day 28      \quad     }
 	 \paragraph{Day 29      \quad     }   
 	 \paragraph{Day 30      \quad     }
 	 \paragraph{Day 31      \quad     }
\section{June}
 	 \paragraph{Day 1       \quad     }
 	 \paragraph{Day 2       \quad     }
 	 \paragraph{Day 3       \quad     }
 	 \paragraph{Day 4       \quad     }
 	 \paragraph{Day 5       \quad     }
 	 \paragraph{Day 6       \quad     }
 	 \paragraph{Day 7       \quad     }
 	 \paragraph{Day 8       \quad     }
 	 \paragraph{Day 9       \quad     }
 	 \paragraph{Day 10      \quad     }
 	 \paragraph{Day 11      \quad     }
 	 \paragraph{Day 12      \quad     }
 	 \paragraph{Day 13      \quad     }
 	 \paragraph{Day 14      \quad     }
 	 \paragraph{Day 15      \quad     }
 	 \paragraph{Day 16      \quad     }
 	 \paragraph{Day 17      \quad     }
 	 \paragraph{Day 18      \quad     }
 	 \paragraph{Day 19      \quad     }
 	 \paragraph{Day 20      \quad     }
 	 \paragraph{Day 21      \quad     }
 	 \paragraph{Day 22      \quad     }
 	 \paragraph{Day 23      \quad     }
 	 \paragraph{Day 24      \quad     }
 	 \paragraph{Day 25      \quad     }
 	 \paragraph{Day 26      \quad     }
 	 \paragraph{Day 27      \quad     }
 	 \paragraph{Day 28      \quad     }
 	 \paragraph{Day 29      \quad     }   
 	 \paragraph{Day 30      \quad     }
\section{July}
 	 \paragraph{Day 1   入职前准备    \quad     }
 	 \paragraph{Day 2       \quad     }
 	 \paragraph{Day 3       \quad     }
 	 \paragraph{Day 4       \quad     }
 	 \paragraph{Day 5       \quad     }
 	 \paragraph{Day 6       \quad     }
 	 \paragraph{Day 7   入职截止日期    \quad     }
 	 \paragraph{Day 8       \quad     }
 	 \paragraph{Day 9       \quad     }
 	 \paragraph{Day 10      \quad     }
 	 \paragraph{Day 11      \quad     }
 	 \paragraph{Day 12      \quad     }
 	 \paragraph{Day 13      \quad     }
 	 \paragraph{Day 14      \quad     }
 	 \paragraph{Day 15      \quad     }
 	 \paragraph{Day 16      \quad     }
 	 \paragraph{Day 17      \quad     }
 	 \paragraph{Day 18      \quad     }
 	 \paragraph{Day 19      \quad     }
 	 \paragraph{Day 20      \quad     }
 	 \paragraph{Day 21      \quad     }
 	 \paragraph{Day 22      \quad     }
 	 \paragraph{Day 23      \quad     }
 	 \paragraph{Day 24      \quad     }
 	 \paragraph{Day 25      \quad     }
 	 \paragraph{Day 26      \quad     }
 	 \paragraph{Day 27      \quad     }
 	 \paragraph{Day 28      \quad     }
 	 \paragraph{Day 29      \quad     }   
 	 \paragraph{Day 30      \quad     }
 	 \paragraph{Day 31      \quad     }
\section{August}
 	 \paragraph{Day 1       \quad     }
 	 \paragraph{Day 2       \quad     }
 	 \paragraph{Day 3       \quad     }
 	 \paragraph{Day 4       \quad     }
 	 \paragraph{Day 5       \quad     }
 	 \paragraph{Day 6       \quad     }
 	 \paragraph{Day 7       \quad     }
 	 \paragraph{Day 8       \quad     }
 	 \paragraph{Day 9       \quad     }
 	 \paragraph{Day 10      \quad     }
 	 \paragraph{Day 11      \quad     }
 	 \paragraph{Day 12      \quad     }
 	 \paragraph{Day 13      \quad     }
 	 \paragraph{Day 14      \quad     }
 	 \paragraph{Day 15      \quad     }
 	 \paragraph{Day 16      \quad     }
 	 \paragraph{Day 17      \quad     }
 	 \paragraph{Day 18      \quad     }
 	 \paragraph{Day 19      \quad     }
 	 \paragraph{Day 20      \quad     }
 	 \paragraph{Day 21      \quad     }
 	 \paragraph{Day 22      \quad     }
 	 \paragraph{Day 23      \quad     }
 	 \paragraph{Day 24      \quad     }
 	 \paragraph{Day 25      \quad     }
 	 \paragraph{Day 26      \quad     }
 	 \paragraph{Day 27      \quad     }
 	 \paragraph{Day 28      \quad     }
 	 \paragraph{Day 29      \quad     }   
 	 \paragraph{Day 30      \quad     }
 	 \paragraph{Day 31      \quad     }
\section{September}
 	 \paragraph{Day 1       \quad     }
 	 \paragraph{Day 2       \quad     }
 	 \paragraph{Day 3       \quad     }
 	 \paragraph{Day 4       \quad     }
 	 \paragraph{Day 5       \quad     }
 	 \paragraph{Day 6       \quad     }
 	 \paragraph{Day 7       \quad     }
 	 \paragraph{Day 8       \quad     }
 	 \paragraph{Day 9       \quad     }
 	 \paragraph{Day 10      \quad     }
 	 \paragraph{Day 11      \quad     }
 	 \paragraph{Day 12      \quad     }
 	 \paragraph{Day 13      \quad     }
 	 \paragraph{Day 14      \quad     }
 	 \paragraph{Day 15      \quad     }
 	 \paragraph{Day 16      \quad     }
 	 \paragraph{Day 17      \quad     }
 	 \paragraph{Day 18      \quad     }
 	 \paragraph{Day 19      \quad     }
 	 \paragraph{Day 20      \quad     }
 	 \paragraph{Day 21      \quad     }
 	 \paragraph{Day 22      \quad     }
 	 \paragraph{Day 23      \quad     }
 	 \paragraph{Day 24      \quad     }
 	 \paragraph{Day 25      \quad     }
 	 \paragraph{Day 26      \quad     }
 	 \paragraph{Day 27      \quad     }
 	 \paragraph{Day 28      \quad     }
 	 \paragraph{Day 29      \quad     }   
 	 \paragraph{Day 30      \quad     }
\section{October}
 	 \paragraph{Day 1       \quad     }
 	 \paragraph{Day 2       \quad     }
 	 \paragraph{Day 3       \quad     }
 	 \paragraph{Day 4       \quad     }
 	 \paragraph{Day 5       \quad     }
 	 \paragraph{Day 6       \quad     }
 	 \paragraph{Day 7       \quad     }
 	 \paragraph{Day 8       \quad     }
 	 \paragraph{Day 9       \quad     }
 	 \paragraph{Day 10      \quad     }
 	 \paragraph{Day 11      \quad     }
 	 \paragraph{Day 12      \quad     }
 	 \paragraph{Day 13      \quad     }
 	 \paragraph{Day 14      \quad     }
 	 \paragraph{Day 15      \quad     }
 	 \paragraph{Day 16      \quad     }
 	 \paragraph{Day 17      \quad     }
 	 \paragraph{Day 18      \quad     }
 	 \paragraph{Day 19      \quad     }
 	 \paragraph{Day 20      \quad     }
 	 \paragraph{Day 21      \quad     }
 	 \paragraph{Day 22      \quad     }
 	 \paragraph{Day 23      \quad     }
 	 \paragraph{Day 24      \quad     }
 	 \paragraph{Day 25      \quad     }
 	 \paragraph{Day 26      \quad     }
 	 \paragraph{Day 27      \quad     }
 	 \paragraph{Day 28      \quad     }
 	 \paragraph{Day 29      \quad     }   
 	 \paragraph{Day 30      \quad     }
 	 \paragraph{Day 31      \quad     }
\section{November}
 	 \paragraph{Day 1       \quad     }
 	 \paragraph{Day 2       \quad     }
 	 \paragraph{Day 3       \quad     }
 	 \paragraph{Day 4       \quad     }
 	 \paragraph{Day 5       \quad     }
 	 \paragraph{Day 6       \quad     }
 	 \paragraph{Day 7       \quad     }
 	 \paragraph{Day 8       \quad     }
 	 \paragraph{Day 9       \quad     }
 	 \paragraph{Day 10      \quad     }
 	 \paragraph{Day 11      \quad     }
 	 \paragraph{Day 12      \quad     }
 	 \paragraph{Day 13      \quad     }
 	 \paragraph{Day 14      \quad     }
 	 \paragraph{Day 15      \quad     }
 	 \paragraph{Day 16      \quad     }
 	 \paragraph{Day 17      \quad     }
 	 \paragraph{Day 18      \quad     }
 	 \paragraph{Day 19      \quad     }
 	 \paragraph{Day 20      \quad     }
 	 \paragraph{Day 21      \quad     }
 	 \paragraph{Day 22      \quad     }
 	 \paragraph{Day 23      \quad     }
 	 \paragraph{Day 24      \quad     }
 	 \paragraph{Day 25      \quad     }
 	 \paragraph{Day 26      \quad     }
 	 \paragraph{Day 27      \quad     }
 	 \paragraph{Day 28      \quad     }
 	 \paragraph{Day 29      \quad     }   
 	 \paragraph{Day 30      \quad     }
\section{December}
 	 \paragraph{Day 1       \quad     }
 	 \paragraph{Day 2       \quad     }
 	 \paragraph{Day 3       \quad     }
 	 \paragraph{Day 4       \quad     }
 	 \paragraph{Day 5       \quad     }
 	 \paragraph{Day 6       \quad     }
 	 \paragraph{Day 7       \quad     }
 	 \paragraph{Day 8       \quad     }
 	 \paragraph{Day 9       \quad     }
 	 \paragraph{Day 10      \quad     }
 	 \paragraph{Day 11      \quad     }
 	 \paragraph{Day 12      \quad     }
 	 \paragraph{Day 13      \quad     }
 	 \paragraph{Day 14      \quad     }
 	 \paragraph{Day 15      \quad     }
 	 \paragraph{Day 16      \quad     }
 	 \paragraph{Day 17      \quad     }
 	 \paragraph{Day 18      \quad     }
 	 \paragraph{Day 19      \quad     }
 	 \paragraph{Day 20      \quad     }
 	 \paragraph{Day 21      \quad     }
 	 \paragraph{Day 22      \quad     }
 	 \paragraph{Day 23      \quad     }
 	 \paragraph{Day 24      \quad     }
 	 \paragraph{Day 25      \quad     }
 	 \paragraph{Day 26      \quad     }
 	 \paragraph{Day 27      \quad     }
 	 \paragraph{Day 28      \quad     }
 	 \paragraph{Day 29      \quad     }   
 	 \paragraph{Day 30      \quad     }
 	 \paragraph{Day 31      \quad     }
	
	
\end{document} 
