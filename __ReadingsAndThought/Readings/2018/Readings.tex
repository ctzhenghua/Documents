\documentclass[UTF8,a4paper,8pt]{ctexbook} 

\usepackage{graphicx}%学习插入图
\usepackage{verbatim}%学习注释多行
\usepackage{booktabs}%表格
\usepackage{geometry}%图片
\usepackage{amsmath} 
\usepackage{amssymb}
\usepackage{listings}%代码
\usepackage{xcolor}  %颜色
\usepackage{enumitem}%列表格式
\usepackage{hyperref}
\CTEXsetup[format+={\flushleft}]{section}

\geometry{left=1.6cm,right=1.8cm,top=2cm,bottom=1.7cm} %设置文章宽度

\pagestyle{plain} 		  %设置页面布局
\author{郑华}
\title{2017读书笔记}
%代码效果定义
\definecolor{codegreen}{rgb}{0,0.6,0}
\definecolor{codegray}{rgb}{0.5,0.5,0.5}
\definecolor{codepurple}{rgb}{0.58,0,0.82}
\definecolor{backcolour}{rgb}{0.95,0.95,0.92}

\lstdefinestyle{mystyle}{
	backgroundcolor=\color{backcolour},   
	commentstyle=\color{codegreen},
	keywordstyle=\color{magenta},
	numberstyle=\tiny\color{codegray},
	stringstyle=\color{codepurple},
	basicstyle=\footnotesize,
	breakatwhitespace=false,         
	breaklines=true,                 
	captionpos=b,                    
	keepspaces=true,                 
	%numbers=left,                    
	%numbersep=5pt,                  
	showspaces=false,                
	showstringspaces=false,
	showtabs=false,                  
	tabsize=2
}
\lstset{style=mystyle, escapeinside=``}

\begin{document}          %正文排版开始
	\maketitle
	\tableofcontents
	

\chapter{论语}
	\section{为人励志}
		\paragraph{学习}
			\subparagraph{原文}\verb|->|
			
				学而时习之,不亦说乎? 有朋自远方来,不亦说乎?人不知而不愠,不亦君子乎。
			
				学而不思则罔,思而不学则殆
				
			\subparagraph{理解}\verb|->|
			
				首先学习什么是个问题,大千世界,万物皆可学,而孔丘所说乃为人处事。
				
				\textbf{学什么} 原来学并不是单纯的学习知识这么简单,而是惠及万千,包含了太多,年轻的时候还是不懂,当懂得时候已经不再年轻。
				
				\textbf{时习} 对于学的东西光了,那么习的范围也就不局限于复习知识了,其中就包括自己的所做所为、所经历的事情等,眼光局限了那么就可能体会不到了, 这时候就像自己翻阅笔记时,看到自己曾经做的事然后体会种种心路历程一般,受教颇深。
				
				\textbf{有朋自远方来} 人生本来得一知己少之又少,且行且珍惜
				
				\textbf{人不知而不愠} 这个知并不是知其名的知,而是理解与懂得,这是一种境界,就好比自己不被别人理解,然而却能平然处之,乃为一种气度、仁。
		
		\paragraph{吾日三省吾身}
			\subparagraph{原文}\verb|->|
		
				曾子曰:吾日三省吾身:为人谋而不忠乎?与朋友交而不信乎?传不习乎?
			
			\subparagraph{理解}\verb|->|
			
				\textbf{为人谋而不忠乎} 替别人办事是不是尽心尽力了。
				
				\textbf{与朋友交而不信乎} 与朋友相处是不是诚实守信了。
				
				\textbf{传不习乎} 对老师传授的课程是不是用心复习了呢?
			
		\paragraph{谨而信,行有馀而学}
			\subparagraph{原文}\verb|->|
		
				弟子入则孝,出则弟,谨而信,泛爱众,而亲仁。行有余力,则以学文。
			
			
				子曰:“先行其言而后从之”
				
				
				人而无信,不知其可也,大车无倪(ni),小车无軏(yue),其何以行之哉。
				
			\subparagraph{理解}\verb|->|
			
				\textbf{谨而信} 说话要谨慎,言而有信。谨慎是一种修养,只有谨慎的人,才不会对他人做出许诺,不会信口开河。一旦说出口,就要努力、尽心竭力的完成,做到言而有信。
	
				\textbf{先行、后从},对于你说的话,先实行了,然后再说出来
				
				
		\paragraph{待人接物}
			\subparagraph{原文}\verb|->|
			
				有子曰:“信近于义,言可复也; 恭近于礼,远耻辱也; 因不失其亲, 亦可宗也。”
			
				子夏曰:“贤贤易色; 事父母,能竭其力;事君,能致其身;与朋友交,言而有信。虽曰未学,吾必谓之学矣。”
				
				子曰:“句子不重则不威,学则不固,主忠信。无友不如己者。过则勿惮改。”
			\subparagraph{理解}\verb|->|
			
				\textbf{接物:信近于义} 做事应该强调:合理、恰当、正义,如果不合理不正当,即便做出承诺,这种承诺也是不合适的,勉强自己去做,固然守全自己“信”之名,但是所做之事会危害社会他人,做这样的事自己是会有心理负担的。 所以不义之信很难兑现,而正义的诺言可以很轻松的实现。
				
				\textbf{待人:恭近于礼} 一个人若是能够处处做到恭谨有礼,尊重他人,自然会赢得别人的信任与尊重,不受到他人的羞辱。
			
				\textbf{概括} 这样做没有失去仁德的方向,因此可以效法和受到尊重。
			
		\paragraph{-君子好学}
			\subparagraph{原文}\verb|->|
			
				子曰: “君子食无求饱,居无求安,敏于事而慎于言,就有道而正焉,可谓好学也已。”	
			
			\subparagraph{理解}\verb|->|
		
		\paragraph{告诸往而知来者}
			\subparagraph{原文}\verb|->|
				诗经\verb|-> |如切如磋、如琢如磨。
			
			\subparagraph{理解}\verb|->|
			
		\paragraph{知己与知人(态度)}
			\subparagraph{原文}\verb|->|
			
				子曰:“不患人之不己知,患不知人也。”
			
				视其所以,观其所由,察其所安。
			
			\subparagraph{理解}\verb|->|
				意思是不怕别人不了解自己,只怕自己不了解别人。
	
				\textbf{别人不了解自己,是别人的事情; 自己不了解别人是自己的事情;我们能做好的是自己的态度},努力、真诚的认知、结识,而不是一味的误解积怨,须充分认识到这个人或者事情。
		
				\textbf{了解一个人的途径}:看一个人的所做所为,考察他的处事的动机,了解他心安于什么事情。
				
		\paragraph{追求与自我完善}
			\subparagraph{原文}\verb|~|
			
				子曰:“吾十有五而志于学,三十而立,四十而不惑,五十而知天命,六十而耳顺,七十而从心所欲,不逾矩。”
				
				成事不说,遂事不谏,既往不咎。
				
			\subparagraph{理解}\verb|~|
			
				在追求的过程中完善自我,不论修道还是自我提升都是一个漫长并且循序渐进的过程,需要常年累月的积累与反思提升。
				
				已经过去的事不用解释了,已经完成的事不要再劝谏了,已经过去的事就不要再追究了。
				
				但是与此同时,我们对待自己需要时刻反省。
				
	\section{处事之道}
		\paragraph{过犹不及}
				子曰:“攻乎异端,斯害也已”
				
				乐而不淫,哀而不伤。
			
			\subparagraph{理解} 做事情过或不及,都是祸害啊!
			
			原意是男女情爱是很正常很自然的事情,但是情感的表露应当以礼节之,不应过分快乐或哀伤,要把握好自己在情感上的尺度。
			
			应用到全面处事上的话,道理也同样适应,情感的表露应该有所节制,不应该过分的表现出。这也是情商的一种表现。		
		
		\paragraph{言寡尤,行寡悔}
			\subparagraph{原文}\verb|~|
			
				子曰:“多闻阙疑,慎言其馀,则寡尤。多见阙殆,慎行其馀,则寡悔。言寡尤,行寡悔,禄在其中矣。”
			
			\subparagraph{理解}\verb|~|
			
				\begin{itemize}[itemindent = 1em]
					\item \textbf{阙}:quē,空着,搁置不要动。\textbf{尤}:过错,过失。多去听一些道理,\textit{将那些尚存疑虑的地方先暂且搁置下来},\textit{对那些看似没有疑问的地方谨慎发表言论,过错自然就会少}。
					\item \textit{多去见识一些事情,将那些让内心感到不安的地方先暂且搁置下来,对其余那些看似没有问题的部分去谨慎施行},则令你会事后后悔的事情就会减少。
					\item 如果一个人在说话里面少了很多的指责、抱怨,在行为中少了很多让自己后悔的经验,这个人出去做官做事他就能成功了。
				\end{itemize}
			
			\verb|减少过失的方法->|
				\begin{enumerate}[itemindent =2em]
						\item 多闻充识
						\item 对于不了解的事情不要妄下评论
						\item 对于自己了解的东西谨慎发表言论					
				\end{enumerate}

				
			\verb|减少后悔的方法->|
				\begin{enumerate}[itemindent = 2em]
					\item  多观察
					\item  把不懂的事情放下慢慢研习
					\item  对于自己懂的事情谨慎的践行									
				\end{enumerate}			

				
			\verb|成功的途径->|
				\begin{enumerate}[itemindent = 2em]
					\item 说话很少有过错
					\item 行动很少有后悔								
				\end{enumerate}

			
		\paragraph{在此刻隐忍中消除远处的忧愁} 篱笆与钉子
		
			无远虑必有近忧
		
		\paragraph{为与不为}
			\subparagraph{原文}\verb|~|
			
			 	非其鬼而祭之,谄(chǎn )也。见义不为,无勇也。
			
			\subparagraph{理解}
				\begin{itemize}[itemindent=1em]
					\item 不是你当祭的鬼而祭他,这是你存心谄媚.
					\item \textbf{遇见你该当做的事不做,这是你没勇气。}
					\item 各司其职
				\end{itemize}
		
		\paragraph{文化、形式的重要}
			\subparagraph{原文}\verb|~|
				
				子曰:“夷狄之有君,不如诸夏之亡也”
			
				尔爱其羊,我爱其礼。
				
			\subparagraph{理解}
				夷狄有君主,还不如中原之地的没有君主而讲礼节的。
				
				你爱惜那只羊,我则爱惜那种礼。
		
		\paragraph{君臣之道}
			\subparagraph{原文}\verb|~|
			
				定公问:“君使臣,臣事君,如之何?” 孔子对曰:“君使臣以礼,臣事君以忠。”
			
		
		\paragraph{三思有度}
			三思而后行,而往往不能思虑太多,否则可能会成为犹豫不决的表现,反而错失好多。
			
			\subparagraph{原文}\verb|->|
				
				季子文三思而后行。子闻之,曰:“再,斯可矣。”	
				
			\subparagraph{翻译}\verb|->|
				
				季文子办事,要反复考虑多次后才行动。孔子听到后,说:“考虑两次就可以了”	
			
			\subparagraph{理解}\verb|->|
				
				凡事都应该有一个度,慎重如果过了头就变成懦弱了。“三思而后行” 是一句传世名言,很多人奉为处世法则,但是,凡事的确应该考虑利和弊,但是思考太多,便会犹豫不决,可能错失行动的时机。
				
				
				文子生平盖祸福利害之计太明,故其美恶两不相掩,皆三思之病也。其思之至三者,特以世故太深,过为谨慎;然其流弊将至利害询一己之私矣。
				
		\paragraph{真正的直率、不为虚名}
			\subparagraph{原文}\verb|->|
			
				子曰:“熟谓微生高直? 或乞醯(xi)焉”
				
			\subparagraph{理解}
				微生高 从邻居家借醋 给前来借醋的人,有做做之嫌,不是真正的直率。孔子表明:“怎样就是怎样的直到,不为虚名牵累自己”	
				
		
\chapter{中庸}
	\textbf{中}:不偏之谓中
	
	\textbf{庸}:不易之谓庸
	
	\section{中、庸、慎其独}
		\paragraph{原文}\verb|->|
		
			天命谓之性,率性之谓道,修道之谓教。
			
			道也者,不可须臾离也,可离非道也。是故君子戒慎乎其所不睹,恐惧乎其所不闻。莫见乎隐,莫显于微。故君子慎其独也。
		
			喜怒哀乐之未发,谓之中;发而皆中节,谓之和。 中也者,天下之大本也;和也者,天下之达道也。致中和,天地位焉,万物育焉。
			
		\paragraph{理解}\verb|->|
			
			\subparagraph{率性谓之道} 根据圣人之道,充分发挥自己的主观能动性,感化周围民众,磨砺杰出才能。从自己的兴趣出发,完善自己的性格,提高自己的能力,在此基础上完成建功立业。
			
			\subparagraph{君子慎其独} 道是一个持续的过程,正如中庸的庸所阐述的意思一般-“不易之谓庸”,即稳定不变持续的过程。所以在修道的过程中,不论是在人前、还是人后都需保持统一。 在原文中意思是没有人的时候是最易放松自我的,最微小的东西最容易显示的时候,因此在独处是君子最为重视的地方。
			
			诚于人者,清白无碍; 诚于心者,问心无愧,可直面天地鬼神。诚于心,形于外,这样才称得上真正的慎独。
	
	\section{君子中庸}
			\paragraph{原文}\verb|->|
				
				仲尼曰:“君子中庸,小人反中庸。君子之中庸也,君子而时中;小人之反中庸也,小人而无忌惮也。”
				
			\paragraph{理解}\verb|->|
				\subparagraph{译文}君子之所以能合乎中庸的道理,是因为君子能随时守住中道,无过无不及;小人之所以违反中庸,是因为小人无所顾忌,肆无忌惮。

				\subparagraph{君子中庸}仁人君子拥有博大的胸襟,卓越的见识,远大的目光,能够从长远考虑问题事情,从细微处把握事情的发展动向与本质特点,并及时做出判断。他们深悟中庸之道,做事情即不偏激,也不极端,既不专断,也不投机,而是从论理大道出发,包容众长。
				
				\subparagraph{小人反中庸}他们不包容让人,也不能和他人分享,忽视他人的存在。他们往往目光短浅,只看重眼前利益,急功近利,以短暂性的顷刻满足为一切处事的出发点和原则,这样就违背了中庸之道,和这种大智慧就越来越远。
				
·		\section{鲜能知味}
			\paragraph{原文}\verb|->|
				
				子曰:“道之不行也,我知之矣:知者过之,愚者不及也。 道之不明也,我之知也:贤者过之,不肖者不及也。人莫不饮食也,鲜能知味也。”
				
			\paragraph{理解}\verb|->|
			
				\subparagraph{知者过、愚者不及}智者别画蛇添足,掌握好适度原则,免得造成过犹不及的悔恨。
				
				\subparagraph{人莫不饮食也,鲜能知味也} 人生开门七件事,柴米油盐酱醋茶,讲的就单只是一个“吃”字。只可惜虽然人人吃饭,却不是人人识得此中真味。
				
				但是我还觉得,此处还存在另外一个意思,即如果不实践,很少能体会其中味道,正如人如果不吃,怎么知道其中的味道一样。而这也复合了孔夫子在推行中庸之道这一时间段。
	
	\section{自作聪明,不知避祸}
			\paragraph{原文}\verb|->|
					
				子曰:“人皆曰‘予知’,驱而纳诸罟擭陷阱之中,而莫之知避也。人皆曰‘予知’,择乎中庸,而不能斯月守也。”
			\paragraph{理解}\verb|->|
				孔子说,人人都说自己是明智的,但是在利欲的驱使下,他们却都像禽兽那样落入网笼陷阱中,连如何躲避都不知道。
				人人都说自己是明智的,但是选择了中庸之道却连一个月也都不能坚持下去。
			
				
				\subparagraph{自作聪明} 自作聪明往往自以为是,没有自知之明,所以做起事情来往往一意孤行,我行我素,他们在自己所为的聪明的蒙蔽下,已经不知道什么事情该做,什么事情不该做,也不知道事情做到什么程度就已经恰到好处,在做下去就不完美了。
				
				\subparagraph{去留无意,宠辱不惊}深刻领悟中庸之道。	
\chapter{资治通鉴故事}
	\section{七国争雄}
		\subsection{赵魏韩三分晋}
			\subparagraph{总结}有才无德必亡奕!
			
			\subparagraph{故事}
				  初, 智 宣 子 将以 瑶 为 后, 智 果 曰:“ 不如 宵 也。 瑶 之 贤 于 人者 五, 其 不 逮 者 一 也。 美 鬓 长大 则 贤, 射 御 足 力 则 贤, 伎 艺 毕 给 则 贤, 巧 文 辩 惠 则 贤, 强 毅 果敢 则 贤; 如是 而 甚 不仁。 夫 以其 五 贤 陵 人 而以 不仁 行 之, 其 谁能 待 之? 若果 立 瑶 也, 智 宗 必 灭。” 弗 听。 智 果 别 族 于 太史, 为辅 氏。
				
				  赵 简 子 之子, 长 曰 伯 鲁, 幼 曰 无 恤。 将 置后, 不知 所立, 乃 书 训戒 之辞 于 二 简, 以 授 二 子 曰:“ 谨 识 之!” 三年 而 问 之, 伯 鲁 不能 举 其辞; 求 其 简, 已 失之 矣。 问 无 恤, 诵 其辞 甚 习; 求 其 简, 出 诸 袖 中 而 奏 之。 于是 简 子 以 无 恤 为 贤, 立 以为 后。 简 子 使 尹 铎 为 晋 阳, 请 曰:“ 以为 茧丝 乎? 抑 为 保障 乎?” 简 子 曰:“ 保障 哉!” 尹 铎 损 其 户数。 简 子 谓 无 恤 曰:“ 晋 国有 难, 而 无以 尹 铎 为 少, 无以 晋 阳 为 远, 必 以为 归。” 
				  
				  及 智 宣 子 卒, 智 襄 子 为政, 与 韩 康 子、 魏 桓 子 宴 于 蓝 台。 智 伯 戏 康 子 而 侮 段 规。 智 国 闻 之, 谏 曰:“ 主 不备 难, 难 必 至 矣!” 智 伯 曰:“ 难 将由 我。 我不 为难, 谁敢 兴 之!” 对 曰:“ 不然。 夏 书 有之:‘ 一人 三 失, 怨 岂 在 明, 不见 是 图。’ 夫 君子 能 勤 小 物, 故 无 大患。 今 主 一 宴 而 耻 人之 君 相, 又 弗 备, 曰‘ 不敢 兴 难’, 无 乃 不可 乎! 蚋、 蚁、 蜂、 虿, 皆 能 害人, 况 君 相 乎!” 弗 听。 
				  
				  智 伯 请 地 于 韩 康 子, 康 子 欲 弗 与。 段 规 曰:“ 智 伯 好 利 而 愎, 不 与, 将 伐 我; 不如 与之。 彼 狃 于 得 地, 必 请于 他人; 他人 不 与, 必 响 之以 兵, 然后 我得 免于 患 而 待 事 之 变 矣。” 康 子 曰:“ 善。” 使 使者 致 万家 之 邑 于 智 伯。 智 伯 悦。 又 求 地 于 魏 桓 子, 桓 子 欲 弗 与。 任 章 曰:“ 何故 弗 与?” 桓 子 曰:“ 无故 索 地, 故 弗 与。” 任 章 曰:“ 无故 索 地, 诸 大夫 必 惧; 吾 与之 地, 智 伯 必 骄。 彼 骄 而 轻敌, 此 惧 而 相亲; 以 相亲 之兵 待 轻敌 之人, 智 氏 之 命 必 不长 矣。 周 书 曰:‘ 将 欲 败 之, 必 姑 辅 之。 将 欲取 之, 必 姑 与之。’ 主 不如 与之, 以 骄 智 伯, 然后 可以 择交 而 图 智 氏 矣, 柰 何 独 以 吾 为 智 氏 质 乎!” 桓 子 曰:“ 善。” 复 与之 万家 之 邑 一。 
				  
				  智 伯 又 求 蔡、 皋 狼 之地 于 赵 襄 子, 襄 子 弗 与。 智 伯 怒, 帅 韩、 魏 之 甲 以 攻 赵 氏。 襄 子 将 出, 曰:“ 吾 何 走 乎?” 从者 曰:“ 长子 近, 且 城 厚 完。” 襄 子 曰:“ 民 罢 力 以 完 之, 又 毙 死 以 守 之, 其 谁 与我!” 从者 曰:“ 邯郸 之 仓库 实。” 襄 子 曰:“ 浚 民 之 膏 泽 以 实 之, 又 因而 杀 之, 其 谁 与我! 其 晋 阳 乎, 先主 之 所属 也, 尹 铎 之所 宽 也, 民 必 和 矣。” 乃 走 晋 阳。 
				  
				  三家 以 国人 围 而 灌 之, 城 不 浸 者 三 版; 沈 灶 产 蛙, 民 无 叛意。 智 伯 行 水, 魏 桓 子 御, 韩 康 子 骖 乘。 智 伯 曰:“ 吾 乃 今 知 水 可以 亡人 国 也。” 桓 子 肘 康 子, 康 子 履 桓 子 之 跗, 以 汾水 可以 灌 安 邑, 绛 水 可以 灌 平 阳 也。 絺 疵 谓 智 伯 曰:“ 韩、 魏 必 反 矣。” 智 伯 曰:“ 子 何以 知 之?” 絺 疵 曰:“ 以 人事 知 之。 夫 从 韩、 魏 之兵 以 攻 赵, 赵 亡, 难 必 及 韩、 魏 矣。 今 约 胜 赵 而 三分 其 地, 城 不 没 者 三 版, 人马 相 食, 城 降 有 日, 而 二 子 无 喜 志, 有 忧色, 是非 反而 何?” 明日, 智 伯 以 絺 疵 之言 告 二 子, 二 子 曰:“ 此 夫 谗 人欲 为 赵 氏 游说, 使 主 疑 于 二 家 而 懈 于 攻 赵 氏 也。 不然, 夫 二 家 岂不 利 朝夕 分 赵 氏 之 田, 而 欲 为 危难 不可 成 之事 乎!” 二 子 出, 絺 疵 入 曰:“ 主 何以 臣 之言 告 二 子 也?” 智 伯 曰:“ 子 何以 知 之?” 对 曰:“ 臣 见 其 视 臣 端 而 趋 疾, 知 臣 得 其情 故 也。” 智 伯 不 悛。 絺 疵 请 使于 齐。
				  
				  赵 襄 子 使 张 孟 谈 潜 出 见 二 子, 曰:“ 臣 闻 唇 亡 则 齿 寒。 今 智 伯 帅 韩、 魏 以 攻 赵, 赵 亡 则 韩、 魏 为之 次 矣。” 二 子 曰:“ 我 心知 其然 也; 恐 事 未遂 而 谋 泄, 则 祸 立 至 矣。” 张 孟 谈 曰:“ 谋 出 二 主 之口, 入 臣 之 耳, 何 伤 也!” 二 子 乃 潜 与 张 孟 谈 约, 为之 期 日 而 遣 之。 襄 子夜 使人 杀 守 堤 之 吏, 而 决 水 灌 智 伯 军。 智 伯 军 救 水 而 乱, 韩、 魏 翼 而 击 之, 襄 子 将 卒 犯 其 前, 大败 智 伯 之众, 遂 杀 智 伯, 尽 灭 智 氏 之 族。 唯 辅 果 在。 
				  
				  \textbf{臣光曰}: \textit{智 伯 之 亡 也, 才 胜 德 也。 夫 才 与 德 异, 而 世俗 莫 之 能辨, 通 谓之 贤, 此 其 所以 失 人也。 夫 聪 察 强 毅 之 谓 才, 正直 中和 之 谓 德。 才者, 德 之 资 也; 德 者, 才 之 帅 也。 云梦 之 竹, 天下 之 劲 也; 然而 不 矫 揉, 不 羽 括, 则 不 能以 入 坚。 棠 溪 之 金, 天下 之 利 也; 然而 不 熔 范, 不 砥砺, 则 不 能以 击 强。 是故 才德 全 尽 谓之“ 圣人”, 才德 兼 亡 谓之“ 愚人”; 德 胜 才 谓之“ 君子”, 才 胜 德 谓之“ 小人”。 凡 取 人 之术, 苟 不得 圣人、 君子 而 与之, 与其 得 小人, 不 若 得 愚人。 何 则? 君子 挟 才 以 为善, 小人 挟 才 以为 恶。 挟 才 以为 善者, 善 无 不至 矣; 挟 才 以为 恶者, 恶 亦无 不至 矣。 愚者 虽 欲 为 不善, 智 不能 周, 力 不能 胜, 譬如 乳 狗 搏 人, 人 得而 制 之。 小人 智 足以 遂其 奸, 勇 足以 决 其 暴, 是 虎 而 翼 者 也, 其为 害 岂 不多 哉! 夫 德 者 人 之所 严, 而 才者 人之 所爱; 爱 者 易 亲, 严 者 易 疏, 是以 察 者 多 蔽 于 才 而 遗 于 德。 自古 昔 以来, 国 之 乱臣, 家 之 败 子, 才有 馀 而 德 不足, 以至于 颠覆者 多 矣, 岂 特 智 伯 哉! 故 为国 为 家 者 苟 能 审 于 才德 之分 而知 所 先后, 又 何 失 人之 足 患 哉!}
				  
		\subsection{魏文侯}
			\subparagraph{总结}
				\begin{itemize}[itemindent = 1em]
					\item 礼贤下士
					\item 诚信
					\item 担当,自我牺牲的担当
					\item 提纲挈领
				\end{itemize} 
			
			\subparagraph{故事}
				魏 文 侯 以 \textbf{卜子夏}(孔子 文学方面的顶尖学生、72贤、10哲)、 \textbf{田子方} 为师。 每过 \textbf{段干木} 之庐 必式。 四方 贤士 多 归 之。
				
				文 侯 与 群臣 饮酒, 乐, 而 天 雨, 命 驾 将 适 野。 左右 曰:“ 今日 饮酒 乐, 天 又 雨, 君 将 安之?” 文 侯 曰:“ 吾 与 虞 人 期 猎, 虽 乐, 岂可 无一 会期 哉!” 乃 往, 身 自 罢 之。
				
				韩 借 师 于 魏 以 伐 赵, 文 侯 曰:“ 寡 人与 赵, 兄弟 也, 不敢 闻 命。” 赵 借 师 于 魏 以 伐 韩, 文 侯 应 之 亦然。 二 国 皆 怒 而去。 已 而知 文 侯 以 讲 于 己 也, 皆 朝 于 魏。 魏 于是 始 大于 三 晋, 诸侯 莫能 与之 争。
				
				使 乐 羊 伐 中山, 克 之; 以 封 其 子 击。 文 侯 问 于 群臣 曰:“ 我 何如 主?” 皆 曰:“ 仁 君。” 任 座 曰:“ 君 得 中山, 不以 封 君 之 弟 而以 封 君 之子, 何谓 仁 君!” 文 侯 怒, 任 座 趋 出。 次 问 翟 璜, 对 曰:“ 仁 君。” 文 侯 曰:“ 何以 知 之?” 对 曰:“ 臣 闻 君 仁 则 臣 直。 响 者 任 座 之言 直, 臣 是以 知 之。” 文 侯 悦, 使 翟 璜 召 任 座 而 反之, 亲 下堂 迎 之, 以 为上 客。
				
				文 侯 与 田 子 方 饮, 文 侯 曰:“ 锺 声 不比 乎? 左 高。” 田 子 方 笑。 文 侯 曰:“ 何 笑?” 子 方 曰:“ 臣 闻 之, 君 明 乐 官, 不明 乐音。 今 君 审 于 音, 臣 恐 其 聋 于 官 也。” 文 侯 曰:“ 善。”

		
		\subsection{魏国选相}
			\subparagraph{总结}居 视 其所 亲, 富 视 其所 与, 达 视 其所 举, 穷 视其 所不 为, 贫 视其 所不 取
				
				深远,厚重
			
			
			\subparagraph{故事}
				\textbf{文 侯} 谓 李 克 曰:“ 先生 尝 有言 曰:‘ 家 贫 思 良 妻; 国 乱 思 良 相。’ 今 所 置 非 成 则 璜, 二 子 何如?” 对 曰:“ 卑 不 谋 尊, 疏 不 谋 戚。 臣 在 阙 门 之外, 不敢当 命。” 文 侯 曰:“ 先生 临 事 勿 让!” 克 曰:“ 君 弗 察 故 也。 \textbf{居 视 其所 亲, 富 视 其所 与, 达 视 其所 举, 穷 视其 所不 为, 贫 视其 所不 取, 五 者 足以 定 之 矣}, 何 待 克 哉!” 文 侯 曰:“ 先 生就 舍, 吾 之 相 定 矣。” 李 克 出, 见 翟 璜。 翟 璜 曰:“ 今 者 闻 君 召 先 生而 卜 相, 果 谁 为之?” 克 曰:“ 魏 成。” 翟 璜 忿然作色 曰:“ 西 河 守 吴 起, 臣 所 进 也。 君 内 以 邺 为 忧, 臣 进 西门 豹。 君 欲 伐 中山, 臣 进 乐 羊。 中山 已 拔, 无 使 守 之, 臣 进 先生。 君 之子 无 傅, 臣 进 屈 侯 鲋。 以 耳目 之所 睹 记, 臣 何 负于 魏 成!” 李 克 曰:“ 子 言 克 于 子 之 君 者, 岂 将比 周 以求 大官 哉? 君 问 相 于 克, 克 之 对 如是。 所以 知 君 之 必 相 魏 成 者, 魏 成 食禄 千 钟, 什 九 在外, 什 一 在内; 是 以东 得 卜 子 夏、 田 子 方、 段 干 木。 此 三 人者, 君 皆 师 之; 子 所 进 五人 者, 君 皆 臣 之。 子 恶 得 与 魏 成 比 也!” 翟 璜 逡 巡 再拜 曰:“ 璜, 鄙人 也, 失 对, 愿 卒 为 弟子!”


				\textbf{魏(武侯)} 置 相, 相 田 文。 吴 起 不悦, 谓 田 文 曰:“ 请与 子 论 功 可 乎?” 田 文 曰:“ 可。” 起 曰:“ 将 三军, 使 士卒 乐死, 敌国 不敢 谋, 子 孰 与 起?” 文 曰:“ 不如 子。” 起 曰:“ 治 百官, 亲 万民, 实 府库, 子 孰 与 起?” 文 曰:“ 不如 子。” 起 曰:“ 守 西 河, 秦 兵 不敢 东乡, 韩、 赵 宾 从, 子 孰 与 起?” 文 曰:“ 不如 子。” 起 曰:“ 此 三 者 子 皆 出 吾 下, 而 位居 吾 上, 何 也?” 文 曰:“ \textbf{主 少 国 疑, 大臣 未附, 百姓 不信, 方 是 之时, 属 之子 乎, 属之 我 乎?}” 起 默然 良久 曰:“ 属 之子 矣!” 
		
		\subsection{豫让刺杀赵襄子}
			\subparagraph{总结}忠心的人值得人尊敬,懂得赏识别人才能的人定能得到其感恩之心!
			
			\subparagraph{故事}
				三家 分 智 氏 之 田。 赵 襄 子 漆 智 伯 之 头, 以为 饮 器。 智 伯 之 臣 豫 让 欲 为之 报仇, 乃 诈 为 刑 人, 挟 匕首, 入 襄 子宫 中 涂 厕。 襄 子 如厕 心动, 索 之, 获 豫 让。 左右 欲 杀 之, 襄 子 曰:“ 智 伯 死 无后, 而 此 人欲 为 报仇, 真 义士 也, 吾 谨 避 之 耳。” 乃 舍 之。 豫 让 又 漆 身为 癞, 吞 炭 为 哑。 行乞 于 市, 其 妻 不识 也。 行 见 其 友, 其 友 识 之, 为之 泣 曰:“ 以 子 之才, 臣 事 赵 孟, 必得 近 幸。 子 乃 为所欲为, 顾 不易 邪? 何 乃 自 苦 如此? 求 以 报仇, 不 亦 难 乎!” 豫 让 曰:“ 既已 委 质 为 臣, 而又 求 杀 之, 是 二心 也。 凡 吾 所为 者, 极难 耳。 然 所以 为此 者, 将以 愧 天下 后世 之为 人臣 怀 二心 者 也。” 襄 子 出, 豫 让 伏 于 桥下。 襄 子 至 
				
		\subsection{聂政}
			\subparagraph{总结}用人诚!为人孝!堪入史册。
			
			\subparagraph{故事}
				三月, 盗 杀 韩 相 \textbf{侠累}。 侠 累 与 濮 阳 严 仲 子 有 恶。 仲 子 闻 轵 人 聂 政 之 勇, 以 黄金 百 溢 为政 母 寿, 欲 因 以 报仇。 政 不受, 曰:“ 老母 在, 政 身 未敢 以 许 人也!” 及 母 卒, 仲 子 乃 使 政 刺 侠 累。 侠 累 方 坐 府上, 兵 卫 甚 众, 聂 政 直入 上 阶, 刺杀 侠 累, 因 自 皮面 决 眼, 自 屠 出 肠。 韩 人 暴 其 尸 于 市, 购 问, 莫能 识。 其 姊 荌 闻 而 往, 哭 之 曰:“ 是 轵 深井 里 聂 政 也! 以 妾 尚在 之故, 重 自 刑 以 绝 从。 妾 柰 何 畏 殁 身 之 诛, 终 灭 贤弟 之名!” 遂 死于 政 尸 之旁。
			
						
		\subsection{名将吴起}
			\subparagraph{总结}随有才,但无德、性格刚烈自负,导致最后多次易主,没有稳定发挥才能的机会。
			
			\subparagraph{故事}
				吴 起 者, 卫 人, 仕 于 鲁。 齐 人 伐 鲁, 鲁人 欲 以为 将, 起 取齐 女 为妻, 鲁人 疑 之, 起 杀妻 以求 将, 大破 齐 师。 或 谮 之 鲁 侯 曰:“ 起始 事 曾 参, 母 死不 奔丧, 曾 参 绝 之; 今 又 杀妻 以求 为 君 将。 起, 残忍 薄 行人 也! 且 以 鲁国 区区 而有 胜 敌 之名, 则 诸侯 图 鲁 矣。” 起 恐 得罪, 闻 魏 文 侯 贤,乃 往 归 之。 文 侯 问 诸 李 克, 李 克 曰:“ 起 贪 而 好色; 然 用兵, 司马 穰 苴 弗 能 过 也。” 于是 文 侯 以为 将, 击 秦, 拔 五 城。 起 之为 将, 与 士卒 最下 者 同 衣食, 卧 不设 席, 行不 骑乘, 亲 裹 赢 粮, 与 士卒 分 劳苦。 卒 有病 疽 者, 起 为 吮 之。 卒 母 闻 而 哭 之。 人 曰:“ 子, 卒 也, 而 将军 自 吮 其 疽, 何 哭 为?” 母 曰:“ 非 然也。 往年 吴 公 吮 其父 疽, 其父 战不旋踵, 遂 死于 敌。 吴 公 今 又 吮 其 子, 妾 不知 其 死 所 矣, 是以 哭 之。”
	
				武 侯 浮 西 河 而下, 中流 顾 谓 吴 起 曰:“ 美 哉 山河 之 固, 此 魏国 之宝 也!” 对 曰:“ 在 德 不在 险。 昔 三 苗 氏, 左 洞 庭, 右 彭 蠡; 德 义 不修, 禹 灭 之。 夏 桀 之 居, 左 河 济, 右 泰 华, 伊 阙 在 其 南, 羊肠 在 其 北; 修 政 不仁, 汤 放 之。 商 纣 之国, 左 孟 门, 右 太行, 常 山 在 其 北, 大河 经 其 南; 修 政 不 德, 武 王 杀 之。 由此 观 之, 在 德 不在 险。 若 君 不修 德, 舟 中 之人 皆 敌国 也!” 武 侯 曰:“ 善。” 魏 置 相, 相 田 文。 吴 起 不悦, 谓 田 文 曰:“ 请与 子 论 功 可 乎?” 田 文 曰:“ 可。” 起 曰:“ 将 三军, 使 士卒 乐死, 敌国 不敢 谋, 子 孰 与 起?” 文 曰:“ 不如 子。” 起 曰:“ 治 百官, 亲 万民, 实 府库, 子 孰 与 起?” 文 曰:“ 不如 子。” 起 曰:“ 守 西 河, 秦 兵 不敢 东乡, 韩、 赵 宾 从, 子 孰 与 起?” 文 曰:“ 不如 子。” 起 曰:“ 此 三 者 子 皆 出 吾 下, 而 位居 吾 上, 何 也?” 文 曰:“ 主 少 国 疑, 大臣 未附, 百姓 不信, 方 是 之时, 属 之子 乎, 属之 我 乎?” 起 默然 良久 曰:“ 属 之子 矣!” 久之, 魏 相公 叔 尚 主 而 害 吴 起。 公 叔 之 仆 曰:“ 起 易 去 也。 起 为人 刚劲 自喜。 子 先 言 于 君 曰:‘ 吴 起, 贤人 也, 而 君 之国 小, 臣 恐 起 之 无 留心也。 君 盍 试 延 以 女, 起 无 留心, 则 必 辞 矣。’ 子 因 与 起 归 而使 公主 辱 子, 起见 公主 之 贱 子 也, 必 辞, 则 子 之计 中 矣。” 公 叔 从 之, 吴 起 果 辞 公主。 魏 武 侯 疑 之 而未 信, 起 惧 诛, 遂 奔 楚。 楚 悼 王 素闻 其 贤, 至 则 任 之为 相。 起 明法审令, 捐 不急 之 官, 废 公族 疏远 者, 以 抚养 战斗 之士, 要在 强 兵, 破 游说 之言 从 横 者。 于是 南 平 百 越, 北 却 三 晋, 西 伐 秦, 诸侯 皆 患 楚 之 强; 而 楚 之 贵戚 大臣 多 怨 吴 起 者。
				
				楚 悼 王 薨。 贵戚 大臣 作乱, 攻 吴 起; 起 走 之王 尸 而 伏 之。 击 起 之徒 因 射 刺 起, 并 中 王 尸。 既 葬, 肃 王 即位, 使 令 尹 尽 诛 为 乱 者; 坐起 夷 宗 者 七十 馀 家。

		\subsection{商鞅}
			\subparagraph{总结}守信的力量远远超乎你的想象。
			
			\subparagraph{故事}
				孝 公 下令 国中 曰:“ 昔 我 穆 公, 自 岐、 雍 之间 修 德行 武, 东 平 晋 乱, 以 河 为界, 西 霸 戎 翟, 广地 千里, 天子 致 伯, 诸侯 毕 贺, 为 后世 开业 甚 光 美。 会 往 者 厉、 躁、 简 公、 出 子 之 不宁, 国家 内忧, 未 遑 外事。 三 晋 攻 夺 我 先君 河西 地, 丑 莫大 焉。 献 公 即位, 镇 抚 边境, 徙 治 栎 阳, 且 欲 东 伐, 复 穆 公 之 故地, 修 穆 公 之 政令。 寡人 思念 先君 之意, 常 痛 于心。 宾客 群臣 有能 出奇 计 强 秦 者, 吾 且 尊 官, 与 之分 土。” 于是 卫 公孙 鞅 闻 是 令 下, 乃 西 入 秦。 公孙 鞅 者, 卫 之 庶 孙 也, 好 刑名 之 学。 事 魏 相公 叔 痤, 痤 知 其 贤, 未及 进。 会 病, 魏 惠 王 往 问 之 曰:“ 公 叔 病 如有 不可 讳, 将 柰 社稷 何?” 公 叔 曰:“ 痤 之中 庶子 卫 鞅, 年 虽 少, 有 奇才, 愿 君 举国 而 听之!” 王 嘿 然。 公 叔 曰:“ 君 即 不听 用 鞅, 必 杀 之, 无 令 出境!” 王 许诺 而去。
				
				公 叔 召 鞅 谢 曰:“ 吾 先君 而后 臣, 故 先为 君 谋, 后以 告 子。 子 必 速行 矣!” 鞅 曰:“ 君 不 能用 子 之言 任 臣, 又 安 能用 子 之言 杀 臣 乎!” 卒 不去。 王 出, 谓 左右 曰:“ 公 叔 病 甚, 悲 乎, 欲 令 寡人 以 国 听 卫 鞅 也! 既 又 劝 寡人 杀 之, 岂不 悖 哉!” 卫 鞅 既 至 秦, 因 嬖 臣 景 监 以 求见 孝 公, 说 以 富国强兵 之术; 公 大悦, 与 议 国事。
								
				卫 鞅 欲 变法, 秦 人 不悦。 卫 鞅 言 于 秦 孝 公 曰:“ 夫 民 不 可与 虑 始, 而 可与 乐 成。 论 至 德 者 不和 于 俗, 成 大功 者 不 谋 于 众。 是以 圣人 苟 可以 强国, 不法 其 故。” 甘 龙 曰:“ 不然, 缘 法 而 治 者, 吏 习 而 民安 之。” 卫 鞅 曰:“ 常人 安于 故 俗, 学者 溺于 所闻, 以此 两者, 居官守法 可 也, 非 所 与 论 于 法 之外 也。 智者 作法, 愚者 制 焉; 贤者 更 礼, 不肖 者 拘 焉。” 公 曰:“ 善。” 以 卫 鞅 为 左 庶 长。 卒 定 变法 之 令。 令 民 为 什 伍 而 相 收 司、 连坐, 告 奸者 与 斩 敌 首 同 赏, 不 告 奸者 与 降 敌 同 罚。 有 军功 者, 各 以 率 受 上 爵; 为私 斗 者, 各 以 轻重 被 刑 大小。 僇 力 本 业, 耕 织 致 粟 帛 多者, 复 其 身; 事 末 利 及 怠 而 贫 者, 举 以为 收 孥。 宗室 非有 军功 论, 不得 为 属 籍。 明 尊卑 爵 秩 等级, 各 以 差 次 名 田宅、 臣妾、 衣服。 有功 者 显 荣, 无功 者 虽 富 无所 芬 华。 
				
				令 既 具 未 布, 恐 民 之 不信, 乃 立 三丈 之 木 于 国都 市 南门, 募 民 有能 徙 置 北门 者 予 十 金。 民 怪 之, 莫 敢 徙。 复 曰:“ 能 徙 者 予 五十 金!” 有 一人 徙 之, 辄 予 五十 金。 乃 下令。 
				
				令 行期 年, 秦 民 之 国都 言 新 令 之 不便 者 以 千 数。 于是 太子 犯法。 卫 鞅 曰:“ 法 之 不行, 自上 犯 之。” 太子, 君 嗣 也, 不可 施 刑, 刑 其 傅 公子 虔, 黥 其 师 公孙 贾。 明日, 秦 人 皆 趋 令。 行 之 十年, 秦国 道不拾遗, 山 无 盗贼, 民 勇于 公 战, 怯 于 私 斗, 乡 邑 大治。 秦 民初 言 令 不便 者, 有来 言 令 便。 卫 鞅 曰:“ 此 皆 乱 法 之 民 也!” 尽 迁 之 于 边。 其后 民 莫 敢 议 令。 
				
				\textbf{臣 光 曰}: \textit{夫 信者, 人君 之大 宝 也。 国 保 于民, 民 保 于 信; 非 信 无 以使 民, 非 民 无以 守 国。 是故 古 之王 者 不 欺 四海, 霸 者 不 欺 四邻, 善 为国 者 不 欺 其 民, 善 为 家 者 不 欺 其 亲。 不善 者 反之, 欺 其 邻国, 欺 其 百姓, 甚 者 欺 其 兄弟, 欺 其父 子。 上 不信 下, 下 不 信上, 上下 离心, 以至于 败。 所 利 不能 药 其 所伤, 所获 不能 补 其所 亡, 岂不 哀哉! 昔 齐 桓 公 不 背 曹 沫 之 盟, 晋 文 公 不 贪 伐 原 之 利, 魏 文 侯 不弃 虞 人之 期, 秦 孝 公 不 废 徙 木 之 赏。 此 四 君 者 道 非 粹 白, 而 商 君 尤 称 刻薄, 又 处 战 攻 之 世, 天下 趋于 诈 力, 犹 且 不敢 忘 信 以 畜 其 民, 况 为 四海 治 平 之 政 者 哉!}
 
		\subsection{孙膑VS庞涓}
			\subparagraph{总结}做事动脑!如果想要阻止两个人打架,并不需要直接去阻拦他们,我们应该做的是让他们有所顾忌。一般人肯定会直接冲过去阻拦,而孙膑则不同。  分析、\textbf{洞察全局}
			
			\subparagraph{故事}
				  魏 惠 王 伐 赵, 围 邯郸。 楚王 使 景 舍 救 赵。 
				  
				  齐 威 王 使 田 忌 救 赵。 
				  
				  初, 孙膑 与 庞 涓 俱 学 兵法, 庞 涓 仕 魏 为 将军, 自 以 能不 及 孙膑, 乃 召 之; 至, 则以 法 断 其 两足 而 黥 之, 欲 使 终身 废弃。 齐 使者 至 魏, 孙膑 以 刑 徒 阴 见, 说 齐 使者; 齐 使者 窃 载 与之 齐。 田 忌 善 而 客 待 之, 进 于 威 王。 威 王 问 兵法, 遂 以 为师。 于是 威 王 谋 救 赵, 以 孙膑 为 将; 辞 以 刑 馀 之人 不可, 乃 以 田 忌 为 将 而 孙子 为师, 居 辎 车 中, 坐 为 计谋。 
				  
				  田 忌 欲 引 兵 之 赵。 孙子 曰:“ \textbf{夫 解 杂 乱纷 纠 者 不 控 拳, 救 斗 者 不 搏 撠, 批亢捣虚, 形格势禁, 则 自为 解 耳}。 今 梁、 赵 相 攻, 轻兵 锐 卒 必 竭 于外, 老弱 疲于 内; 子 不 若 引 兵 疾走 魏 都, 据 其 街 路, 冲 其 方 虚, 彼 必 释 赵 以 自救: 是我 一举 解 赵 之 围 而 收 弊 于 魏 也。” 田 忌 从 之。 十月, 邯郸 降 魏。 魏 师 还, 与 齐 战 于 桂 陵, 魏 师 大败。
				  
		\subsection{楚昭王}
			\subparagraph{总结}好坏话都得听,单听一面之词容易迷失自己。
			
			\subparagraph{故事}		  
				  楚 昭 奚 恤 为 相。 江 乙 言 于 楚王 曰:“ 人有 爱 其 狗 者, 狗 尝 溺 井, 其 邻人 见, 欲 入 言之, 狗 当 门 而 噬 之。 今 昭 奚 恤 常 恶 臣 之见, 亦 犹 是 也。 且 人 有好 扬 人之 善者, 王 曰:‘ 此 君子 也,’ 近 之; 好 扬 人之 恶者, 王 曰:‘ 此 小人 也,’ 远 之。 然则 且有 子 弑 其父、 臣 弑 其 主 者, 而 王 终 己 不知 也。 何者? 以 王 好闻 人 之美 而 恶 闻 人之 恶 也。” 王 曰:“ 善, 寡人 愿 两 闻 之。”
				  
		\subsection{燕昭王求贤}
			\subparagraph{总结} 求才心切,重金获取死千里马头,活千里马自上门也。
			
			\subparagraph{故事} 郭隗以死千里马头自喻,楚昭王仿例招贤。
		
		\subsection{赵武陵王胡服骑射}
			\subparagraph{总结}学习优秀的技术本领
			
			\subparagraph{故事}
				学胡人骑射、服饰

		\subsection{乐毅伐齐}
			\subparagraph{总结}眼界、视野很重要,考虑问题全面、考虑问题深度足。
			
			\subparagraph{故事}
				燕伐齐。
				
				乐毅谋划5国合攻,并在见胜负已定之时,让秦国和韩国的人回去,说以后会重重答谢他们,因为毕竟与他们没仇恨。而对于魏国和让其去攻打原宋国的地盘,让赵国去收复原自己失去的河间地区,魏赵两国见有大便宜可捡,便都同意了。
				
				而对于燕国,乐毅决定继续追击齐国的败军并直入齐国深地,但是他旁边的谋士剧信觉得不妥,并说"我们燕国与齐国比起来,势力要小得多啊!今天我们之所以能打得过齐国是因为有其他诸侯国的帮助,而单凭我们再想打败齐国就难了。"
				
				乐毅说:“你虽然聪明,但是目光未免太短浅了,齐国公自等位以来,对百姓、对诸侯都不好,而且对忠臣不重要而信小人,你想这样的君主能受百姓爱戴吗?如今齐国吃了败仗,只要我们乘胜追击,齐国必定大乱。到时候趁着齐国大乱,便可以轻轻松松把齐国拿下,这难道不比几座城池有价值吗? 如果我们没有抓住机会,齐王回去痛改前非,任用贤才,体恤民情,到时候我们还有机会么?那这几座城池还能保得住吗?”
				
				齐国乱,燕伐之!齐向楚国求救,楚国与燕同分齐地。
		
		\subsection{将相和}
			\subparagraph{总结}看待问题不只是看到表面,需要考虑问题的全面性,问题的深度!
			
			\subparagraph{故事}
				蔺相如:逼秦王击乐,为赵王挽回面子
				
				回赵国后,蔺相如被谨慎为最高官-上卿,廉颇知道后,非常生气,并说“他凭什么比我官大,我廉颇是赵国的大将军,每次攻城打仗都靠我,以前蔺相如只是个普通百姓,凭借花言巧语就在我之上,这是我最大的耻辱!”
				
				蔺相如听到后,天天避着与廉颇相见,别人问及对曰:“你们说是我怕廉颇吗?你们认为我是因为懦弱才不敢和廉颇相遇吗?我问你们,廉颇老将军与秦军比起来,哪个更加厉害,更加有威严?”“你们想想,连秦王那么有威严的人我都不怕,难道我会怕廉颇?我虽然没本事,但是还不至于这么胆小吧!我那么做是因为我考虑到秦国现在自所以不敢大举进攻我们赵国,就是因为有我和廉颇老将军在!如果我们不和,那么一定会削弱赵国的力量,使秦国有机可乘。正所谓‘两虎相争,必有一伤’,我那么做是为了赵国考虑!”
				
				廉颇知道后,背着荆条请罪。
		
		
		\subsection{范雎(ju)睚眦必报}
			\subparagraph{原文}初, 魏 人 范 睢 从中 大夫 须 贾 使于 齐, 齐 襄 王 闻 其 辩 口, 私 赐 之 金 及 牛、 酒。 须 贾 以为 睢 以 国 阴 事 告 齐 也, 归 而 告 其 相 魏 齐。 魏 齐 怒, 笞 击 范 睢, 折 胁, 折 齿。 睢 佯死, 卷 以 箦, 置 厕 中, 使 客 醉 者 更 溺 之, 以 惩 后, 令 无 妄言 者。 范 睢 谓 守 者 曰:“ 能 出 我, 我 必有 厚 谢。” 守 都 请 弃 箦 中 死人。 魏 齐 醉, 曰:“ 可 矣。” 范 睢 得出。 魏 齐 悔, 复 召 求之。 魏 人 郑 安平 遂 操 范 睢 亡 匿, 更 姓名 曰 张 禄。 
			
			秦 谒者 王 稽 使于 魏, 范 睢 夜 见 王 稽。 稽 潜 载 与 俱 归, 荐 之 于 王, 王 见之于 离 宫。 范 睢 佯为 不知 永 巷 而 入 其中, 王 来 而 宦 者 怒 逐 之, 曰:“ 王 至!” 范 睢 谬 曰:“ 秦 安 得 王, 秦 独有 太后、 穰 侯 耳!” 王 微 闻 其言, 乃 屏 左右, 跽 而 请 曰:“ 先生 何以 幸 教 寡人?” 对 曰:“曰:“ 唯唯。” 如是 者 三。 王 曰:“ 先生 卒 不幸 教 寡人 邪?” 范 睢 曰:“ 非 敢 然也! 臣, 羁 旅 之 臣 也, 交 疏于 王, 而 所愿 陈 者 皆 匡 君 之事, 处 人 骨肉 之间, 愿 效 愚 忠 而 未知 王 之心 也, 此 所以 王 三 问 而 不敢 对 者 也。 臣 知 今日 言之 于 前, 明日 伏诛 于 后, 然 臣 不敢 避 也。 且 死者, 人之 所 必 不免 也, 苟 可以 少有 补 于 秦 而 死, 此 臣 之所 大 愿 也。 独 恐 臣 死 之后, 天下 杜口裹足, 莫 肯 乡 秦 耳。” 王 跽 曰:“ 先生, 是 何 言 也! 今 者 寡人 得见 先生, 是 天 以 寡人 溷 先 生而 存 先王 之 宗庙 也。 事 无 大小, 上 及 太后, 下 至 大臣, 愿 先生 悉 以 教 寡人, 无疑 寡人 也!” 范 睢 拜, 王 亦 拜。 范 睢 曰:“ 以 秦国 之大, 士卒 之 勇, 以 治 诸侯, 譬 若 走 韩 卢 而 博 蹇 兔 也, 而 闭关 十五年, 不敢 窥 兵 于 山东 者, 是 穰 侯 为 秦 谋 不忠, 而 大王 之计 亦有 所失 也。” 王 跽 曰:“ 寡人 愿闻 失计!” 然 左右 多 窃听 者, 范 睢 未 敢言 内, 先 言外 事, 以 观 王 之 俯仰。 因 进 曰:“ 夫 穰 侯 越 韩、 魏 而 攻 齐 刚、 寿, 非 计 也。 齐 湣 王 南 攻 楚, 破 军 杀 将, 再 辟 地 千里, 而 齐 尺寸 之地 无 得 焉 者, 岂不 欲得 地 哉? 形势 不 能有 也。 诸侯 见 齐 之 罢 敝, 起兵 而 伐 齐, 大破 之, 齐 几 于 亡, 以其 伐 楚 而 肥 韩、 魏 也。 今 王 不如 远 交 而近 攻, 得 寸 则 王 之 寸 也, 得 尺 亦 王 之 尺 也。 今 夫 韩、 魏, 中国 之处 而 天下 之 枢 也。 王 若 用 霸, 必 亲 中国 以为 天下 枢, 以 威 楚、 赵, 楚 强 则 附 赵, 赵 强 则 附 楚, 楚、 赵 皆 附, 齐 必 惧 矣, 齐 附则 韩、 魏 因 可 虏 也。” 王 曰:“ 善。” 乃 以 范 睢 为 客卿, 与 谋 兵 事。 
			
			秦 中 更 胡 伤 攻 赵 阏 与, 不 拔。 
			 
		    秦王 用 范 睢 之 谋, 使 五 大夫 绾 伐 魏, 拔 怀。 
		    
		    秦 悼 太子 质 于 魏 而 卒。 
		    
		    秦 拔 魏 邢 丘。 范 睢 日益 亲, 用事, 因 承 间 说 王 曰:“ 臣 居 山东 时, 闻 齐 之 有 孟 尝 君, 不 闻 有 王; 闻 秦 有 太后、 穰 侯, 不 闻 有 王。 夫 擅 国 之 谓 王, 能 利害 之 谓 王, 制 杀生 之 谓 王。 今 太后 擅 行 不顾, 穰 侯 出使 不报, 华 阳、 泾 阳 击 断 无讳, 高 陵 进退 不请, 四 贵 备 而 国 不 危 者, 未之 有 也。 为此 四 贵 者 下, 乃 所谓 无 王 也。 穰 侯 使者 操 王 之 重, 决 制 于 诸侯, 剖 符 于 天下, 征 敌 伐 国, 莫 敢 不听; 战胜 攻取 则 利 归于 陶, 战败 则 结怨 于 百姓 而 祸 归于 社稷。 臣 又 闻 之, 木 实 繁 者 披 其 枝, 披 其 枝 者 伤 其 心; 大 其 都 者 危 其 国, 尊 其 臣 者 卑 其 主。 淖 齿 管 齐, 射 王 股, 擢 王 筋, 悬 之 于 庙 梁, 宿 昔 而 死。 李 兑 管 赵, 囚 主 父 于 沙丘, 百日 而 饿死。 今 臣 观 四 贵 之 用事, 此 亦 淖 齿、 李 兑 之类 也。 夫 三代 之所以 亡国 者, 君 专 授 政 于 臣, 纵酒 弋 猎; 其所 授 者 妒贤疾能, 御下蔽上 以 成 其 私, 不 为主 计, 而 主 不 觉悟, 故 失 其 国。 今 自有 秩 以 上至 诸 大 吏, 下 及 王 左右, 无非 相国 之 人者, 见 王 独 立于 朝, 臣 窃 为王 恐, 万世 之后 有 秦国 者, 非 王 子孙 也!” 王 以为 然, 于是 废 太后, 逐 穰 侯、 高 陵、 华 阳、 泾 阳 君 于 关外, 以 范 睢 为 丞相, 封为 应 侯。
			
			魏 王 使 须 贾 聘 于 秦, 应 侯 敝 衣 间 步 而 往 见之。 须 贾 惊 曰:“ 范 叔 固 无恙 乎!” 留 坐 饮食, 取 一 绨 袍 赠 之。 遂为 须 贾 御 而至 相 府, 曰:“ 我为 君 先入 通 于 相 君。” 须 贾 怪 其 久 不出, 问 于 门下, 门下 曰:“ 无 范 叔; 乡 者 吾 相 张 君 也。” 须 贾 知 见 欺, 乃 膝行 入 谢罪。 应 侯 坐, 责 让 之, 且 曰:“ 尔 所以 得不 死者, 以 绨 袍 恋恋 尚有 故人 之意 耳!” 乃 大 供 具, 请 诸侯 宾客; 坐 须 贾 于 堂 下, 置 莝、 豆 于 前 而 马 食 之, 使 归 告 魏 王 曰:“ 速 斩 魏 齐 头 来! 不然, 且 屠 大梁!” 须 贾 还, 以 告 魏 齐。 魏 齐 奔 赵, 匿 于 平原 君 家。

			
	\section{楚汉}
		\subsection{刘邦用人}
			\subparagraph{总结}
				\begin{itemize}
					\item 陈平
					\item 疑人不用,用人不疑(执狐疑之心者,来谗贼之口)
				\end{itemize}
			\subparagraph{故事}
				
		\subsection{项羽失败的启示}
			\subparagraph{总结}
				\begin{itemize}
					\item 范增死心塌地多年辅佐,却顶不住别的使者几句话,明显是挑拨,我在你心目中就这个地位?我在你心目中的信誉就值这么一点点?这使得范增对项羽很失望、很生气,而这就是范增觉得项羽不可共事的原因,然后离去。
				\end{itemize}
			\subparagraph{故事}
				陈平使用离间计挑拨项羽与范增的关系,先是问来者的主人,然后说是项羽派来的,然后撤回上上来的好的招待物品,换成不好的,回去使者一五一十的给项羽转达,然后对范增的关键计谋没有听从,范增知道其中缘由后,失望出走,气死在半路。
			
		\subsection{立太子}
			\subparagraph{故事}
				\begin{itemize}
					\item 赵武灵王  易太子
					\item 刘邦      四大高老等,不再改立太子
				\end{itemize}
			
			\subparagraph{总结}
				大局,私欲。
				克己复礼。
		
		\subsection{汉文帝}
			\subparagraph{总结}
				\begin{itemize}[itemindent = 1em]
					\item 追求事实。
					\item 处事圆滑,考虑长远周全(拖刘濞年迈,周亚夫,面面俱到);小错不纠,以德服人;城府深。
				\end{itemize}
			
			\subparagraph{故事}
				
				七国之乱:刘濞之子、汉文帝之子汉景帝 冲突。刘濞(bi) 准备造反。
				
				汉文帝赐予 几和杖,平复了火苗。
				
				汉景帝则不同,晁错觐见削藩!--> 七个诸侯国反!出卖晁错!
				
				周亚夫三个月平叛7国之乱,但最后还是被汉景帝杀了。
		
		\subsection{周亚夫}
		
		
			
	\section{三国}
		\subsection{魏}
		
		
		\subsection{蜀}
		
		
		\subsection{吴}
		
		
		
	\section{五国}
		\subsection{晋}
		
		\subsection{宋}
		
		\subsection{齐}
		
		\subsection{梁}
		
		\subsection{陈}			
		
	\section{随}
	
	
	\section{唐}
		
		\subsection{李世羁ji}
	
		
\end{document}