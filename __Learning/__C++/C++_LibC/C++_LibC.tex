\documentclass[UTF8,a4paper,12pt]{ctexbook} 

 \usepackage{graphicx}%学习插入图
 \usepackage{verbatim}%学习注释多行
 \usepackage{booktabs}%表格
 \usepackage{geometry}%图片
 \usepackage{amsmath}
 \usepackage{amssymb}
 \usepackage{listings}%代码
 \usepackage{xcolor}  %颜色
 \usepackage{enumitem}%列表格式
 \usepackage{tcolorbox}
 \usepackage{algorithm}  %format of the algorithm
 \usepackage{algorithmic}%format of the algorithm
 \usepackage{multirow}   %multirow for format of table
 \usepackage{tabularx} 	%表格排版格式控制
 \usepackage{array}	%表格排版格式控制
 
 %\CTEXsetup[format+={\flushleft}]{section}
 
 \newcommand{\EQ}[1]{$\textbf{EQ:}#1\ $}
 \newcommand{\ALGORITHM}[1]{$\textbf{Algorithm:}#1\ $}
 \newcommand{\Figure}[1]{$\textbf{Figure }#1\ $}
 \renewcommand{\figurename}{Fig}
 \geometry{left=1.6cm,right=1.8cm,top=2cm,bottom=1.7cm} %设置文章宽度
 \pagestyle{plain} 		  %设置页面布局

  %代码效果定义
  \definecolor{mygreen}{rgb}{0,0.6,0}
  \definecolor{mygray}{rgb}{0.5,0.5,0.5}
  \definecolor{mymauve}{rgb}{0.58,0,0.82}
  \lstset{ %
  	backgroundcolor=\color{white},   % choose the background color
  	basicstyle=\footnotesize\ttfamily,        % size of fonts used for the code
  	columns=fullflexible,
  	breaklines=true,                 % automatic line breaking only at whitespace
  	captionpos=b,                    % sets the caption-position to bottom
  	tabsize=4,
  	commentstyle=\color{mygreen},    % comment style
  	escapeinside={\%*}{*)},          % if you want to add LaTeX within your code
  	keywordstyle=\color{blue},       % keyword style
  	stringstyle=\color{mymauve}\ttfamily,     % string literal style
  	frame=single,
  	rulesepcolor=\color{red!20!green!20!blue!20},
  	% identifierstyle=\color{red},
  	language=c++,
  }

 \author{\kaishu 郑华}
 \title{\heiti C++\_C标准库函数}
 
\begin{document}          %正文排版开始
 	\maketitle
 	
\chapter{string.h}
	\begin{itemize}
		\item \textbf{memcpy}: void* memcpy(void* s1, const void* s2, size\_t n); \\
			从s2 中复制 n个同类型的 元素到s1.
		
		\item \textbf{memset}: void memset()
	\end{itemize}
	
\chapter{ctype.h}
	\begin{itemize}
		\item \textbf{isalnum} :检查ch是否字母或数字
		\item \textbf{isalpha} :检查ch是否字母
		\item \textbf{isdigit} :检查ch是否数字(0~9)
		\item \textbf{islower} :检查ch是否是小写字母(a~z)
		\item \textbf{isspace} :检查ch是否空格、跳格符(制表符)或换行符
		\item \textbf{isupper} :检查ch是否大写字母(A~Z)
		\item \textbf{isxdigit}:检查ch是否一个十六进制数字(即0~9,或A~F,a~f)
		\item \textbf{tolower} :将ch字符转换为小写字母
		\item \textbf{toupper} :将ch字符转换为大写字母
	\end{itemize}
\chapter{math.h}
	\begin{itemize}
		\item \textbf{ceil}:求取不小于该浮点数的最小整数
		\item \textbf{floor}:求取不大于该浮点数的最大整数
		\item \textbf{fabs}: 求取绝对值
		\item \textbf{fmod}: 整除x/y 的余数
		\item \textbf{pow(x,y)}: $x^y$
	\end{itemize}
\chapter{stdlib.h} 

\chapter{assert.h}

\chapter{limits.h}

\chapter{time.h}

\chapter{stddef.h} 

\chapter{error.h}

\chapter{locale.h}

\chapter{float.h}
		    
\end{document} 
 		    