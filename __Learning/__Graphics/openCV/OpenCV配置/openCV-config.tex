
\documentclass[UTF8,a4paper,8pt]{ctexart} 

\usepackage{graphicx}%学习插入图
\usepackage{verbatim}%学习注释多行
\usepackage{booktabs}%表格
\usepackage{geometry}%图片
 \usepackage{amsmath} 
 \usepackage{amssymb}

%设置文章宽度
\geometry{textwidth=18cm}
%设置页面布局
\pagestyle{plain}
\author{郑华
	\small	(西北农林科技大学信息工程学院,杨凌 712100)}
\title{OpenCV2.4.8 + VisualStudio2010 配置心得}

 %正文排版开始
 \begin{document} 
 	\maketitle
 	
 	\begin{abstract}
 		参考文献:
 	http://blog.csdn.net/poem\_qianmo/article/details/19809337
 		
    本文介绍 本人配置OpenCV的过程
	\end{abstract}
	
	
\section{下载配置所需}
	\paragraph{1.VisualStudio2010}
	\paragraph{2.OpenCV2.4.8}
	
\section{安装}
    \paragraph{1.安装位置不影响}
    \paragraph{2.配置Path系统环境}	
    
\section{配置}
    \paragraph{1.Create Project}建立一个VisualStudio  ConsoleApplication Empty Project
    \paragraph{2.配置}
     	\begin{itemize}
     		\item -右键项目
     		\item -Properties
     		\item -VC++ Directories
     		\item -配置include Directories += 安装目录$ \verb|\|$opencv$ \verb|\|$build$ \verb|\|$include; += 安装目录$ \verb|\|$ opencv$ \verb|\|$build$ \verb|\|$include$ \verb|\|$opencv; +=安装目录$ \verb|\|$opencv$ \verb|\|$build$ \verb|\|$include$ \verb|\|$opencv2;
     		\item -配置Liberary Directories += 安装目录$ \verb|\|$opencv$ \verb|\|$build$ \verb|\|$x86$ \verb|\|$vc10$ \verb|\|$lib;
     		\item -Linker选项 -input
     		\item Additional Dependecies 添加如下库:见参考文献
     	\end{itemize}
    
\section{编程测试}注意:路径中的单杠要换成双杠\verb|\\|

\section{配置全局VC++目录}
	
	1. 随便打开一个项目,然后点击菜单中的 视图->其他窗口->属性管理器
	
	2. 打开属性管理器,点击项目前的箭头,展开项目,找到debug或者release下面的Microsoft.Cpp.Win32.user这个属性,
	
	3. 双击会出现一个跟在项目上右键属性一样的窗口,修改里面的“VC++目录”就是修改了全局的,
\end{document} 
		