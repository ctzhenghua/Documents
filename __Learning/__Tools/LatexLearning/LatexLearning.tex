\documentclass[UTF8,a4paper,8pt]{ctexart} 

 \usepackage{graphicx}%学习插入图
 \usepackage{verbatim}%学习注释多行
 \usepackage{booktabs}%表格
 \usepackage{geometry}%图片
 \usepackage{amsmath} 
 \usepackage{amssymb}
 \usepackage{listings}%代码
 \usepackage{xcolor}  %颜色
 \usepackage{enumitem}%列表格式
 \usepackage{hyperref}
 \CTEXsetup[format+={\flushleft}]{section}
 \newtheorem{thm}{定理}

\geometry{left=1.6cm,right=1.8cm,top=2cm,bottom=1.7cm} %设置文章宽度

\pagestyle{plain} 		  %设置页面布局
\author{郑华}
\title{Latex 学习笔记}
 %代码效果定义
 \definecolor{codegreen}{rgb}{0,0.6,0}
 \definecolor{codegray}{rgb}{0.5,0.5,0.5}
 \definecolor{codepurple}{rgb}{0.58,0,0.82}
 \definecolor{backcolour}{rgb}{0.95,0.95,0.92}
 
 \lstdefinestyle{mystyle}{
 	backgroundcolor=\color{backcolour},   
 	commentstyle=\color{codegreen},
 	keywordstyle=\color{magenta},
 	numberstyle=\tiny\color{codegray},
 	stringstyle=\color{codepurple},
 	basicstyle=\footnotesize,
 	breakatwhitespace=false,         
 	breaklines=true,                 
 	captionpos=b,                    
 	keepspaces=true,                 
 	%numbers=left,                    
 	%numbersep=5pt,                  
 	showspaces=false,                
 	showstringspaces=false,
 	showtabs=false,                  
 	tabsize=2
 }
\lstset{style=mystyle, escapeinside=``}


%learn newcommand
\newcommand \loves[2]{#1 love #2}

\newcommand \love[3][默认第一个参数]{#2 #1 #3}
\begin{document}          %正文排版开始
 	\maketitle
 
\section{排版命令的参数}
	\verb|\排版命令[可选参数]{其他参数}|
  
	  \begin{itemize}
	  	\item 方括号中的参数为可选
	  	\item 花括号中的其他参数是不可省略的参数
	  	\item 命令也可以不带参数
	  \end{itemize}
\section{对齐环境}
	 \verb|\begin{flushleft}//flushrihgt,center...// \排版命令[可选参数]{其他参数}\end{}|
	    \begin{itemize}
	    	\item 默认两侧对齐
	    	\item flushleft
	    	\item flushright
	    	\item center
	    \end{itemize}

\section{字体}
		\begin{itemize}
			\item \kaishu \verb|\kaishu|
			\item \zihao{-3} \verb|\zihao|
		\end{itemize}
		
\section{绘制图片}
	    \begin{itemize}
	    	\item \verb|\usepackage{tikz}|
	    	\item \verb|\begin{tikzpicture}|
	    	\item \verb|\node[box] (标识名) at(0,0) {文字}:话带字方框|
	    	\item \verb|\draw[->] (标识名1) -- node[above]{字} (标识名2);从标识名1 画 箭头 到 标识名2.|
	    \end{itemize}
	    
\section{自定义环境}
		\begin{itemize}
			\item  \verb|\newtheorem{thm}{定理}|
			\item  \verb|\begin{\thm}[勾股定理]|
			\item  	\begin{thm}[勾股定理]
				$A_2$
			\end{thm}	
		\end{itemize}
\section{大纲}:文档设计工作主要在导言区通过引入宏包、定义命令和设置参数来完成。
		\begin{itemize}
			\item 引入宏包:如\verb|sepackage[nottoc]{tocbibind}|
			\item 标题作者:
			\item 新建环境:如\verb|\newenvironment{myquote}{\begin{quote} \kaishu \zihao{-5} } {\end{quote}}|
			
			\item 新建命令:如\verb|\newcommand \degree{\circ}|
				\begin{lstlisting}
	\newcommand <命令> [<参数个数>][<首参默认值>]{<具体定义>}
	参数个数可以是1到9
	使用各参数分别使用#1, #2...
				\end{lstlisting}
			
			\verb|\loves{1}{2}| --> \loves{1}{2}
			
			\verb|\love{2}{3}| --> \love{2}{3}
			
			\verb|\love[第一个参数]{2}{3}| --> \love[第一个参数]{2}{3}
			
			\item 新建命令:\url{http://blog.163.com/chen_dawn/blog/static/112506320138762130518/}
			
		\end{itemize}                                                               
		
\section{抄录命令}
	  \verb|\verb 符号  内容  符号|
	   
\section{让不同的行在指定的地方对齐}
	   \begin{itemize}
		   	\item \verb|\\|分隔行与行
		   	\item  \verb|\=|设置制表位
		   	\item  \verb|\>|跳到设定好的下一个制表位
		   	\item  \verb|\‘|使它前后的文字以当前制表位为中心对齐
		   	\item  \verb|\’|使后面的文字右对齐
		   	\item  \verb|\<|跳到设定好的上一个制表位
		   	\item  \verb|\+|后面的行开始都右跳一个制表位
		   	\item  \verb|\-|后面的行开始都左跳一个制表位
		   	\item  \verb|\kill|忽略这一行的内容, 只保留制表位的设置。		   			   	
		   	\item  \verb|\hspace{3em} 类似于 \quad|不过可以指定几个字的距离
	   \end{itemize}
 	
	 	
\section{algorithm 环境配置}
		\begin{itemize}
			\item  在latex 安装目录查找algorithmic.sty
			\item  添加自己的命令,或修改原来的命令
			
			\item  配置参考:\url{http://blog.sina.com.cn/s/blog\_5a13c966010167fj.html}
			\item  使用参考:\url{http://www.cnblogs.com/piags/archive/2012/11/06/2757683.html}
		\end{itemize}	

\section{样式修改}
		\url{http://blog.renren.com/share/11982/4080240272}     
\end{document} 
 		    