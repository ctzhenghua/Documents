\documentclass[UTF8,a4paper,12pt]{ctexart} 

\usepackage{graphicx}%学习插入图
\usepackage{verbatim}%学习注释多行
\usepackage{booktabs}%表格
\usepackage{longtable} 
\usepackage{geometry}%图片
\usepackage{amsmath}
\usepackage{amssymb}
\usepackage{listings}%代码
\usepackage{xcolor}  %颜色
\usepackage{enumitem}%列表格式
\setenumerate[1]{itemsep=0pt,partopsep=0pt,parsep=\parskip,topsep=5pt}
\setitemize[1]{itemsep=0pt,partopsep=0pt,parsep=\parskip,topsep=5pt}
\setdescription{itemsep=0pt,partopsep=0pt,parsep=\parskip,topsep=5pt}

\usepackage{tcolorbox}
\usepackage{algorithm}  %format of the algorithm
\usepackage{algorithmic}%format of the algorithm
\usepackage{multirow}   %multirow for format of table
\usepackage{tabularx} 	%表格排版格式控制
\usepackage{array}	%表格排版格式控制
\usepackage{hyperref}

\CTEXsetup[format+={\flushleft}]{section}
%%%% 下面的命令添加新字体 %%%%%


%%%%%% 设置字号 %%%%%%
\newcommand{\chuhao}{\fontsize{42pt}{\baselineskip}\selectfont}
\newcommand{\xiaochuhao}{\fontsize{36pt}{\baselineskip}\selectfont}
\newcommand{\yihao}{\fontsize{28pt}{\baselineskip}\selectfont}
\newcommand{\erhao}{\fontsize{21pt}{\baselineskip}\selectfont}
\newcommand{\xiaoerhao}{\fontsize{18pt}{\baselineskip}\selectfont}
\newcommand{\sanhao}{\fontsize{15.75pt}{\baselineskip}\selectfont}
\newcommand{\sihao}{\fontsize{14pt}{\baselineskip}\selectfont}
\newcommand{\xiaosihao}{\fontsize{12pt}{\baselineskip}\selectfont}
\newcommand{\wuhao}{\fontsize{10.5pt}{\baselineskip}\selectfont}
\newcommand{\xiaowuhao}{\fontsize{9pt}{\baselineskip}\selectfont}
\newcommand{\liuhao}{\fontsize{7.875pt}{\baselineskip}\selectfont}
\newcommand{\qihao}{\fontsize{5.25pt}{\baselineskip}\selectfont}

%%%% 段落首行缩进两个字 %%%%
\makeatletter
\let\@afterindentfalse\@afterindenttrue
\@afterindenttrue
\makeatother
\setlength{\parindent}{2em}  %中文缩进两个汉字位


%%%% 下面的命令重定义页面边距,使其符合中文刊物习惯 %%%%
\addtolength{\topmargin}{-54pt}
\setlength{\oddsidemargin}{0.63cm}  % 3.17cm - 1 inch
\setlength{\evensidemargin}{\oddsidemargin}
\setlength{\textwidth}{14.66cm}
\setlength{\textheight}{24.00cm}    % 24.62

%%%% 下面的命令设置行间距与段落间距 %%%%
\linespread{1.4}
\setlength{\parskip}{0.5\baselineskip}

%%%% 下面的命令定义图表、算法、公式 %%%%
\newcommand{\EQ}[1]{$\textbf{EQ:}#1\ $}
\newcommand{\ALGORITHM}[1]{$\textbf{Algorithm:}#1\ $}
\newcommand{\Figure}[1]{$\textbf{Figure }#1\ $}

%%%% 下面命令改变图表下标题的前缀 %%%%% 如:图-1、Fig-1
\renewcommand{\figurename}{Fig}

\geometry{left=1.6cm,right=1.8cm,top=2cm,bottom=1.7cm} %设置文章宽度

\pagestyle{plain} 		  %设置页面布局

%代码效果定义
\definecolor{mygreen}{rgb}{0,0.6,0}
\definecolor{mygray}{rgb}{0.5,0.5,0.5}
\definecolor{mymauve}{rgb}{0.58,0,0.82}
\lstset{ %
	backgroundcolor=\color{white},   % choose the background color
	basicstyle=\footnotesize\ttfamily,        % size of fonts used for the code
	columns=fullflexible,
	breaklines=true,                 % automatic line breaking only at whitespace
	captionpos=b,                    % sets the caption-position to bottom
	tabsize=4,
	commentstyle=\color{mygreen},    % comment style
	escapeinside={\%*}{*)},          % if you want to add LaTeX within your code
	keywordstyle=\color{blue},       % keyword style
	stringstyle=\color{mymauve}\ttfamily,     % string literal style
	frame=single,					%tb top and bottom; L left double line
	xleftmargin=.06\textwidth, 
	%xrightmargin=.1\textwidth,
	rulesepcolor=\color{red!20!green!20!blue!20},
	% identifierstyle=\color{red},
	language={[5.1]Lua},
}

\setcounter{tocdepth}{4}	
\author{\kaishu 郑华}
\title{Lua 学习笔记}

\begin{document} 
 	\maketitle

\newpage
\section{变量}
	\paragraph{赋值与定义}
		不需要指明类型,系统会根据给出的值进行确定
		
		除此之外可以 如下赋值
		 \begin{lstlisting}
	a, b, c = 0, 1, 2    -->a=0, b=1, c=2
		 \end{lstlisting}
		 
	\paragraph{局部变量}
		局部变量加个 local 关键字,不是则不加
\newpage  

\section{控制结构}
	\paragraph{for}\verb|->|
		 \begin{lstlisting}
	for i = 1, 10, 1 do      --初始值   结束值   步长(+x)
	 print(i)
	end
	
	
	--范型 for 循环	  
	-- print all values of array 'a'
	for i,v in ipairs(a) do print(v) end       --i为当前的下标   v为当前下标的临时局部值
	
	-- print all keys of table 't
	for k in pairs(t) do print(k) end
	
	--[[
	1. 控制变量是局部变量
	2. 不要修改控制变量的值
	--]]
		 \end{lstlisting}

	\paragraph{while}\verb|->|
		 \begin{lstlisting}
	while i<x do
		print(i)
	end
		 \end{lstlisting}
	 
	 \paragraph{if}\verb|->|
		  \begin{lstlisting}
	if i<x then
		print(i)
	else
		print(i+1)
	end
		  \end{lstlisting}
	\paragraph{return}\verb|->|
		可以返回多个变量
		 \begin{lstlisting}
	function func(valueUsed)  
		return  returbValueOne,returnValueTwo... -->有几个返回值写几个
		 \end{lstlisting}
		 
	\paragraph{可变参数}Lua 函数可以接受可变数目的参数,和 C 语言类似在函数参数列表中使用三点(...)
	表示函数有可变的参数。Lua 将函数的参数放在一个叫 arg 的表中,除了参数以外,arg表中还有一个域 n 表示参数的个数。
		 \begin{lstlisting}
	--有时候我们可能需要几个固定参数加上可变参数
	function g (a, b, ...) end
	
	--[[ 
	CALL PARAMETERS
	g(3) a=3, b=nil, arg={n=0}
	g(3, 4) a=3, b=4, arg={n=0}
	g(3, 4, 5, 8) a=3, b=4, arg={5, 8; n=2}
	--]]
	
	--重写 print 函数:
	printResult = ""
	function print(...)
		for i,v in ipairs(arg) do
			printResult = printResult .. tostring(v) .. "\t"
		end
		printResult = printResult .. "\n"
	end
		 \end{lstlisting}
\newpage
\section{字符串}
 	 \paragraph{连接 用符号- ..   }\verb|->|
	 		 \begin{lstlisting}
	print("Hello" .. 'HH')    --> HelloHH
	 		 \end{lstlisting}
 		 
 	 \paragraph{字符到数字的智能转换}\verb|->|
	 		  \begin{lstlisting}
	print("10"+11)    --> HelloHH
	 		  \end{lstlisting}
	 		  
	 \paragraph{字符串查找替换}会全部替换
			  \begin{lstlisting}
	a = "one string one"
	b = string.gsub(a,"one","other")
	
	print(a)       -->one string one
	print(b)	   -->other string other
		  	  \end{lstlisting}
	 		  
	 \paragraph{注释}\verb|->|
		 单行注释 是“--”
		 
		 多行注释 是“--[[      --]]”
		 
	 \paragraph{类型函数type}\verb|->|
		   \begin{lstlisting}
	print(type("Hello World"))    --> string
	print(type(10.4*3))           --> number
	print(type(print))            --> function
	print(type(true))             --> boolean
	print(type(nil))              --> nil
		   \end{lstlisting}
	\paragraph{查找函数 find}\verb|->|
		 \begin{lstlisting}
	s,e = string.find("Hello Lua World!","World")  -->Lua可以返回多个变量
				 
	print(s,e)  --->s为目标字符串在给定字符串的起始位置,e则为终止位置   11  15
		 \end{lstlisting}
\newpage
\section{数组或表}
	\paragraph{下标从1开始不是0}\verb|->|
			\begin{lstlisting}
	days={"Sunday", "Monday", "Tuesday", "Wednesday",
	"Thursday", "Friday", "Saturday"}
	
	print(days[4]) --> Wednesday
	
	
	tab = {sin(1), sin(2), sin(3), sin(4),
	sin(5),sin(6), sin(7), sin(8)}
	
	a = {x=0, y=0} <--> a = {}; a.x=0; a.y=0
	
	w = {x=0, y=0, label="console"}
	x = {sin(0), sin(1), sin(2)}
	w[1] = "another field"
	
	--不管用何种方式创建 table,我们都可以向表中添加或者删除任何类型的域,构造函数仅仅影响表的初始化。
	x.f = w							--原来没有的东西如果写出来,则自动会加进到X
	
	print(w["x"]) 				    --> 0  即访问的是 表中的X项,如果是w[x] -->nil
	print(w[1]) --> another field
	print(x.f[1]) --> another field
	w.x = nil -- remove field "x"X
			\end{lstlisting}
	\paragraph{为什么有的东西可以用下标访问,有些不能}
			原来,只有在表不提供任何关键字时,才会按照下标进行寻找,否则,必须按照提供的关键字访问
			 \begin{lstlisting}
	 local a = {x = 12, mutou = 99, [3] = "hello"}
	 print(a["x"]);
	 
	 local a = {x = 12, mutou = 99, [3] = "hello"}
	 print(a.x);
	 
	 
	 local a = {[1] = 12, [2] = 43, [3] = 45, [4] = 90}
	 
	 --如果说,大家习惯了数组,用数字下标,又不想自己一个个数字地定义,比如:
	 local a = {12, 43, 45, 90}
	 print(a[1]);
			  \end{lstlisting}
\newpage
\section{Function}
	\paragraph{定义与使用函数}\verb|->|
		 \begin{lstlisting}
	function fact(n)
		if n == 0 then
			return 1
		else
			return n*fact(n-1)
		end
	end
		
	print("enter a number")
	
	a = io.read("*number")
	print(fact(a))
		 \end{lstlisting}
\newpage
\section{Lua 调用其他文件}
	\paragraph{调用Lua}\verb|->|
	 \begin{lstlisting}
	--lib.lua file
	function norm(x,y)
		local n2 = x^2 + y^2
		return math.sqrt(n2)
	end
	
	function twice(x)
		return 2*x
	end
	
	--lua 调用file
	dofile("lib.lua")
	n = norm(3.4,1.0)
	print(twice(n))	 
	 \end{lstlisting}
\end{document} 
 		    