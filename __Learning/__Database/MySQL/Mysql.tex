\documentclass[UTF8,a4paper,12pt]{ctexbook} 

 \usepackage{graphicx}%学习插入图
 \usepackage{verbatim}%学习注释多行
 \usepackage{booktabs}%表格
 \usepackage{geometry}%图片
 \usepackage{amsmath}
 \usepackage{amssymb}
 \usepackage{listings}%代码
 \usepackage{xcolor}  %颜色
 \usepackage{enumitem}%列表格式
 \usepackage{tcolorbox}
 \usepackage{algorithm}  %format of the algorithm
 \usepackage{algorithmic}%format of the algorithm
 \usepackage{multirow}   %multirow for format of table
 \usepackage{longtable} 
 \usepackage{tabularx} 	%表格排版格式控制
 \usepackage{array}	%表格排版格式控制
 \usepackage{hyperref} %超链接 \url{URL}
 % \setCJKmainfont{方正兰亭黑简体}  %中文字体设置
 % \setCJKsansfont{华康少女字体} %设置中文字体
 % \setCJKmonofont{华康少女字体} %设置中文字体

 \CTEXsetup[format+={\flushleft}]{section}

 \geometry{left=1.6cm,right=1.8cm,top=2cm,bottom=1.7cm} %设置文章宽度
 
 \pagestyle{plain} 		  %设置页面布局

 %代码效果定义
 \definecolor{mygreen}{rgb}{0,0.6,0}
 \definecolor{mygray}{rgb}{0.5,0.5,0.5}
 \definecolor{mymauve}{rgb}{0.58,0,0.82}
 \lstset{ %
 	backgroundcolor=\color{white},   % choose the background color
 	basicstyle=\footnotesize\ttfamily,      % size of fonts used for the code
 	%stringstyle=\color{codepurple},
 	%basicstyle=\footnotesize,
 	%breakatwhitespace=false,         
 	%breaklines=true,                 
 	%captionpos=b,                    
 	%keepspaces=true,                 
 	%numbers=left,                    
 	%numbersep=5pt,                  
 	%showspaces=false,                
 	%showstringspaces=false,
 	%showtabs=false,        
 	columns=fullflexible,
 	breaklines=true,                 % automatic line breaking only at whitespace
 	captionpos=b,                    % sets the caption-position to bottom
 	tabsize=4,
 	commentstyle=\color{mygreen},    % comment style
 	escapeinside={\%*}{*)},          % if you want to add LaTeX within your code
 	keywordstyle=\color{blue},       % keyword style
 	xleftmargin=.06\textwidth, 
 	stringstyle=\color{mymauve}\ttfamily,     % string literal style
 	frame=single,
 	rulesepcolor=\color{red!20!green!20!blue!20},
 	% identifierstyle=\color{red},
 	language=c++,
 }
 \author{\kaishu 郑华}
 \title{\heiti MySQL 数据库学习笔记}
 
\begin{document}          %正文排版开始
 	\maketitle
 \chapter{基础概念}
	 \section{术语}
		 \begin{itemize}
		 	\item  \textbf{数据库}: 数据库是一些关联表的集合。.
		 	\item  \textbf{数据表}: 表是数据的矩阵。在一个数据库中的表看起来像一个简单的电子表格。
		 	\item  \textbf{列}: 一列(数据元素) 包含了相同的数据, 例如邮政编码的数据。
		 	\item  \textbf{行}:一行(=元组,或记录)是一组相关的数据,例如一条用户订阅的数据。
		 	\item  \textbf{冗余}:存储两倍数据,冗余降低了性能,但提高了数据的安全性。
		 	\item  \textbf{主键}:主键是唯一的。一个数据表中只能包含一个主键。你可以使用主键来查询数据。
		 	\item  \textbf{外键}:外键用于关联两个表。
		 	\item  \textbf{复合键}:复合键(组合键)将多个列作为一个索引键,一般用于复合索引。
		 	\item  \textbf{索引}:使用索引可快速访问数据库表中的特定信息。索引是对数据库表中一列或多列的值进行排序的一种结构。类似于书籍的目录。
		 	\item  \textbf{参照完整性}: 参照的完整性要求关系中不允许引用不存在的实体。与实体完整性是关系模型必须满足的完整性约束条件,目的是保证数据的一致性。
		 \end{itemize}
		
	 
 \chapter{数据库基本操作}
 
	 \section{SQL 分类}
		\begin{itemize}
			\item 数据库查询:代表关键字 \verb|select|
			\item 数据库操纵:代表关键字 \verb|insert delete update|
			\item 数据库定义:代表关键字 \verb|create drop   alter|
			\item 事务控制:代表关键字 \verb|commit rollback|
			\item 数据控制:代表关键字 \verb|grant, revoke|
		\end{itemize} 
	
	\section{常用命令}
		\subparagraph{显示当前的数据库们} \verb|show databases;|
		\subparagraph{使用某个数据库} \verb|use databaseName;|
		\subparagraph{显示数据库中的表们} \verb|show tables;|
		\subparagraph{查看表的创建语句} \verb|show create table tableName;|
		\subparagraph{查看表的结构} \verb|desc tableName;|
		\subparagraph{select version();} 显示当前数据库管理系统的版本。
		\subparagraph{重名命结果} \verb|as| ,如 \verb|select lower(ename) as E from emp;|
		\subparagraph{创建数据库} \verb|create database Name;|
		\subparagraph{终止一条语句} \verb|\c|
		\subparagraph{退出管理系统} \verb|exit Or quit Or ctrl+c|	
	\section{查询语句}
		\subsection{条件查询}
				\begin{table}[H]
					\centering
					\caption{查询符号}
					\begin{longtable}{c|m{10cm}}
						\hline
						运算符   &   功能说明\\
						\hline
						\verb|=| &  等于 \\
						\verb|!=| & 不等于 \\
						\verb|between ... and ..| & 等同于 \verb|>= ... and <= ...| \\
						\verb|is null| & 为\verb|null(is not null 不为空)| \\
						\verb|and| & 并且 \\
						\verb|or| & 或者\\
						\verb|in| & 包含,相当于多个\verb|or|,(not in 不在这个范围内)\\
						\verb|not| & 取非\\
						\verb|like| & 为模糊查询,支持\verb|%|或\verb|_|匹配,其中\verb|%|匹配任意个字符,\verb|_|只匹配一个字符\\
						\hline	
					\end{longtable}
				\end{table}
				
		\verb|example ->| 
			\begin{lstlisting}
// 执行顺序
	select  // 3
		xx, xx2, xx3
	 from   // 1
		XX
	 where  // 2
		 xx = xx;
		 
// in 示例	查找job是什么的,不是什么的	 
	select 
		ename,job
	from
		emp
	where
		job in('MANAGER','SALESMAN');
		
	select 
		ename,job
	from
		emp
	where
		job not in('MANAGER','SALESMAN');

// like 示例, 查找以S 开头的名字
	select 
		ename
	from 
		emp
	where
		ename like 'S%'	
			\end{lstlisting}
			
		\subsection{排序}
			\verb|order  by|
		
			\begin{lstlisting}
// order  示例 默认升序,(desc 降序)
select 
	ename,salary
from
	emp
order by
	salary

// 按照第几个字段排序	
select 
	ename,salary // 1,2 字段
from
	emp
order by
	2  // 第2个字段
	
// 多个字段排序, ename 升序,salary 降序,使用逗号分割
select 
	ename,salary
from
	emp
order by
	salary desc, ename
			\end{lstlisting}
		
		
		\subsection{数据处理函数(单行)}处理单行后结束
			\begin{itemize}[itemindent = 2em]
				\item \verb|lower| :转换小写
				\item \verb|upper| :转换大写
				\item \verb|substr| :取子串(被截取的串,起始位置,截取长度)
				\item \verb|length| :取长度
				\item \verb|trim| :去空格
				\item \verb|round| :四舍五入
				\item \verb|rand()| :生成随机数
				\item \verb|ifnull(xx, num)|: 可以将\verb|null| 值转换成一个具体值
			\end{itemize}
		
		
		\subsection{分组函数、聚合函数(多行)}
		
			处理多行后结束,自动忽略空值
			
			先分组,然后再执行分组函数, 而where在分组函数之前执行,所以不能
			
			\verb|where| 中不能出现分组函数
			
			\begin{itemize}[itemindent = 2em]
				\item \verb|count |:取得记录数
				\item \verb|sum |:求和
				\item \verb|avg |:求平均
				\item \verb|max |:取最大值
				\item \verb|min |:取最小值
			\end{itemize}
			
			\verb|distinct 去重关键字 -> select distinct job from emp;|  只能出现在所有\textbf{字段}的最前面
			
			\verb|select count(distinct job) from emp;|
			
		\subsection{分组查询}
			\subparagraph{group by}: 通过哪个或哪些字段进行分组,使用后select 后只能跟参与分组的字段和分组函数。
			
				\verb|example->| 找出每个工作岗位的最高薪水【先按照工作岗位分组,使用max 函数求每一组的最高工资】
				
					\begin{lstlisting}
// 先按照job 分组,然后对每一组使用max(salary) 求最大值。					
	select 			//3
		max(salary)
	from			//2
		emp;
	group by		//1
		job;	
		
// 结合where 限定分组前条件,即分组前过滤
	select
		job, max(sal)
	from 
		emp
	where 
		job != 'MANAGER'
	group by
		job;
	
					\end{lstlisting}
			
				\verb|example->| 找出每个工作岗位的平均薪水,要求显示平均薪水大于1500
				 where  处理不了
				
			\subparagraph{having} 与where 都是为了完成数据的过滤, where 和 having 后面都是添加过滤条件,where是 在group by 之前执行, 而having 是在group  by  后执行。
			
				\begin{lstlisting}
//上例子解法
select 
	job,avg(sal)
from 
	emp
group by
	job
having 
	avg(sal) >  1500;
 
				\end{lstlisting}
			
			
			\subsection{查询语句总结}
			
				\subparagraph{关键字顺序不能变}:
				
					\begin{lstlisting}
	select 
		...
	from
		...
	where
		...
	group by
		...
	having
		...
	order by
		...
					\end{lstlisting}
			
				\subparagraph{执行顺序}:
					\begin{enumerate}[itemindent = 2em]
						\item \verb|from| 从某张表中检索数据
						\item \verb|where| 经过某条件进行过滤
						\item \verb|group by| 然后分组
						\item \verb|having| 分组之后不满意再过滤
						\item \verb|select| 查询出来
						\item \verb|order by| 排序输出
					\end{enumerate}
		
\section{连接查询}
			查询的时候只从一张表检索数据称为单表查询
			
			在实际的开发中,数据并不是存储在一张表中的,是同时存储在多张表中,这些表和表之间存在关系,我们在检索的时候通常需要将多长表联合起来取得有效数据,这种多表查询被称为连接查询或者叫做跨表查询。
			
			连接查询根据连接方式可以分为如下方式:
				\begin{itemize}[itemindent = 2em]
					\item  内连接
						\begin{itemize}[itemindent = 3em]
							\item 等值连接
							\item 非等值连接
							\item 自连接
						\end{itemize}
						
					\item  外连接
						\begin{itemize}[itemindent = 3em]
							\item 左外连接
							\item 右外连接
						\end{itemize}
						
					\item  全连接【几乎不用】
				\end{itemize}
				
		\subsection{内连接}		
				查找两张表匹配的数据。
				
				A表和B表能够完全匹配的记录查询出来,被称为内连接。
				
				\subparagraph{别名的使用,内连接的等值连接} 在进行多表连接查询的时候,尽量给表起别名,这样效率高,可读性高
					\begin{lstlisting}
// 将表emp 用别名 e表示..	

// 查询员工名与其对应的部门名
	select	
		e.ename, d.dname
	from 
		emp e, dept d;
	where 
		e.depno = d.depno
		
// SQL99 语法,使得表连接独立出来了,结构更清晰
	select 
		e.ename, d.dname
	from 
		emp e
	join	  // 内连接的inner 可以省略 
	    dept d
	on 
		e.depno = d.depno;
					\end{lstlisting}
					
				\subparagraph{内连接的非等值连接} 范围
					\begin{lstlisting}
// 找出员工名, 薪水,与其的薪水等级
	select 	
		e.name, e.sal, s.grade
	from
		emp e
	join 
		salgrade s
	on e.sal >= s.lower and e.sal <= s.higher;  // 可以使用between and 替代
					\end{lstlisting}
				 
				
				\subparagraph{内连接的自连接} 自己与自己连接,将自己视为两张表
					\begin{lstlisting}
//  找出每一个员工的上级领导,要求显示员工名以及对应的领导名
	表结构:
	empno ename mgr
	7369  SMITH 7123

// 要点:将自己视为两张表
	select
		a.ename empname, b.ename leaderName
	from 
		emp a
	join
		emp b
	on 
		a.mgr = b.empno;
					\end{lstlisting}	
				
	\subsection{外连接}

		A表和B表能够匹配的记录查询出来之外,将其中一张表的记录完全无条件的完全查询出来,对方表没有匹配的记录,会自动模拟出NULL与之匹配。
		
		$$\verb|外连接查询的结构条数| >= \verb|内连接的查询结果数量|$$
		
		可以添加除了内连接外的其他数据。
		
		
		\verb|example ->|找出每一个员工对应的部门名称,并且显示所有部门名称,注意部门可能没有员工。
		
				
		\subparagraph{左外连接}
			\verb|select e.ename, d.dname  from dept d left join emp e on e.deptno = d.deptno;|
		
		\subparagraph{右外连接}
			\verb|select e.ename, d.dname  from emp e right join dept d on e.deptno = d.deptno;|// outer  省略
				
		
		\subparagraph{总结}
			希望将哪边表的数据完全显示出来, join 的前边的修饰词 \verb|right left| 可以恰好说明,如上,希望将dept表完全显示,那么\verb|先写dept| 的话,那么就在\verb|join 的左边|,就是 \verb|left join|.		
\chapter{高级操作} 
\section{主从模式-replication}

\section{集群}	
		    
\end{document} 
 		    